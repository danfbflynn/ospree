\documentclass[11pt,letter]{article}
\usepackage[top=1.00in, bottom=1.0in, left=1.1in, right=1.1in]{geometry}
\renewcommand{\baselinestretch}{1.2}
\usepackage{graphicx}
\usepackage{natbib}
\usepackage{amsmath}
\usepackage{amssymb} % for math symbols


\def\labelitemi{--}
\parindent=0pt
\begin{document} 
\bibliographystyle{/Users/Lizzie/Documents/EndnoteRelated/Bibtex/styles/besjournals} 

{\bf Notes by Lizzie on molecular pathways to spring events ...} \\

\emph{Chat with Loren Rieseberg} -- 16 Feb 2022 (orange notebook):
\begin{itemize}
\item  Floragen is a protein of FT (not a hormone that controls flowering, as everyone used to think it would be).
\item  Almost always see a shift in FT with species etc.
\item  His guess is most species have mltiple pathways to affect FTs (but maybe not all ... for example, the Silver Leaf sunflower he works on wants to flower really late, so maybe then one cue that nothing else can screw up is best). 
\item  For {\bf my paper} he suggests saying: there are lots of pathways, no rules, multiple explanations for one `phenomenon'---so what we need is an order of importance of factors. Parallelism is on the pathway, not the gene (except for FT, FLC). 
\item  Field flowering times are super noisy, compared to common garden flowering times (77 day shift in common garden that they found would be harder to see in natural systems).
\end{itemize}

\emph{Chat with Rob Guy} -- later afternoon 16 Feb 2022 (orange notebook):\\

{\bf To  do:} Look up Bholemius (worked with Bhalero), and papers that Rob sent. Look up VERN genes in winter wheat, do they exist in trees?\\
\begin{itemize}
\item  For chilling we don't know: (1) receptor, (2) signal conductor ... we know callose but that's downstream of the action. Whatever the mechanism is, it's a \emph{slow} one so look for changes over a long period. Probably multiple genes involved (McKown work).
\item  Best work on chilling is winter wheat (vernalization). 
\item  Vernalization to FT; FT relates to budset ... both these things are often conserved.
\item  Lateral buds are happily dormant for a long time (Karen Tanino work). 
\item  Rob's take-home from Cooke et al. 2012 paper: We know a little, but not much.
\item  Kalcits: Magnetic resonance microimaging indicates water diffusion correlates with dormancy induction in cultured hybrid poplar
(Populus spp.) buds -- restricted water movement seems to be associated with dormancy.
\end{itemize}

\emph{Papers that I have read/reviewed:} 

\cite{Maurya2017,McKown2018ecogen,Todesco2020,Azeez2021,Bennett2021,Satake2022}

\begin{enumerate} 
\item Utah model (Richardson 1974) -- this paper seriously just sort of formalizes what Erez \& Lavee 1971 had done in experiments into a model. It uses those experiments to suggest a model where there is optimum chill (around 6C) and some negative effects above 15C. They do an hourly to daily temperature change, then compare their model to a few bits of data and say it works. The paper is way less nuanced compared to  Erez \& Lavee 1971.
\item  Erez \& Lavee 1971 -- does three very useful experiments on peaches. 
\begin{enumerate} 
\item ``This is interpreted as the `chilling requirement' ...''
\item Interestingly they mention light might matter (need to get those papers) but also cite \emph{Fagus} studies. 
\item They show 6C is optimum for chill, in contrast to Weinbgerger who seemed to be pushing from anything below 45 F. 
\item They do a couple experiments to show alternating day/night temps don't matter if you have a model where 6C is optimum
\item They also look at effects of intermitten warm weather. 
\end{enumerate} 
\item \cite{McKown2018ecogen}
\begin{enumerate} 
\item ``Bud-break had a quadratic relationship with latitude, where southern- and northern-most provenances generally broke bud earlier than those from central parts of the species’ range''
\item ``variation in bud-break reflects different selection for winter chilling and heat sum accumulation''
\item Nicereview of the genetics on first full paragraph of second page. 
\item ``The chilling necessary to transition from endodormancy to ecodormancy is highly heritable in most species, indicating a strong genetic component.''
\item ``Lower total chilling require- ments are common in northern tree populations (Farmer \& Reinholt, 1986; Myking \& Heide, 1995; Junttila \& Hanninen, 2012).''
\end{enumerate} 
\item \cite{Luedeling:2011qe}
\begin{enumerate} 
\item \cite{Luedeling2009} seems the better paper, I should check it; it says: ``Ratios varied substantially between locations and between years. Particularly strong differences were detected between temperature regimes prevailing in a cold chamber (constant temperatures of 6C) and ambient orchard temperatures, indicating that chilling requirements determined by expos- ing trees to constant low temperatures may not be valid under field conditions (Luedeling et al. 2009e).''
\item this paper here is good for some simple bold statements about chilling, such as ...
\item ``...the physiological and genetic processes occurring in trees during winter chill accumulation are poorly understood. Models of winter chill accumulation are thus purely empirical and based on either field observations (e.g., Linkosalo et al. 2008) or controlled temperature experiments (e.g., Fishman et al. 1987a) rather than on a functional understanding of tree physiology.''
\item ``However, it is often historic precedence rather than systematic research that determines which model is used in a given growing region.''
\item ``The two main factors that have led researchers and growers to prefer one winter chill model over another are mean temperature of the site and continentality.''
\item ``Until much more functional understanding of these processes is gained, all winter chill models will only be proxies of the true biological processes.''
\end{enumerate} 
\item \cite{Azeez2021}
\begin{enumerate} 
\item Actually pretty damn readable for a gene paper!
\item ``EBB3 is a temperature-responsive, epigenetically regulated, positive regulator of bud-break that provides a direct link to activation of the cell cycle during bud-break.''
\item ``Thus, growth cessation and budset prior to dormancy establishment have co-opted genes and signaling cascades that regulate the photoperiodic floral initiation pathway.''
\item ``Specifically, under SDs, ABA concentration and signaling increase and promote the biosynthesis and deposition of callose at the plasmodesmata (PD) to develop obstructions known as PD sphincters15–18.''
\item ``How temperature controls dormancy release and bud-break is poorly understood at the molecular level.''
\item ``Expression of SVL is regulated by both SD photoperiod during dormancy initiation and low temperature during dormancy release29,33. Thus, SVL provides the regulatory link between the photoperiodic and thermo-signaling''
\item What they found and what to do next ... ``EBB1 acts at the top of the pathway and represses the expression of its direct downstream target SVL.' ... The identification of these regulatory interactions would benefit further from in planta analysis of their spatiotemporal specificity during dormancy-activity cycle''
\item ``Thus, SVL upregulation is a result of the combined and opposing regulatory activities of EBB1 and ABA. This situation is gradually reversed during dormancy release and bud-break .... Thus, EBB1 not only affects SVL expression but also likely indirectly plays a part in reducing the concentration of ABA during chilling to release apices from dormancy''
\item ``EBB3 is almost undetectable during dormancy establishment and significantly increases with the progression of the cold treatment peaking right around the budbreak time.''
\item Figure 7 reviews their hypothesized pathway. (Very handy.)
\item ``This correlation between repressive marks and expression is not perfect and indicates that EBB3 is under complex regulatory mechanism including but not limited to epigenetic modifications''
\item ``it is clear that there are multiple diverse regulatory layers that differentially act on the different components of the cascade.''
\end{enumerate} 
\item 
\begin{enumerate} 
\end{enumerate} 
\item \cite{Satake2022} 
\begin{enumerate} 
\item ``Monitoring gene expression in natural conditions at the population and community levels will help to unravel key genes associated with phenological traits and coordinated or synchronized expression of these genes across individuals in a population or species in a community.''
\item The heritability of the onset time of bud set and bud break was > 60\% in Populus trichocarpa for Swedish and Canadian populations (McKown et al., 2014; Richards et al., 2020).
\item Fig 2 (b) shows long-term and short-term cold exposure pathways for flowering (and see bottom of pg9 to 10)
\item Pg 6 DAM genes {\bf for Nacho's paper}
\item ``this temperature response pathway diverges across species in contrast to the photoperiod pathway, which is highly conserved among species.''
\end{enumerate} 
\item \cite{Chew:2012pd} 
\begin{enumerate} 
\item seems mainly about day/night temperature; using Wilcezk model (and hard to tell if just adding variation made the model run better) 
\item ``Our study suggests that different molecular pathways interact and predominate in natural environments that change seasonally.''
\item ``comprehensive ecological observatories will allow the collection of meteorological, physiological and genomic data on species with a range of adaptive strategies in their natural habitats.''
\end{enumerate} 
\item \cite{kudoh2016} 
\begin{enumerate} 
\item This is a weird article proprosing a lag hypothesis with some useful stuff; pg 402 (molecular phenology) suggests you can move away from event times to me
\item ``Studies introduced in this review indicate that plants can respond to seasonal temperatures by monitoring the long-term temperature  change and filtering short-term fluctuations.''
\item ``Growth chamber experiments in which phase lags are manipulated between different signals will be revealing.''
\end{enumerate} 
\item \cite{Maurya2017}
\begin{enumerate} 
\item Mostly reads as something for gene jockey folks to read one good sections on pg 357 
\item ``The opening of plasmodesmata and removal of dormancy sphincters is associated with dormancy release, which is initiated by prolonged exposure to low temperature in birch and hybrid aspen (Rinne and Van der Schoot, 1998; Rinne
et al., 2011). The opening of plasmodesmata could potentially restore the supply of growth-promotive signals to the meristem. However, these signals are currently unknown''
\item Also compares bud and seed dormancy!
\item ``One interesting insight from studies so far is how similar the photoperiodic signalling pathway in the control of growth cessation in boreal tree species is to the pathway that regulates flowering in annual plants such as arabidopsis.''
\end{enumerate} 
\end{enumerate} 

\bibliography{..//..//refs/ospreebibplus}


\end{document}
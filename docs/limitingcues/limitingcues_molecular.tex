\documentclass[11pt,letter]{article}
\usepackage[top=1.00in, bottom=1.0in, left=1.1in, right=1.1in]{geometry}
\renewcommand{\baselinestretch}{1.9}
\usepackage{graphicx}
\usepackage{natbib}
\usepackage{amsmath}
\usepackage{amssymb} % for math symbols


\def\labelitemi{--}
\parindent=0pt
\begin{document} 


Kalcits: Magnetic resonance microimaging indicates water diffusion correlates with dormancy induction in cultured hybrid poplar
(Populus spp.) buds -- restricted water movement seems to be associated with dormancy.


\cite{Satake2022} ``Monitoring gene expression in natural conditions at the population
and community levels will help to unravel key genes associated with
phenological traits and coordinated or synchronized expression of
these genes across individuals in a population or species in a
community.''

The heritability of
the onset time of bud set and bud break was > 60\% in Populus
trichocarpa for Swedish and Canadian populations (McKown et al.,
2014; Richards et al., 2020).

Fig 2 (b) shows long-term and short-term cold exposure pathways for flowering (and see bottom of pg9 to 10)

Pg 6 DAM genes -- > Nacho's paper


this temperature response pathway diverges across species in
contrast to the photoperiod pathway, which is highly conserved
among species.

\cite{Chew:2012pd} -- seems mainly about day/night temperature; using Wilcezk model (and hard to tell if just adding variation made the model run better) ``Our study suggests that different molecular pathways interact and predominate in natural
environments that change seasonally.''

``comprehensive ecological observatories will allow the collection
of meteorological, physiological and genomic data on species
with a range of adaptive strategies in their natural habitats.''


\cite{kudoh2016} This is a weird article proprosing a lag hypothesis with some useful stuff; pg 402 (molecular phenology) suggests you can move away from event times to me

``Studies introduced in this review indicate that plants can respond to
seasonal temperatures by monitoring the long-term temperature
change and filtering short-term fluctuations.''

``Growth chamber experiments in which phase lags are manipulated
between different signals will be revealing.''

\cite{Maurya2017} Mostly reads as something for gene jockey folks to read one good sections on pg 357 
``The opening of plasmodesmata and removal
of dormancy sphincters is associated with dormancy release,
which is initiated by prolonged exposure to low temperature in
birch and hybrid aspen (Rinne and Van der Schoot, 1998; Rinne
et al., 2011). The opening of plasmodesmata could potentially
restore the supply of growth-promotive signals to the meristem.
However, these signals are currently unknown''

Also compares bud and seed dormancy!

``One interesting insight from studies so far is
how similar the photoperiodic signalling pathway in the control
of growth cessation in boreal tree species is to the pathway that
regulates flowering in annual plants such as arabidopsis.''

\end{document}
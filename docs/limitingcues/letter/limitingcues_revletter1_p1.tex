\documentclass[11pt,a4paper]{letter}
\usepackage[top=1.00in, bottom=1.0in, left=1in, right=1in]{geometry}
\usepackage{graphicx}
\parindent=0pt


\begin{document}
    \begin{flushright}
\includegraphics[width=0.1\textwidth]{/Users/Lizzie/Documents/Professional/images/letterhead/ubc/UBClogo.jpg}
    \end{flushright}
\pagenumbering{gobble}
Dear Drs. Hetherington and Austin:
\vspace{1.5ex}\\
Please consider our manuscript, ``Integrating experiments to predict interactive cue effects on spring phenology with warming,'' for publication as a Research review in \emph{New Phytologist}. This is a revision of manuscript NPH-R-2021-38748.
\vspace{1.5ex}\\
This article draws on long-term observations, controlled environment experiments (including a meta-analysis of seven decades of controlled environment studies), and physiological advances at the molecular level to outline a path forward for studies of spring plant phenology. We show why a greater integration across fields will be critical for accurate forecasts of plant phenology. % Our review highlights how researchers could better harness the power of controlled environment experiments to transform our fundamental understanding of phenology and advance forecasting.  While understanding, modeling and predicting interactions among cues and their effects on phenology is challenging, it will yield more accurate predictions---with valuable implications to more realistically assess the effects of climate change on plant biodiversity, including agricultural and forest species.
\vspace{-1ex}\\
Three referees and the editor provided very helpful comments on the previously submitted manuscript. We extend many thanks for the positive tone of the comments and the remarkably consistent suggestions of how to improve the manuscript---specifically that we needed to incorporate more insights from recent advances in the molecular underpinnings of spring phenology and their potential to advance our current approaches to climate change research. In response, we have worked to address this throughout, with both small additions and several new paragraphs in places specifically suggested by the reviewers. We now include a new paragraph outlining our basic understanding early in the section `‘Why observational studies are not enough to robustly estimate cues,' and later on in a forward-looking section, `Incorporating our understanding of physiology into forecasts of phenology.' Additionally, we provide an extended discussion of new molecular insights into chilling in the Supporting Information. We believe the current version is greatly improved from our first submission because of these changes.
\vspace{1.5ex}\\
Upon acceptance for publication, data from a systematic literature review included in the paper will be freely available at KNB (knb.ecoinformatics.org); the full dataset is available to reviewers and editors upon request.  All authors substantially contributed to this work and approved of this version for submission. The manuscript is approximately 4650 words with 200 word summary, four figures, and 65 references. It is not under consideration elsewhere. We hope that you will find it suitable for publication in \emph{New Phytologist} and look forward to hearing from you.
\vspace{1.5ex}\\
Sincerely,\\

\includegraphics[scale=1]{/Users/Lizzie/Documents/Professional/Vitas/Signatures/SignatureLizzieSm.png} \\

Elizabeth M Wolkovich\\
Associate Professor of Forest \& Conservation Sciences\\ 
University of British Columbia\\


\end{document}



% \signature{Elizabeth M Wolkovich}
\address{Forest and Conservation Sciences\\
University of British Columbia\\
2424 Main Mall\\
Vancouver, BC V6T 1Z4}


% including the complexity in measuring it and how it may structure communities in stationary and non-stationary systems. We've been working on a version of the storage effect model that gives us some interesting insights via simulations and I think a Review \& Synthesis where we marry these results with some of the long-term and experimental data available now could help advance the field.




Controlled environment studies can help predict non-linear responses by allowing researchers to examine the effects of one cue with the others held constant, and examine interactive effects. 

Indeed, one of the major advantages of experiments is that they allow treatments outside of the historical range of a species' or region's climate---an option long-term observational data cannot provide.

Research on phenology had been conducted for centuries before anthropogenic climate change caused earlier budburst and leafout across much of the globe \citep{Lamb:1948aa,Sparks:1995mv}. Decades of controlled environment studies contributed to our fundamental understanding of the drivers of spring plant phenology: forcing (warm temperatures, generally occurring in the late winter and early spring), chilling (cool temperatures, generally occurring in the fall and winter), and photoperiod (daylength). \\

We argue that controlled environment experiments will be critical for accurate predictions of phenology given future warming. As such we reviewed controlled environment studies over the last seven decades to understand the range of treatments already available, and how they compare to current and future conditions.



\documentclass[11pt,a4paper]{article}
\usepackage[top=1.00in, bottom=1.0in, left=1in, right=1in]{geometry}
\usepackage{graphicx}
\usepackage{natbib}
\parindent=0pt


\begin{document}
\bibliographystyle{/Users/Lizzie/Documents/EndnoteRelated/Bibtex/styles/besjournals} % change to pnas to count refs
\renewcommand{\refname}{\CHead{}}
    \begin{flushright}
\includegraphics[width=0.1\textwidth]{/Users/Lizzie/Documents/Professional/images/letterhead/ubc/UBClogo.jpg}
    \end{flushright}

\vspace{1.5ex}\\
\pagenumbering{gobble}
Dear Dr. Hetherington:
\vspace{1.5ex}\\
Please consider our manuscript, ``Integrating experiments to predict interactive cue effects on spring phenology with warming,'' for publication as a Research review in \emph{New Phytologist}. This article draws on long-term observations, controlled environment experiments and physiological advances to outline a path forward for studies of spring plant phenology. 
\vspace{1.5ex}\\
Climate change has led to growing research in phenology---a fundamental process of plant biology that influences carbon storage and climate change itself. Much of this research, however, has progressed without a strong connection to the basic biology of phenology. This disconnect may explain why recent trends in spring phenology have failed to match predictions based on simple linear models \citep{fu2015,piao2017}. In contrast, decades of controlled environment studies show that responses will be non-linear because of the cues that determine spring plant phenology (forcing, chilling, and photoperiod).
\vspace{1.5ex}\\
We show why a greater integration across fields will be critical for accurate forecasts of plant phenology. Our review highlights how researchers could better harness the power of controlled environment experiments to transform our fundamental understanding of phenology and advance forecasting. Controlled environment studies can critically rule out, or support, hypotheses to explain observed discrepancies in long-term data and open up new pathways to understand current trends. While understanding, modeling and predicting interactions among cues and their effects on phenology is challenging, it will yield more accurate predictions---with valuable implications to more realistically assess the effects of climate change on plant biodiversity, including agricultural and forest species. 
\vspace{1.5ex}\\
This review includes a meta-analysis of seven decades of controlled environment studies to understand the cue-space (i.e., the possible range of each cue and interactions across cues) already studied, and how it compares to current and future conditions. Based on this, we outline a cross-disciplinary path forward where advances in physiology \citep[e.g.,][]{singh2019,chang2021}, greater integration of controlled environment studies with forecasting, and multi-species modeling can yield more robust predictions.
\vspace{1.5ex}\\
% Finally, we provide a framework to leverage existing ecological theory to understand how tracking in stationary and non-stationary systems may shape communities, and thus help predict the indirect consequences of climate change.
% % Climate change upends the assumption of stationarity. By causing increases in temperature, larger pulses of precipitation, increased drought, and more storms \citep{ipcc2013}, climate change has fundamentally shifted major attributes of the environment from stationary to non-stationary regimes.
Upon acceptance for publication, data from a systematic literature review included in the paper will be freely available at KNB (knb.ecoinformatics.org); the full dataset is available to reviewers and editors upon request. %This work includes a meta-analysis, so data have been previously published; however, the synthesis of these data and the tables, figures, models, and materials presented in this manuscript have not been previously published nor are they under consideration for publication elsewhere.
All authors substantially contributed to this work and approved of this version for submission. The manuscript is approximately 3955 words with 200 word summary, four figures, and 60 references. It is not under consideration elsewhere. We hope that you will find it suitable for publication in \emph{New Phytologist} and look forward to hearing from you.
\vspace{1.5ex}\\
Sincerely,\\

\includegraphics[scale=1]{/Users/Lizzie/Documents/Professional/Vitas/Signatures/SignatureLizzieSm.png} \\

Elizabeth M Wolkovich\\
Associate Professor of Forest \& Conservation Sciences\\ 
University of British Columbia\\

\emph{References:}
\vspace{-5ex}
\bibliography{..//..//..//refs/ospreebibplus}


\end{document}



% \signature{Elizabeth M Wolkovich}
\address{Forest and Conservation Sciences\\
University of British Columbia\\
2424 Main Mall\\
Vancouver, BC V6T 1Z4}


% including the complexity in measuring it and how it may structure communities in stationary and non-stationary systems. We've been working on a version of the storage effect model that gives us some interesting insights via simulations and I think a Review \& Synthesis where we marry these results with some of the long-term and experimental data available now could help advance the field.




Controlled environment studies can help predict non-linear responses by allowing researchers to examine the effects of one cue with the others held constant, and examine interactive effects. 

Indeed, one of the major advantages of experiments is that they allow treatments outside of the historical range of a species' or region's climate---an option long-term observational data cannot provide.

Research on phenology had been conducted for centuries before anthropogenic climate change caused earlier budburst and leafout across much of the globe \citep{Lamb:1948aa,Sparks:1995mv}. Decades of controlled environment studies contributed to our fundamental understanding of the drivers of spring plant phenology: forcing (warm temperatures, generally occurring in the late winter and early spring), chilling (cool temperatures, generally occurring in the fall and winter), and photoperiod (daylength). \\

We argue that controlled environment experiments will be critical for accurate predictions of phenology given future warming. As such we reviewed controlled environment studies over the last seven decades to understand the range of treatments already available, and how they compare to current and future conditions.



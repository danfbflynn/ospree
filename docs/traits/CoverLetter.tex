\documentclass[11pt,a4paper]{article}
\usepackage[top=1.00in, bottom=1.0in, left=1.1in, right=1.1in]{geometry}
\usepackage{graphicx}
\usepackage[sort&compress, super, numbers]{natbib}
\usepackage[export]{adjustbox}

% 50 word answers
\begin{document}

\pagenumbering{gobble}
%dlOct10: is "shape" to strong a word? It implies causation in my mind. Should we say "how budbrust timing relates to woody plant strategies and traits"?
\noindent \includegraphics[width=0.4\textwidth, right]{letterhead/Facultyofforestry.png}
\noindent Dear Dr. Hetherington:
\vspace{1.5ex}\\
\noindent Please consider our paper, ``Budburst timing within a functional trait framework'' for publication as a research article in \emph{New Phytologist}. We combine multiple global databases to show how woody plant budburst timing relates to a number of major functional plant traits, supporting foundational---but generally untested hypotheses---of the role of phenology within plant strategies. % emw -- I would not include 'strategies' and 'traits' in the title; I am also not so sure about what shapes what .... So maybe: 'The role of budburst timing in woody plant strategies' or 'Budburst timing within a functional trait framework' or something... 
\vspace{1.5ex}\\ 
\noindent Climate change is impacting species phenologies---timing of life history events---and reshaping ecological communities. Increasing research suggests these changes are linked, as species that shift the most phenologically with warming appear to also perform better. Yet research has failed to test whether species phenologies or their growth strategies drive this relationship. Teasing this apart has been challenging because the variability of plant phenology has led it to be systematically omitted from global trait frameworks that link to plant growth strategies. % and can help predict future community dynamics and ecosystem functioning. % emw -- I think you have way too much going on here and the sentences are pretty clunky ... I tried to simplify .. you could still cut: and can help predict future community dynamics and ecosystem functioning
\vspace{1.5ex}\\
\emph{What hypotheses or questions does this work address?} We test how differences in phenology relate to well-established relationships between plant growth strategies defined by major functional traits. We combine global experimental data on budburst and commonly measured traits with a cutting-edge Bayesian approach to test the relationship between budburst and traits in response to temperature and light cues.  %49
\vspace{1.5ex}\\
\emph{How does this work advance our current understanding of plant science?} We found strong evidence that plant phenologies are inextricably linked to their growth strategies, as was proposed in the 1970s but has rarely, if ever, been tested. Merging multiple global datasets, we found earlier species exhibited acquisitive traits---such as shorter maximum heights, and denser, lower nitrogen leaves---while later-active species are taller with low nitrogen leaves.
\vspace{1.5ex}\\
\emph{Why is this work important and timely?} These are some of the first findings to fit budburst phenology firmly within major functional trait frameworks. By teasing apart the underlying mechanisms shaping phenology and traits globally, this work can help predict how species and community-level growth strategies may shift alongside changing phenology with climate change. 
\vspace{1.5ex}\\
\noindent All authors contributed to this work and approved this version for submission. The manuscript is 3569 words with a 146 word summary, and 4 figures. It is not under consideration elsewhere. We hope you find it suitable for publication in \emph{New Phytologist}, and look forward to hearing from you. 
%\vspace{1.5ex}\\
%\noindent We recommend the following reviewers: Dr. Angela Moles,  Dr. Daniel. C. Laughlin, Dr. Maria Sporbert, Dr. Lee E. Frelich.
\vspace{1.5ex}\\
\noindent Sincerely, \\
\includegraphics[scale=.2]{letterhead/sigDL.png} \\
\noindent Deirdre Loughnan\\
\noindent Forest \& Conservation Sciences\\
\noindent University of British Columbia

\newpage

\end{document}

%\noindent Here we overcome these challenges by combining global data from experiments on budburst phenology and plant traits, with cutting-edge Bayesian approaches that can jointly model budburst timing in response to environmental cues and in relation to major plant traits. Our dataset represents one of the most comprehensive datasets of trait syndromes available, making it an important first step to identify general trends that scale across populations and species. Further, by using a joint modelling approach, we are the first to identify broader trait relationships to phenological cues based on species-level trait variation, while also accounting for the high degree of uncertainty that arises when combining datasets of diverse communities and locations.
%\vspace{1.5ex}\\
%\noindent Our findings show how traits and phenologies are inextricably linked to strategies for growth. Earlier species exhibit acquisitive traits---shorter maximum heights, and denser, lower nitrogen leaves---while later-active species are taller with low nitrogen leaves. Our results fit budburst phenology firmly within major functional trait frameworks, allowing us to tease apart the underlying mechanisms shaping species phenology and traits across communities, and provide novel insights that can be used to better predict how communities may shift in their growth strategies alongside changing phenology with climate change. 
%\vspace{1.5ex}\\

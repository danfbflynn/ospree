\documentclass[11pt,a4paper]{article}
\usepackage[top=1.00in, bottom=1.0in, left=1.1in, right=1.1in]{geometry}
\usepackage{graphicx}
\usepackage[sort&compress, super, numbers]{natbib}
\usepackage[export]{adjustbox}

\begin{document}

\pagenumbering{gobble}
%dlOct10: is "shape" to strong a word? It implies causation in my mind. Should we say "how budbrust timing relates to woody plant strategies and traits"?
\noindent \includegraphics[width=0.4\textwidth, right]{letterhead/Facultyofforestry.png}
\noindent Dear Dr. Hetherington:
\vspace{1.5ex}\\

\noindent Please consider our paper, ``How budburst timing shapes woody plant strategies and traits'' for publication as a full paper in \emph{New Phytologist}. 
\vspace{1.5ex}\\ 

\noindent Climate change is impacting species phenologies---timing of life history events---reshaping communities, and altering ecosystem functioning \citep{Cleland2007a,Beard2019,Gu2022}. Increasing research suggests these changes are linked, as species that shift the most phenologically with warming appear to alter the assembly of communities and out-compete later species, either through their phenologies or growth strategies. Understanding this is critical to accurate forecasts, but extremely challenging because of high variability in observational phenology. The result is that plant phenology has been systematically omitted from global trait frameworks that link to plant growth strategies and can help predict future community dynamics and ecosystem functioning.
\vspace{1.5ex}\\

\emph{What hypotheses or questions does this work address?} We test the relationships between commonly measured woody plant traits to phenological cues. We combine global data from experiments on budburst phenology and plant traits, with cutting-edge Bayesian approaches, allowing us to overcome the challenges of the hight variability in phenology observations and account for the high degree of uncertainty that arises when combining datasets of diverse communities and locations.

\emph{How does this work advance our current understanding of plant science?} 


Our findings provide the strongest evidence for how plant traits and phenologies are inextricably linked to strategies for growth. Using one of the most comprehensive datasets of trait syndrome available, we found phenology to follow similar gradients in growth strategies as those associated with the leaf economic spectrum. We found earlier species to exhibit acquisitive traits---such as shorter maximum heights, and denser, lower nitrogen leaves---while later-active species are taller with low nitrogen leaves. 


making it an important first step to identify general trends that scale across populations and species. 

\emph{Why is this work important and timely?} Our results fit budburst phenology firmly within major functional trait frameworks to allow us to tease apart the underlying mechanisms shaping species phenology and traits across communities, and provide novel insights that can be used to better predict how communities may shift in their growth strategies alongside changing phenology with climate change. 


provide novel insights that can be used to better predict how communities may shift in their growth strategies alongside changing phenology with climate change. 


\noindent We overcome these challenges by combining global data from experiments on budburst phenology and plant traits, with cutting-edge Bayesian approaches that can jointly model budburst timing in response to environmental cues and in relation to major plant traits. Our dataset represents one of the most comprehensive datasets of trait syndrome available, making it an important first step to identify general trends that scale across populations and species. Further, by using a joint modelling approach, we are the first to identify broader trait relationships to phenological cues based on species-level trait variation, while also accounting for the high degree of uncertainty that arises when combining datasets of diverse communities and locations.

\vspace{1.5ex}\\
\emph{How does this work advance our current understanding of plant science?} We found strong evidence that plant phenologies are inextricably linked to their growth strategies, as was proposed in the 1970s but has rarely, if ever, been tested. Merging multiple global datasets, we found earlier species exhibited acquisitive traits---such as shorter maximum heights, and denser, lower nitrogen leaves---while later-active species are taller with low nitrogen leaves.
\vspace{1.5ex}\\
\noindent All authors contributed to this work and approved this version for submission. The manuscript is XXX words with a ZZZ word summary, and X figures. It is not under consideration elsewhere. We hope you find it suitable for publication in \emph{New Phytologist}, and look forward to hearing from you. 

\vspace{1.5ex}\\
\noindent Sincerely, \\
\includegraphics[scale=.2]{letterhead/sigDL.png} \\
\noindent Deirdre Loughnan\\
\noindent Zoology\\
\noindent University of British Columbia

\newpage

\end{document}

%\noindent Here we overcome these challenges by combining global data from experiments on budburst phenology and plant traits, with cutting-edge Bayesian approaches that can jointly model budburst timing in response to environmental cues and in relation to major plant traits. Our dataset represents one of the most comprehensive datasets of trait syndromes available, making it an important first step to identify general trends that scale across populations and species. Further, by using a joint modelling approach, we are the first to identify broader trait relationships to phenological cues based on species-level trait variation, while also accounting for the high degree of uncertainty that arises when combining datasets of diverse communities and locations.
%\vspace{1.5ex}\\
%\noindent Our findings show how traits and phenologies are inextricably linked to strategies for growth. Earlier species exhibit acquisitive traits---shorter maximum heights, and denser, lower nitrogen leaves---while later-active species are taller with low nitrogen leaves. Our results fit budburst phenology firmly within major functional trait frameworks, allowing us to tease apart the underlying mechanisms shaping species phenology and traits across communities, and provide novel insights that can be used to better predict how communities may shift in their growth strategies alongside changing phenology with climate change. 
%\vspace{1.5ex}\\

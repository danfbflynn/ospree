\documentclass{article}\usepackage[]{graphicx}\usepackage[]{color}

\usepackage{alltt}
\usepackage{float}
\usepackage{graphicx}
\usepackage{tabularx}
\usepackage{siunitx}
\usepackage{amssymb} % for math symbols
\usepackage{amsmath} % for aligning equations
\usepackage{textcomp}
\usepackage{booktabs}
\usepackage{mdframed}
\usepackage{natbib}
\usepackage[colorinlistoftodos]{todonotes} % to make comments on the margin
\usepackage[small]{caption}
\setlength{\captionmargin}{30pt}
\setlength{\abovecaptionskip}{0pt}
\setlength{\belowcaptionskip}{10pt}
\topmargin -1.5cm        
\oddsidemargin -0.04cm   
\evensidemargin -0.04cm
\textwidth 16.59cm
\textheight 21.94cm 
%\pagestyle{empty} %comment if want page numbers
\parskip 7.2pt
\renewcommand{\baselinestretch}{1.5}
\parindent 0pt
%\usepackage{lineno}
%\linenumbers

%% R Script

\title{Woody plant phenological responses are strongly associated with key functional traits- Outline}

\begin{document}

\maketitle

\noindent Authors:\\
The Wolkovich Lab in 2021 $^{1,2,3,4}$
\vspace{2ex}\\
\emph{Author affiliations:}\\
$^{1}$Forest \& Conservation Sciences, Faculty of Forestry, University of British Columbia, 2424 Main Mall, Vancouver, BC V6T 1Z4;\\
$^{2}$Arnold Arboretum of Harvard University, 1300 Centre Street, Boston, Massachusetts, USA;\\
$^{3}$Organismic \& Evolutionary Biology, Harvard University, 26 Oxford Street, Cambridge, Massachusetts, USA;\\
$^{4}$Edificio Ciencias, Campus Universitario 28805 Alcalá de Henares, Madrid, Spain\\
 

\vspace{2ex}
$^*$Corresponding author: deirdre.loughnan@alumni.ubc.ca\\
\renewcommand{\thetable}{\arabic{table}}
\renewcommand{\thefigure}{\arabic{figure}}
\renewcommand{\labelitemi}{$-$}
\setkeys{Gin}{width=0.8\textwidth}

Climate change is altering the timing of species phenologies, with such changes in temporal niches reshaping ecological communities and interactions between species. In temperate systems, the observed advances in plant phenological events, such as budburst, leafout, and flowering times, are associated with changes in seasonal temperatures, particularly warming winter and spring conditions \citep{Menzel2006,Fitter2002}. But despite this strong general trend, phenological responses vary across species and geographically, and we have yet to fully understand the underlying mechanisms driving observed differences \citep{Chuine2010,Morin2009}. As the effects of climate change become more pronounced, understanding these relationships is of increasing importance if we are to predict and preserve the diversity and services found in temperate forest ecosystems. %The relationships between environmental cues and phenological events define for example, the duration of the growing season, the trajectory of community assembly, species ranges, and ecosystem services \citep{Cleland2007,Lopez2008,Chuine2010}, ultimately shaping and mitigating the impact of climate change on forest communities.

Over several decades of work has identified three cues -- chilling, forcing, and photoperiod -- to be the primary proximate drivers of budburst and leafout in temperate deciduous species \citep{Chuine2016}. For budburst to occur, species must experience extended period of cold temperatures to break dormancy \citep{Cooke2012}, with species with higher chill requirements budbursting later in the season. Spring forcing temperatures, or the temperatures needed to cue species to initiate growth after dormancy release, are also changing as temperatures warm and the timing at which suitable temperature thresholds are met occur earlier within the season (citation). Photoperiod cues can also determine some species ability to initiate growth \citep{Basler2014,Zohner2020}, however, species with strong photoperiod requirements are expected to be more constrained in their ability to track changes in temperature and may face fitness costs and novel species interactions as a result \citep{Guy2014}. Previous studies support the general trend of advancing budburst in response to each cue \citep{Flynn2018}, but with considerable variation in the relative importance of different cues across species \citep{Chuine2016,Flynn2018}. Some woody plant species, for example, require less forcing to budburst after experiencing a cool winter with more chilling, while also having the ability to compensate for low chilling with high forcing conditions or longer photoperiods \citep{Laube2014,Harrington2015,Flynn2018,Caffarra2011,Basler2014,Zohner2016}. Evidence for the role of photoperiod is largely species specific  \citep{Heide1993, Basler2014, Singh2017, Zohner2016}, with few studies testing for its importance across species in a community (but see  \cite{Flynn2018}). Species that are less dependent on photoperiod cues and able to track trends in temperatures may benefit from greater intra-annual phenotypic plasticity resulting in greater fitness outcomes under increasingly variable climates (citation). Despite the insights that identifying these proximate drivers have provided, we still lack the generalizable and mechanistic understanding of why species and populations differ in their cue use that is needed to predict future changes in species sensitivities and community structures.

In our efforts to understand variation spring phenologies, several potential mechanisms have been tested to identify the drivers of species cue use. Work exploring drivers of intraspecific cue use, for example, has found age or the development stage of woody plants to be an important factor. Younger life stages, including both seedlings and younger understory trees both budburst earlier than mature individuals in the canopy \citep{Vitasse2013,Seiwa1991}. These trends reflect both differences in the temperature sensitivities across life stages and effects of ontogenic changes as trees mature \citep{Vitasse2013,Seiwa1991}. Interspecific differences in cues in contrast have been studied in relation to species' phylogenetic relatedness. Work on this topic has found strong evidence for events like flowering-time and budburst to be consistent within taxonomic families, suggesting conservatism in the genetic and physiological mechanisms that determine species phenologies \citep{Kochmer1986,Davies2013,Gougherty2018}. Studies of woody plant phenologies across species ranges have also highlighted the importance of local adaptations, with the presence of gradients in phenological responses and presumably cue use at northern range limits and relative similarities between congener species across continents \citep{Lechowicz1984,Chuine2001,Chuine2010}. %find additional citations
 In temperate systems for example, greater temperature variation in North America was associated with higher chilling requirements and more conservative phenological responses \citep{Zohner2017}.  Across species latitudinal ranges, stronger responses to photoperiod cues have been observed at lower latitudes \citep{Zohner2016}. Exploring these potential drivers of plant phenologies have illustrated the nuanced nature of phenology in shaping diverse communities, but they are still limited in the degree to which they explain the variation we observe across species and ecosystems.

Taking a functional trait approach to phenological research has also been proposed as a promising means to further explain the variation in cue use across species and geographically \citep{Flynn2018,Osada2017}. As a functional trait, phenology defines a species' temporal niche and can be used as proxies for tradeoffs in the timing of resource availability and growth, disturbance regimes, or abiotic risk factors. This is in line with several existing axes of variation for other well studied leaf and structural traits. Selection for phenology may therefore be strongly associated with selection for other key functional traits, however, few studies to date have directly tested for these relationships (but see \citep{Osada2017,Sun2006,Lechowicz1984}. Early work on functional traits used trait data from diverse global assemblages of deciduous plants to identify associations between traits and common growth strategies and different niche space \citep{Westoby1998,Wright2004,Chave2009}. The resulting leaf-height-seed scheme and more extensive the leaf economic spectrum found direct associations between several trait values and gradients in species growth rates and competitive abilities \citep{Westoby1998,Wright2004,Diaz2016,Chave2009,Funk2016}. Phenology has been less frequently incorporated into broader trait studies, however, recent experiments have found support for the existence of trade-offs in plant phenology and traits associated with established schemes in growth strategies and changes in climate \citep{Suzuki1997,Ishioka2013}. Species that budburst early in the growing season should have shorter rates of return on resource investments and the ability to take advantage of the greater abundance of soil nutrients and light early in the growing season (citation). This strategy should also favour species with traits that can sustain the costs of higher frost risk early in the growing season, for example producing leaves that are less costly leaves or vascular structures that allow for faster initiation in the spring \citep{Lechowicz1984,Lenz2016}. Previous studies have found early budbursting species to produce leaves that have high specific leaf areas, as well as high leaf nitrogen content which can be used as a proxy for plants' investment in photosynthetic potential and the production of photosynthetic enzymes like Rubisco \citep{Pereira2020}. %Earlier budbursting species experience less competition for light and can invest less in height and stem density \citep{Laughlin2010}. 

This suite of traits should contrast species with slower, more competitive growth strategies that benefit from slower rates of return on resource investment and the longer retention of leaf tissue. To date, research in this area has largely consisted of studies conducted within single sites or on few focal species, and are limited in their abilities to draw causal inferences. The likely associations between phenological events and growth strategies may allow for more generalizable trends across species and sites, and better account for species variability in key environmental cue use. 
 
While there have been numerous studies investigating the relationships between climate and functional traits, and a wealth of literature on the effects of climate cues as drivers of phenology, the interrelatedness of traits, phenological responses, and climate drivers has yet to be widely tested. The same selective pressure shaping species traits under variable temperature are likely to also act on species responses to phenological cues and define a species temporal niche. Here we combined global datasets from the OSPREE database of controlled environmental studies of budburst phenology with and functional trait data from the TRY and BIEN databases, and used this data to explicitly test for the relative differences in functional traits and the timing of budburst in response to variable forcing, chilling, and photoperiod cues. We hypothesize that species that budburst under low chilling, low forcing, and short photoperiod conditions are more likely to have traits associated with faster growth, but low competitiveness, as reflected by high SLA, high LNC, shorter heights, and lower seed mass. In contrast, species that budburst later under high chilling or high forcing temperatures, with long photoperiods  may have traits more associated with higher competitive abilities, such as low SLA, low LNC, greater heights and heavier seeds. 

\section{Methods}
For our analysis, we combined phenological data from the OSPREE database with functional trait data from the TRY and BIEN trait databases. We began by searching for trait data for all 96 species, which represent woody, deciduous species for which experimental data on phenological cues is available, and for which the phylogenetic relationship is well known. Trait data for ten functional trait was requested from the TRY databases for all 96 species (Table S1 - table of requested traits for each database). Additional trait data was acquired from the BIEN database using the BIEN R package (version X). From the BIEN database we obtained data for 34 species and seven species (Table S1). For our analysis we only included trait data from adult individuals, with a minimum height of 1.42 m. We removed all data denoted as being from experiments or growing in non-natural habitats. Finally we collated several similar traits, specifically grouping specific leaf area (SLA) values with, without petioles, and for which no petiole presence was specified, as simply SLA. Duplicated data across the datasets was removed (n= ) and subsetted the data to include only species for which we had a complete dataset for each species and trait. After our selection criteria, out data includes 26 species with at least one measurement for the following six traits: height, SLA, seed mass, LNC, SSD, \& LDMC.

To test for correlations in our six traits and further refine our trait selection, we performed a PCA. The principle component explained 32.2\% of variation while the second explained 23.4\% of the variation (Fig. S1). Given the strong association between the SLA and LDMC leaf traits, and similarly between stem specific density (SSD) and height, we further reduced the number of traits in our analysis to include only height, SLA, seed mass, and leaf nitrogen content. 

To test the relationships between functional traits and species cue responses, we developed a joint hierarchical bayesian model. Our model uses species-level trait values in our first model to predict species sensitivities to forcing, chilling, and photoperiod experimental cues. In addition to including partial pooling across species, the trait portion of the model includes a study level effect, thereby accounting for not only differences across species, but also the effects of methodological differences, and differences across habitats. The first model in our analysis calculates the latent variable that is then incorporated into the second phenology model. Values close to zero reflect small relationships between traits and cues values, while greater values represent high correlations between traits and phenological cues. This model was developed and validated using test data.
 
 % How much detail is needed - justify our approach or just include the model?
 %Do we include code for combining the effect of the grand mean with the species level effect?
\begin{align*}
\hat{trait_i} &= \mu_{grand_{sp}} + \alpha study_{study_i} \\
\mu_{grand_{sp}} &= \alpha_{grand} + \alpha sp_{sp_i} \\
\alpha_{grand}  & \sim N(0, \sigma_{grand})\\
\alpha sp & \sim N(0, \sigma_{sp}) \\
\alpha study& \sim N(0, \sigma_{study}) \\
trait_i & \sim N(\hat{trait}_i, \sigma_{trait}) \\
\vspace{1ex}\\
\hat{pheno}_i  &= \alpha pheno_{sp_i} + \beta force_{sp_i} * Forcing_{i} + \beta photo_{sp_i}  * Photo_{i} + \beta chill_{sp_i} * Chill_{i} \\
\beta force_{sp} &= \alpha force_{sp} + \beta trait.force * \alpha sp_{sp}\\
\beta chill_{sp} &= \alpha chill_{sp} + \beta trait.chill * \alpha sp_{sp}\\
\beta photo_{sp} &= \alpha photo_{sp} + \beta trait.photo * \alpha sp_{sp}\\
\alpha pheno & \sim N(\mu_{pheno}, \sigma_{pheno}) \\
\alpha force& \sim N(\mu_{force}, \sigma_{force}) \\
\alpha chill & \sim N(\mu_{chill}, \sigma_{chill}) \\
\alpha photo & \sim N(\mu_{photo}, \sigma_{photo}) \\
pheno_{i} & \sim N(\hat{pheno_i}, \sigma_{pheno}) \\
\end{align*}

Each trait varies in terms of the number of studies in which it is included as well as the number of individuals for which it is measured. As such, we model each trait individually using the same model specified above, but we the appropriate priors for each trait. Priors were tested using prior predictive checks. All analyses were done in Stan (version) using the rstan package (version) in R (version). 

To test for phylogenetic effects we obtained a rooted phylogenetic tree by pruning the tree from \citep{Smith2018}.

Run PGLS on the mean traits and mean posterior estimates 

\begin{align*}
y & \sim MVN(\mu, S)
\mu = \mu_{grand_{sp}} + \alpha study_{study_i} \\
\end{align*}

\section{Results}
PGLS suggests there are no strong phylogenetic effects

 We hypothesize that species that budburst under low chilling, low forcing, and short photoperiod conditions are more likely to have traits associated with faster growth, but low competitiveness, as reflected by high SLA, high LNC, shorter heights, and lower seed mass. In contrast, species that budburst later under high chilling or high forcing temperatures, with long photoperiods  may have traits more associated with higher competitive abilities, such as low SLA, low LNC, greater heights and heavier seeds. 
 
\section{Discussion}

Only consider above ground traits

\pagebreak
\bibliographystyle{refs/bibstyles/amnat.bst}% 
\bibliography{refs/traitors.bib}



\end{document}
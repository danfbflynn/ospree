\documentclass{article}\usepackage[]{graphicx}\usepackage[]{color}

\usepackage{alltt}
\usepackage{float}
\usepackage{graphicx}
\usepackage{tabularx}
\usepackage{siunitx}
\usepackage{amssymb} % for math symbols
\usepackage{amsmath} % for aligning equations
\usepackage{textcomp}
\usepackage{booktabs}
\usepackage{mdframed}
\usepackage{natbib}
\usepackage[colorinlistoftodos]{todonotes} % to make comments on the margin
\usepackage[small]{caption}
\setlength{\captionmargin}{30pt}
\setlength{\abovecaptionskip}{0pt}
\setlength{\belowcaptionskip}{10pt}
\topmargin -1.5cm        
\oddsidemargin -0.04cm   
\evensidemargin -0.04cm
\textwidth 16.59cm
\textheight 21.94cm 
%\pagestyle{empty} %comment if want page numbers
\parskip 7.2pt
\renewcommand{\baselinestretch}{1.5}
\parindent 0pt
%\usepackage{lineno}
%\linenumbers

%% R Script

\title{Woody plant phenological responses are strongly associated with key functional traits- Outline}

\begin{document}

\maketitle

\noindent Authors:\\
The Wolkovich Lab in 2021 $^{1,2,3,4}$
\vspace{2ex}\\
\emph{Author affiliations:}\\
$^{1}$Forest \& Conservation Sciences, Faculty of Forestry, University of British Columbia, 2424 Main Mall, Vancouver, BC V6T 1Z4;\\
$^{2}$Arnold Arboretum of Harvard University, 1300 Centre Street, Boston, Massachusetts, USA;\\
$^{3}$Organismic \& Evolutionary Biology, Harvard University, 26 Oxford Street, Cambridge, Massachusetts, USA;\\
$^{4}$Edificio Ciencias, Campus Universitario 28805 Alcalá de Henares, Madrid, Spain\\
 

\vspace{2ex}
$^*$Corresponding author: deirdre.loughnan@alumni.ubc.ca\\
\renewcommand{\thetable}{\arabic{table}}
\renewcommand{\thefigure}{\arabic{figure}}
\renewcommand{\labelitemi}{$-$}
\setkeys{Gin}{width=0.8\textwidth}

Climate change is altering the timing of species phenologies, with such changes in temporal niches reshaping ecological communities and interactions between species. In temperate systems, the observed advances in plant phenological events, such as budburst, leafout, and flowering times, are associated with changes in seasonal temperatures, particularly warming winter and spring conditions \citep{Menzel2006,Fitter2002}. But despite this strong general trend, phenological responses vary across species and geographically, and we have yet to fully understand the underlying mechanisms driving observed differences \citep{Chuine2010,Morin2009}. As the effects of climate change become more pronounced, understanding these relationships is of increasing importance if we are to predict and preserve the diversity and services found in temperate forest ecosystems. %The relationships between environmental cues and phenological events define for example, the duration of the growing season, the trajectory of community assembly, species ranges, and ecosystem services \citep{Cleland2007,Lopez2008,Chuine2010}, ultimately shaping and mitigating the impact of climate change on forest communities.

Over several decades of work has identified three cues -- chilling, forcing, and photoperiod -- to be the primary proximate drivers of budburst and leafout in temperate deciduous species \citep{Chuine2016}. For budburst to occur, species must experience extended period of cold temperatures to break dormancy \citep{Cooke2012}, with species with higher chill requirements budbursting later in the season. Spring forcing temperatures, or the temperatures needed to cue species to initiate growth after dormancy release, are also changing as temperatures warm and the timing at which suitable temperature thresholds are met occur earlier within the season (citation). Photoperiod cues can also determine some species ability to initiate growth \citep{Basler2014,Zohner2020}, however, species with strong photoperiod requirements are expected to be more constrained in their ability to track changes in temperature and may face fitness costs and novel species interactions as a result \citep{Guy2014}. Previous studies support the general trend of advancing budburst in response to each cue \citep{Flynn2018}, but with considerable variation in the relative importance of different cues across species \citep{Chuine2016,Flynn2018}. Some woody plant species, for example, require less forcing to budburst after experiencing a cool winter with more chilling, while also having the ability to compensate for low chilling with high forcing conditions or longer photoperiods \citep{Laube2014,Harrington2015,Flynn2018,Caffarra2011,Basler2014,Zohner2016}. Evidence for the role of photoperiod is largely species specific  \citep{Heide1993, Basler2014, Singh2017, Zohner2016}, with few studies testing for its importance across species in a community (but see  \cite{Flynn2018}). Species that are less dependent on photoperiod cues and able to track trends in temperatures may benefit from greater intra-annual phenotypic plasticity resulting in greater fitness outcomes under increasingly variable climates (citation). Despite the insights that identifying these proximate drivers have provided, we still lack the generalizable and mechanistic understanding of why species and populations differ in their cue use that is needed to predict future changes in species sensitivities and community structures.

In our efforts to understand variation spring phenologies, several potential mechanisms have been tested to identify the drivers of species cue use. Work exploring drivers of intraspecific cue use, for example, has found age or the development stage of woody plants to be an important factor. Younger life stages, including both seedlings and younger understory trees both budburst earlier than mature individuals in the canopy \citep{Vitasse2013,Seiwa1991}. These trends reflect both differences in the temperature sensitivities across life stages and effects of ontogenic changes as trees mature \citep{Vitasse2013,Seiwa1991}. Interspecific differences in cues in contrast have been studied in relation to species' phylogenetic relatedness. Work on this topic has found strong evidence for events like flowering-time and budburst to be consistent within taxonomic families, suggesting conservatism in the genetic and physiological mechanisms that determine species phenologies \citep{Kochmer1986,Davies2013,Gougherty2018}. Studies of woody plant phenologies across species ranges have also highlighted the importance of local adaptations, with the presence of gradients in phenological responses and presumably cue use at northern range limits \citep{Lechowicz1984,Chuine2001,Chuine2010}. In temperate systems for example, greater temperature variation in North America was associated with higher chilling requirements and more conservative phenological responses \citep{Zohner2017}.  Across species latitudinal ranges, stronger responses to photoperiod cues have been observed at lower latitudes \citep{Zohner2016}. Exploring these potential drivers of plant phenologies have illustrated the nuanced nature of phenology in shaping diverse communities, but they are still limited in the degree to which they explain the variation we observe across species and ecosystems.

Taking a functional trait approach to phenological research has also been proposed as a means to explain the variation in cue use across species and geographically \citep{Flynn2018,Osada2017}. Early work on functional traits used trait data from diverse global assemblages of deciduous plants to identify associations between traits and common growth strategies and different niche space \citep{Westoby1998,Wright2004,Chave2009}. The resulting leaf-height-seed scheme and more extensive the leaf economic spectrum found direct associations between several trait values and gradients in species growth rates and competitive abilities \citep{Westoby1998,Wright2004,Diaz2016,Chave2009,Funk2016}. While phenological traits have been identified as ecologically important for many years \citep{Weiher1999, Laughlin2014}, only a handful of studies have explored its role in the larger framework. The existence of trade-offs in plant phenology and traits associated with growth strategies and changes in climate have been observed across plant functional groups and habitat types. Deciduous woody species with smaller vessel diameters and diffuse or semi-ring-porous xylem structures have been shown to leaf out earlier, as this anatomy reduces the risk of embolism during freezing events  \citep{Panchen2014, Lechowicz1984}. In testing relationships between budburst and leaf traits of deciduous tree species in Japan, \citep{Osada2017} found positive correlations between budburst date and leaf area, leaf mass, and nitrogen content by both mass and area. \citep{Sun2006} also found deciduous species with high leaf mass area (a trait that is the inverse of specific leaf area) to budburst earlier in deciduous oak forests in eastern China. %Note their prediction is opposite ours
While variation in leafout can also relate to species heights, both intraspecifically and across functional groups such as understory species verses canopy species, with shorter individuals leafing out earlier than taller ones \citep{Seiwa1998, 1999b}. Studies of flowering phenology in herbaceous species also find phenology to relate to commonly measured traits, with early germinating species being taller with greater relative growth rates than later germinating species \citep{Sun2011}. To date, research in this area has largely consisted of studies conducted within single sites, with few focal species or traits, limiting their ability to draw more general and causal inferences. 

While there have been numerous studies investigating the relationships between climate and functional traits, and a wealth of literature on the effects of climate cues as drivers of phenology, the interrelatedness of traits, phenological responses, and climate drivers has yet to be widely tested. The same selective pressure shaping species traits under variable temperature are likely to also act on species responses to phenological cues and define a species temporal niche. Drawing on previous work and the broader trait literature, we predict that species that budburst early in the growing season should have shorter rates of return on resource investments and the ability to take advantage of the greater abundance of soil nutrients and light early in the growing season. Plants that produce leaves with high leaf nitrogen content and SLA can take advantage of greater light availability with greater rates of photosynthesis \citep{Wright2004,Pereira2020}, while also limiting the costs of tissue \citep{Lambers2004, Westoby2006, Herault2011}. Earlier budbursting species also experience less competition for light and can invest less in height and stem density \citep{Laughlin2010}. This suite of traits should contrast species with slower, more competitive growth strategies that benefit from slower rates of return on resource investment and the longer retention of leaf tissue. 

We test for associations between plant phenological responses to environmental cues and common functional traits. Budburst data for tree species in controlled environmental studies was selected from the Observed Spring Phenology Response in Experimental Environments (OSPREE) database and paired with functional trait data from the TRY and BIEN databases. This data was used to explicitly test for the relative differences in functional traits and the timing of budburst in response to variable forcing, chilling, and photoperiod cues. We predict that species that budburst under low chilling, low forcing, and short photoperiod conditions are more likely to have traits associated with faster growth, but low competitiveness, as reflected by high SLA, high LNC, shorter heights, and lower seed mass. In contrast, species that budburst later under high chilling or high forcing temperatures, with long photoperiods  may have traits more associated with higher competitive abilities, such as low SLA, low LNC, greater heights and heavier seeds. 

\section{Methods}
To test for relationships between functional trait and species phenological cues, we combined trait data from the TRY and BIEN trait databases with data on phenological responses to temperature and photoperiod cues from experiments. We began by searching for trait data for all 234 species for which we have budburst data in the OSPREE database. We requested trait data for ten functional traits from the TRY database (Table S1). Data was also obtained from the BIEN database using the BIEN R package \citep{Maitner2017}, and included data from 34 species and seven traits (Table S1). In our analysis we only included trait data from adult individuals of deciduous woody species, with a minimum height of 1.42 m. We removed all data denoted as being from experiments or growing in non-natural habitats. Traits were also grouped where appropriate, for example, separate entries for specific leaf area (SLA) values with petioles, without petioles, and for which no petiole presence was specified were all categorized as a single trait in our analysis (see Table S1). Data from studies duplicated in both the TRY and BIEN datasets were also removed (n= 434905).  Finally, we subsetted the data to include only a complete dataset for each species and trait, resulting in a final dataset which includes 26 species with at least one measurement for the following six traits: height, SLA, seed mass, LNC, SSD, \& LDMC (n = 60740, n = 8524, n = 404, n = 243, n = 5474 for each trait respectively). % no of studies: ht = 24, sla = 19, sm = 5, lnc = 13, sdd = 7, ldmc = 9 

Data of spring phenological responses to forcing, chilling, and photoperiod cues in controlled environments was obtained from the OSPREE database. This database was constructed using a literature search through ISI Web of Science and Google Scholar using the following terms:(budburst OR leaf-out) AND (photoperiod OR daylength) AND temperature* - and - (budburst OR leaf-out) AND dorman*. First published in 2019, this database has since been updated, and is now includes the review of an additional 623 and 270 new publications from each search term respectively. From this subsequent review, we added an additional 12 papers. For additional information on the construction of the OSPREE database, see \citep{OSPREE}. In our analysis, we used the available budburst data for our 26 focal species, from 28 unique studies. %should we include details on how the cues are estimated or just cite the budburst MS?

To test for correlations in our six traits and further refine our trait selection, we performed a PCA. The principle component explained 32.2\% of variation while the second explained 23.4\% of the variation (Fig. S1). Given the strong association between the SLA and LDMC leaf traits, and similarly between stem specific density (SSD) and height, we further reduced the number of traits in our analysis to include only height, SLA, seed mass, and LNC. 

We fit four models, with each trait modelled individually to species budburst data.  Using the following joint hierarchical bayesian models, we estimated the relative effect of a given trait to be included in our model of cue responses to budburst date: 

\begin{align*}
\hat{trait_i} &= \mu_{grand_{sp}} + \alpha study_{study_i} \\
\mu_{grand_{sp}} &= \alpha_{grand} + \alpha sp_{sp_i} \\
\alpha_{grand}  & \sim N(0, \sigma_{grand})\\
\alpha sp & \sim N(0, \sigma_{sp}) \\
\alpha study& \sim N(0, \sigma_{study}) \\
trait_i & \sim N(\hat{trait}_i, \sigma_{trait}) \\
\vspace{1ex}\\
\hat{pheno}_i  &= \alpha pheno_{sp_i} + \beta force_{sp_i} * Forcing_{i} + \beta photo_{sp_i}  * Photo_{i} + \beta chill_{sp_i} * Chill_{i} \\
\beta force_{sp} &= \alpha force_{sp} + \beta trait.force * \alpha sp_{sp}\\
\beta chill_{sp} &= \alpha chill_{sp} + \beta trait.chill * \alpha sp_{sp}\\
\beta photo_{sp} &= \alpha photo_{sp} + \beta trait.photo * \alpha sp_{sp}\\
\alpha pheno & \sim N(\mu_{pheno}, \sigma_{pheno}) \\
\alpha force& \sim N(\mu_{force}, \sigma_{force}) \\
\alpha chill & \sim N(\mu_{chill}, \sigma_{chill}) \\
\alpha photo & \sim N(\mu_{photo}, \sigma_{photo}) \\
pheno_{i} & \sim N(\hat{pheno_i}, \sigma_{pheno}) \\
\end{align*}

Where \emph{i} represents a unique trait or observation in the first and second section of the model respectively, \emph{sp} represents the species, \emph{study} represents studies, alpha parameters are the intercepts for species level effects for trait differences and cue responses, beta parameters represent the slope estimates for cues responses, trait is the estimated trait value, and pheno the estimated day of budburst since cues were applied. % \alpha and \beta not working...to be fixed

To explicitly compare the effects of chilling, forcing, and photoperiod, we used standardized z-scored values for the predictor variables which accounts for the differences in the scale of predictor variables across studies \citep{Gelman2006}, as well as the natural units for the cues (including chill units, degree C, and hours for chilling, forcing, and photoperiod respectively). Our model estimates a latent parameters for the species-level effect of a trait to then estimate species sensitivities to forcing, chilling, and photoperiod cues. Values close to zero reflect small relationships between traits and cues values, while greater values represent high correlations between traits and phenological cues. In addition to including partial pooling across species, the trait portion of the model includes a study level effect. By including the effect of study, we can account for not only differences across species, but also the effects of methodological differences, and differences across habitats. We fit our models using the Stan programming language (Stan citation), interfaced with using the rstan package (version, citation). Our model was first developed using test data prior to being used used on the real data. In our models, we used weakly informative priors, with four simultaneous chains of 2,000 sampling iterations and 4,000 posterior samples for each parameter. The models produced Rhat values close to 1 and neffs greater than 10\% of the number of sampling iterations %triple check this is true
, indicating that the model performed sufficiently well.  

Given the abundance of height data and overrepresentation of 13 of our focal species, we randomly sampled 5000 height measurements for each of these species to include in our analysis. This reduces the effect of trait values from frequently observed species from overwhelming the partial pooling effect in our model. In addition we excluded seed mass data from the HE Marx dataset, it only included one value, making it challenging to include the study level effect.
 All analyses were done in Stan (citation?) using the rstan package (version) in R (version). 

Finally, we used a phylogenetic greneralized least-squares regression model (PGLS) to test the relationship between day of budburst and individual traits, allowing us to test for phylogenetic non-independence in the relationship \citep{Freckleton2002}. We obtained a rooted phylogenetic tree by pruning the tree developed by \citep{Smith2018}. The PGLS analysis was done using the "Caper" package in R \citep{Orne2013}.

Run PGLS on the mean traits and mean posterior estimates 

\section{Results}

Our models
PGLS suggests there are no strong phylogenetic effects

 \section{Discussion}


Limitations: Only consider above ground traits



\pagebreak
\bibliographystyle{refs/bibstyles/amnat.bst}% 
\bibliography{refs/traitors.bib}

\section{scrape}
In testing for likely associations between phenological events and other commonly measured traits, this work would improve our understanding of ecosystem functioning and a trade-offs in resource use and conservation across species.  

\end{document}
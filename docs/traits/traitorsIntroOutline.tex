\documentclass{article}

% required 
\usepackage[hyphens]{url} % this wraps my URL versus letting it spill across the page, a bad habit LaTeX has

\usepackage{Sweave}
\usepackage{graphicx}
\usepackage{natbib}
\usepackage{amsmath}
\usepackage{textcomp}%amoung other things, it allows degrees C to be added
\usepackage{float}
\usepackage[utf8]{inputenc} % allow funny letters in citaions 
\usepackage[nottoc]{tocbibind} %should add Refences to the table of contents?
\usepackage{amsmath} % making nice equations 
\usepackage{listings} % add in stan code
\usepackage{xcolor}
\usepackage{capt-of}%alows me to set a caption for code in appendix 
\usepackage[export]{adjustbox} % adding a box around a map
\usepackage{lineno}
\linenumbers
% recommended! Uncomment the below line and change the path for your computer!
% \SweaveOpts{prefix.string=/Users/Lizzie/Documents/git/teaching/demoSweave/Fig.s/demoFig, eps=FALSE} 
%put your Fig.s in one place! Also, note that here 'Fig.s' is the folder and 'demoFig' is what each 
% Fig. produced will be titled plus its number or label (e.g., demoFig-nqpbetter.pdf')
% make your captioning look better
\usepackage[small]{caption}
\usepackage{xr-hyper} %refer to Fig.s in another document
\usepackage{hyperref}
\setlength{\captionmargin}{30pt}
\setlength{\abovecaptionskip}{0pt}
\setlength{\belowcaptionskip}{10pt}

% optional: muck with spacing
\topmargin -1.5cm        
\oddsidemargin 0.5cm   
\evensidemargin 0.5cm  % same as oddsidemargin but for left-hand pages
\textwidth 15.59cm
\textheight 21.94cm 
% \renewcommand{\baselinestretch}{1.5} % 1.5 lines between lines
\parindent 0pt		  % sets leading space for paragraphs
% optional: cute, fancy headers
\usepackage{fancyhdr}
\pagestyle{fancy}
\fancyhead[LO]{Draft 2022}
\fancyhead[RO]{Temporal Ecology Lab}
% more optionals! %

\graphicspath{ {./Figures/} }% tell latex where to find photos 
\externaldocument[supp-]{Synchrony_Manuscript_supp}

%%% end preambling. %%%

\begin{document}


\title{Traitors intro}
\date{\today}

\maketitle 


\begin{enumerate}
\item Spring phenology is ecologically important shaping both ecosystem services and community dynamics % short! 3 sentences is likely enough...
\begin{enumerate}
\item timing of start of the spring defines the start and overall length growing season = potential for forest carbon storage
% Angert lab meeting: both Any and Fredi commented on earlier growth meaning early senescence, maybe this needs to be phrased better 
\item also shapes competitive interactions and the abundance and identity of herbivores and pollinators 
\item Climate change is shifting these things --- changing cues and causing many species to start growing earlier
\end{enumerate}

\item Amidst these trends, spring phenology is highly variable across years and species % or keep more general and don't get into forests yet?
% Angert lab meeting: we don't expand on yearly variation
\begin{enumerate}
\item At the start of spring we see the budburst and leaf out of understory shrubs %and spring ephemerals 
%Angert lab meeting: spring ephemerals are not woody, might confuse the reader
\item Followed later in the season by budburst of taller trees and the closure of the forest canopy.
\item Have identified differences across species (cite Flynn, and Laube)
\item But our understanding of why budburst is so diverse is still limited---does not relate to species larger niche space.
%Angert lab meeting: expand on what we mean by diverse, also what we meant by the second part of the above point ("larger niche space") was not clear to people
\end{enumerate} 

\item Timing of plant phenological events define species' temporal niche = the partitioning of resources across species over time (Gotelli \& Granves 1996 - ch5). % Try to keep as general idea
%Angert lab meeting: Amy felt this was tangential and since the intro is long, suggested moving it to the discussion
\begin{enumerate} 
\item temporal niche differences determine the abiotic environment during growth and biotic interactions -- for example, competitive landscape and pressures from herbivory/disease. % keep this short
%Angert lab meeting: several people felt bringing up herbivore damage and disease was tangential and confusing for a paper focused on plant phenology
\item Distribution of temporal niche within community influences its potential invasibility--- invasive spp tend to be early bb with the can fill vacant niche space early in the season. 
%20FebDL: I moved this to the end of the paragraph
% \item This diversity of selective pressures (biotic and abiotic) is likely to correlate with diversity in other plant traits as well % skip traits later -- but love this phrasing!
% \item A more holistic framework of the drivers of species temporal niches differences = important --- climate change is changing environmental conditions and habitats = shifting phenologies % may not need this

\end{enumerate}

\item In budbursting over the span of several weeks, plants experience different selective pressures.  % Great! Think of this as temporal niches in forests are defined by these pressures ... 
%Angert lab meeting: Both Takuji and Amy suggested having a better link for 3b to paragraph 4 - temporal niche should be shaped by selective pressures....
\begin{enumerate}
\item Species that bb early in the spring = greater abiotic pressures, such as risk of false spring events and frost = potential loss of tissue, but benefit from more light and resource availability
\item early would thus mean cheap tissues a plant could replace
\item Late spp have greater selection from biotic pressures = less light available and competition for resources 
\item late would be costly tissues better at resource acquisition
\item these traits (or trade-offs) relate to a full framework about this ... 
%\item Differences in bb phenology within communities = important in shaping community dynamics including competition \& herbivory. %DLJan14: this seems repetitive 
\end{enumerate}

%20FebDL: In reading this again, could we not just integrate this paragraph to the above? Or do you think it would be too long?
%\item These selective pressures could shape lots of plant attributes .... 
%\begin{enumerate}
%\item early would thus mean cheap tissues a plant could replace
%\item late would be costly tissues better at resource acquisition
%\item these traits (or trade-offs) relate to a full framework about this ... 
%\end{enumerate}

\item trait ecology’s goal = predict sp-level characteristics by traits alone %— how well we can do this to highly variable and species specific traits like phenology is unclear
%Angert lab meeting:  Amy suggested the phrasing "trait ecology's goal is to use traits as species-level proxies for characteristics that predict [lots of things! higher-level patterns, outcomes of interactions, responses to enviro change, etc] OR could simplify to "use traits to predict [stuff]" but i do think it's important to acknowledge that the traits are often pretty distal to the processes they are supposed to capture

could simplify to "use traits to predict [stuff]" but i do think it's important to acknowledge that the traits are often pretty distal to the processes they are supposed to capture
\begin{enumerate}
\item Many leaf and wood traits do follow predictable gradients in their trait values, having associations that range from acquisitive (fast) growth strategies to more conservative (slow) growth strategies.
%Angert lab meeting:  predictable in what ways? Across spp? Across environments?
\item Collectively, these trait relationships led to the development of the leaf economic spectrum and the wood economic spectrum
\item These frameworks have been built into decades of research linking functional traits to species responses to abiotic and biotic factors and community assembly
\item But these frameworks have limitations ... 
\item one is that they don't predict how variable are traits .... this is sometimes used to explain why phenology not incl. in traits % Note: try holding back point that phenology is NOT variable, once you include cues
% themselves can be highly variable, both across and within spp, highlighting the need for trait research that spans diverse species across geographic scales---Violle paper ‘viva la variability. 
\end{enumerate}

\item Despite the lack of integration between functional trait and phenological research, both are likely to shape species growth strategies and the biotic community they interact with.
%\item Considerable research on how functional traits relate to species growth strategies and competitive abilities — few studies include phenological traits. % reword as phenology and trait framework in forests go together critically! Make this a little more hypothesis-y 
\begin{enumerate}
\item timing of species growth likely to be related to species leaf and wood traits 
\item Shrubs and other woody understory species tend to budburst earlier--- soil resources and light availability is greatest--- shorter, with leaf traits favourable to higher light availability (i.e. photosynthetic potential) and tolerance of late spring frost.
\item Canopy species budburst later--- taller, requiring greater wood densities, but more constant light compensates for their slower overall growth rates.
\item But later budbursting species must be better competitors --- compete with more species
\item Associations are intuitive but few studies have tested for similar gradients in growth strategies in phenological events across diverse species
\end{enumerate}

%Angert lab meeting: Courtney brought up that evergreen shrubs have hardy leaves that might be different, especially subalpine species.

\item Based on these likely relationships between phenological and plant traits, we would expect the timing of budburst phenology to follow similar gradients as leaf and wood traits.
%\item We tested for possible relationships between budburst phenological cues and other commonly measured functional traits. % rephrase as: based on the (last paragraph) we would predict specific trait x phen relatioships ... .
\begin{enumerate}
\item spp with traits associated with acquisitive growth (high SLA, high LNC, short heights, small seeds) will have cue requirements associated with early tree budburst. %low forcing, chilling, and photoperiod % switch to arly
\item Spp that are better competitors with conservative growth and later budburst, with low SLA and LNC, tall heights, and large seeds, will have phenological response associated with later season budburst. % high forcing, chilling, and photoperiod requirements % early-late
\item This gradient of early-late = driven by cues: early species are low force/chill/photo, later species are high chill or photo or both and likely higher forcing % may be short para on its own
\end{enumerate}

% Save some of below for discussion? And cut otherwise this paragraph
%\item Previous studies of budburst in woody trees have shown 3 cues are most important for spp responses:
%\begin{enumerate}
%\item Chilling - the period of cold temperatures from late fall to late winter, % releases buds from dormancy % skip dormancy in this paper!
%\item Forcing - the occurrence of warm temperatures in spring % that initiate bud development 
%\item Photoperiod - daylength
%\item While field observations of phenology are highly variable — under controlled environments and set cues, bb is highly predictable
%\item This suggests there is potential to use phenological data from controlled environment studies to identifying the relationship between species cue responses and traits
%\end{enumerate}

\item We used available trait data from trait databases and bb data from the OSPREE database of controlled environment studies of woody plant species to test for associations between budburst responses to environmental cues and common functional traits
%Angert lab meeting:  need to better define chilling, forcing, photoperiod 
\begin{enumerate}
\item Focus on the effects of forcing, chilling, and photoperiod cues and four commonly measured traits — SLA, LNC, height, \& seed mass
\item Our model attributes phenological variation (day of bb) to species’ trait values while including residual variation from species (partial-pooling).
\item When traits explain a significant portion of the variation, spp will explain only a small amount — may be able to predict spp growth strategies and phenological responses from trait values.
\end{enumerate}

\end{enumerate}

Angert lab meeting comments about methods and results
\begin{enumerate}
\item Generally thought intro was too long, suggested combining paragraphs 1 \& 2, 6 \& 7
\item Fredi pointed out we never mention freezing resistance 
\item Natalie suggested an appendix for how we cleaned the data and a table of how many species/studies used to estimate the parameters in the model
\item Was not clear that the trait data and the OSPREE data were different studies, maybe stress this more
\item Colours should be changed - difficult to see the pale bands in fig 2 and the purple crosses in Fig 3
\item Indifference to having the conceptual fig above the fig vs in the intro
\item possibly add a figure of the response to cue on the y and species on the y - show the variation in response to cue by spp 
\item possibly add a figure of the trait data to show that it does correlate with phenology the way we would predict
\item People questioned why we used species that are not the most extreme as the examples in our figures (ie the green is sometimes in the cloud of points). 
\end{enumerate}

%Need to fit in into intro, not sure where:
%\begin{enumerate}
%\item Cues address phenological variability
%\item Be sure to clearly set up acquisitive vs. conservative
%\end{enumerate}

%Stuff we had, but could cut:
%\begin{enumerate}
%\item details of phenological responses - ectodormancy transition to endodormancy -- Cutting this, too much other content
%\item detailed definition of forcing, chilling, photoperiod
%\end{enumerate}

\end{document}


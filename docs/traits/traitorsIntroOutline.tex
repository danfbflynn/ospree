\documentclass{article}

% required 
\usepackage[hyphens]{url} % this wraps my URL versus letting it spill across the page, a bad habit LaTeX has

\usepackage{Sweave}
\usepackage{graphicx}
\usepackage{natbib}
\usepackage{amsmath}
\usepackage{textcomp}%amoung other things, it allows degrees C to be added
\usepackage{float}
\usepackage[utf8]{inputenc} % allow funny letters in citaions 
\usepackage[nottoc]{tocbibind} %should add Refences to the table of contents?
\usepackage{amsmath} % making nice equations 
\usepackage{listings} % add in stan code
\usepackage{xcolor}
\usepackage{capt-of}%alows me to set a caption for code in appendix 
\usepackage[export]{adjustbox} % adding a box around a map
\usepackage{lineno}
\linenumbers
% recommended! Uncomment the below line and change the path for your computer!
% \SweaveOpts{prefix.string=/Users/Lizzie/Documents/git/teaching/demoSweave/Fig.s/demoFig, eps=FALSE} 
%put your Fig.s in one place! Also, note that here 'Fig.s' is the folder and 'demoFig' is what each 
% Fig. produced will be titled plus its number or label (e.g., demoFig-nqpbetter.pdf')
% make your captioning look better
\usepackage[small]{caption}
\usepackage{xr-hyper} %refer to Fig.s in another document
\usepackage{hyperref}
\setlength{\captionmargin}{30pt}
\setlength{\abovecaptionskip}{0pt}
\setlength{\belowcaptionskip}{10pt}

% optional: muck with spacing
\topmargin -1.5cm        
\oddsidemargin 0.5cm   
\evensidemargin 0.5cm  % same as oddsidemargin but for left-hand pages
\textwidth 15.59cm
\textheight 21.94cm 
% \renewcommand{\baselinestretch}{1.5} % 1.5 lines between lines
\parindent 0pt		  % sets leading space for paragraphs
% optional: cute, fancy headers
\usepackage{fancyhdr}
\pagestyle{fancy}
\fancyhead[LO]{Draft 2022}
\fancyhead[RO]{Temporal Ecology Lab}
% more optionals! %

\graphicspath{ {./Figures/} }% tell latex where to find photos 
\externaldocument[supp-]{Synchrony_Manuscript_supp}

%%% end preambling. %%%

\begin{document}


\title{Traitors intro}
\date{\today}

\maketitle 

BROAD outline ...

\begin{enumerate}

\item Open with phenological diversity (lots of it across spp. most of it unexplained, cite Laube, Flynn etc. work)

\begin{enumerate}
\item Timing plant phenological events (budburst) define species' temporal niche = the partitioning of resources across species over time (Gotelli \& Granves 1996 - ch5).
\item Differences along this niche axis contribute to community assembly, defining the abiotic environment experienced during growth and biotic interactions -- competitive landscape and pressures from herbivory.
\item Studies of woody plants show diversity in species temporal niche e.g spring budburst and variability to environmental cues (Laube2014, Flynn2018)
\item Changes in climate are altering spp temporal niche, however, with earlier spring growth having cascading effects to ecological communities and spp interactions
\item But still have a limited mechanistic understanding of how specific spp will respond
\end{enumerate}

\item Set up early vs. late phenology and frost versus competition (could maybe mention invaders within this)? Try to basically set up the idea of traits, without saying traits.
% group a, c-f, frost = abiotic then competition + herbivory = bitoic
\begin{enumerate}
\item The timing of bb in woody spp appear to range from early spring species — bb prior to canopy closure — and later canopy spp. 
\item Early in the spring = greater abitoic pressures, such as risk of false spring events and frost = affects early budbursting spp = potential loss of tissue
\item Late spp have greater selection from biotic pressures = less light available and competition for resources 
\item Differences in bb phenology within communities = important in shaping community dynamics including competition \& herbivory. 
\item But also the potential invasibility of a community, invasive spp tend to be early bb with the potential to fill vacant niche space early in the season. 
\end{enumerate}

\item Now get to phenology (as day of year/early-late) x traits and how that connects to 2.
\begin{enumerate}
\item Considerable work on the functional traits related to species growth strategies and competitive abilities — few studies include phenology
\item Leaf economic spectrum: spp fall along gradients of acquisitive (fast) growth to more conservative (slow) growth %connect more to second point
\item Decades of research linking functional traits to species responses to abiotic and biotic factors and community assembly
\item Strategies that favour fast growth should promote early bb, often associated with traits realted to species responses to abiotic factors (eg frost risk, light caputrue)
\item Spp that are better competitors with conservative growth bb later - associate this with traits that biotic factors, like competition
\item But whether other there are associations between other functional traits and the cues responses that define species temporal niche is unknown.
\end{enumerate}

\item Set up hypotheses...
% I was concerned that it would not make sense to introudce the hypotheses as they were written before with the cues before defining the cues in the next paragraph, but perhaps this makes the hypotheses too vague. We could move this until after the next paragraph
\begin{enumerate}
\item We predict that spp with traits associated with acquisitive growth (high SLA, high LNC, short heights, small seeds) will have will have cue requirements associated with earlier budburst %low forcing, chilling, and photoperiod cue requirements
\item Spp that are better competitors with conservative growth, with low SLA and LNC, tall heights, and large seeds, will have phenological response associated with later budburst %high forcing, chilling, and photoperiod requirements
\end{enumerate}

\item Get into complexity of cues after hypotheses
% have Paragraph on cues as creating early to late --> explain cues -- pt 6? Keep a -c as 5
\begin{enumerate}
% add definitions for the cues
\item Previous studies have shown 3 cues are most important for spp responses:
\begin{enumerate}
\item Chilling - the period of cold temperatures from late fall to late winter, releases buds from dormancy
\item Forcing - the occurrence of warm temperatures in spring that initiate bud developement 
\item Photoperiod - daylength
\end{enumerate}
\item field obs of phenology are highly variable — but under controlled environments and set cues, bb is highly predictable
\item traits themselves can be highly variable, both across and within spp - Violle paper ‘viva la variability’
\item trait ecology’s goal = predict sp-level characteristics by traits alone — how well we can do this to highly variable and species specific traits like phenology is unclear
\item Potential to use phenological data from controlled environment studies to identifying the relationship between sp cue responses and traits


\end{enumerate}

\item Here's what we do here.
\begin{enumerate}
\item we test for associations between plant phenological responses to environmental cues and common functional traits 
\item use available trait data from trait databases with bb data from controlled environment studies of woody plant species from the OSPREE database.
\item We focus on the effects of forcing, chilling, and photoperiod cues and four easy to measure traits — SLA, LNC, height, \& seed mass
\item Our model attributes phenological variation (day of bb) to species’ trait values while including residual variation from species (partial-pooling).
\item When traits explain a significant portion of the variation, spp will explain only a small amount — may be able to predict spp growth strategies and phenological responses from trait values.
\item Potential to use phenological data from controlled environment studies to identifying the relationship between sp cue responses and traits
\end{enumerate}

\item Our model ...
\begin{enumerate}
\item Our model attributes phenological variation (day of bb) to species’ trait values while including residual variation from species (partial-pooling).
\item When traits explain a significant portion of the variation, spp will explain only a small amount — may be able to predict spp growth strategies and phenological responses from trait values.
\end{enumerate}
\end{enumerate}

Need to fit in into intro, not sure where:
\begin{enumerate}
\item Cues address phenological variability
\item Be sure to clearly set up acquisitive vs. conservative
\end{enumerate}

Stuff we had, but could cut:
\begin{enumerate}
\item details of phenological responses - ectodormancy transition to endodormancy -- Cutting this, too much other content
\item detailed definition of forcing, chilling, photoperiod
\end{enumerate}

Fig 1:
i) Can you confirm the slopes (when trait effect = 0) are constant across the top conceptual panels? If not, we should make them that way I think to minimize what changes across them?
DLDec15: Yes they are, the betaChill slope is always -2, betaForceSp =5 and betaPhotoSp = 1 for each plot

ii) Make sure the figure in the Supp that is similar has the same y axis scale -
DLDec15: I will fix that in the next draft for sure

\end{document}


\documentclass{article}

% required 
\usepackage[hyphens]{url} % this wraps my URL versus letting it spill across the page, a bad habit LaTeX has

\usepackage{Sweave}
\usepackage{graphicx}
\usepackage{natbib}
\usepackage{amsmath}
\usepackage{textcomp}%amoung other things, it allows degrees C to be added
\usepackage{float}
\usepackage[utf8]{inputenc} % allow funny letters in citaions 
\usepackage[nottoc]{tocbibind} %should add Refences to the table of contents?
\usepackage{amsmath} % making nice equations 
\usepackage{listings} % add in stan code
\usepackage{xcolor}
\usepackage{capt-of}%alows me to set a caption for code in appendix 
\usepackage[export]{adjustbox} % adding a box around a map
\usepackage{lineno}
\linenumbers
% recommended! Uncomment the below line and change the path for your computer!
% \SweaveOpts{prefix.string=/Users/Lizzie/Documents/git/teaching/demoSweave/Fig.s/demoFig, eps=FALSE} 
%put your Fig.s in one place! Also, note that here 'Fig.s' is the folder and 'demoFig' is what each 
% Fig. produced will be titled plus its number or label (e.g., demoFig-nqpbetter.pdf')
% make your captioning look better
\usepackage[small]{caption}
\usepackage{xr-hyper} %refer to Fig.s in another document
\usepackage{hyperref}
\setlength{\captionmargin}{30pt}
\setlength{\abovecaptionskip}{0pt}
\setlength{\belowcaptionskip}{10pt}

% optional: muck with spacing
\topmargin -1.5cm        
\oddsidemargin 0.5cm   
\evensidemargin 0.5cm  % same as oddsidemargin but for left-hand pages
\textwidth 15.59cm
\textheight 21.94cm 
% \renewcommand{\baselinestretch}{1.5} % 1.5 lines between lines
\parindent 0pt		  % sets leading space for paragraphs
% optional: cute, fancy headers
\usepackage{fancyhdr}
\pagestyle{fancy}
\fancyhead[LO]{Draft 2023}
\fancyhead[RO]{Temporal Ecology Lab}
% more optionals! %

\graphicspath{ {./Figures/} }% tell latex where to find photos 
\externaldocument[supp-]{Synchrony_Manuscript_supp}

%%% end preambling. %%%

\begin{document}


\title{Traitors intro}
\date{\today}

\maketitle 


\begin{enumerate}
\item Spring phenology is ecologically important shaping both ecosystem services and community dynamics % short! 3 sentences is likely enough...
\begin{enumerate}
\item timing of start of the spring defines the start and overall length growing season = potential for forest carbon storage
% Angert lab meeting: both Any and Fredi commented on earlier growth meaning early senescence, maybe this needs to be phrased better 
%emwMar6 -- not sure i get the problem here ... their comments seem to get into the weeds of how earlier growth works (it's not well known that earlier start = earlier end, so I am not sure most readers would trip up here) so I would just ignore this. 
\item also shapes competitive interactions and the abundance and identity of herbivores and pollinators 
\item Climate change is shifting these things --- changing cues and causing many species to start growing earlier
\end{enumerate}

\item Amidst these trends, spring phenology is highly variable across years and species % or keep more general and don't get into forests yet?
% Angert lab meeting: we don't expand on yearly variation
\begin{enumerate}
\item For example, in one of the most well studied systems, forests, we see the budburst and leaf out of understory plants first, followed later in the season by budburst of taller trees
\item We have long known inter-annual variation is driven by shifts in temperature and recent advances underscore this (cite Flynn, and Laube)
\item These advances have also identified important differences across species, but have provided limited reasons for species differences
\end{enumerate} 

\item Within a community different species can start growth each year across several weeks, and will thus experience different selective pressures.  %emwMar6 -- may be useful to give example span of diff species in one community if we have it in a citation.  
\begin{enumerate}
\item Species that bb early in the spring = greater abiotic pressures, such as risk of false spring events and frost = potential loss of tissue, but benefit from more light and resource availability
\item Late spp have greater selection from biotic pressures = less light available and competition for resources 
%\item Differences in bb phenology within communities = important in shaping community dynamics including competition \& herbivory. %DLJan14: this seems repetitive 
\end{enumerate}

%20FebDL: In reading this again, could we not just integrate this paragraph to the above? Or do you think it would be too long?
%emwMar6 -- I started pulling out the traits from the above section and creating a new paragraph, before I read the below. So, no -- I don't think this works for 2 reasons: (1) too long and (2) way too much for the reader (not that your topic sentence above does not include anything about traits, whereas the flow of topic sentences if you re-include the below paragraph is much better). Remember a short and clear paragraph is okay, and better than a long dense one with a poor topic sentence. 
\item These selective pressures could shape lots of plant attributes .... 
\begin{enumerate}
\item early would thus mean cheap tissues a plant could replace
\item late would be costly tissues better at resource acquisition
\item these traits (or trade-offs) relate to a full framework about this ... 
\end{enumerate}

\item trait ecology’s goal = predict sp-level characteristics by traits alone %— how well we can do this to highly variable and species specific traits like phenology is unclear
%Angert lab meeting:  Amy suggested the phrasing "trait ecology's goal is to use traits as species-level proxies for characteristics that predict [lots of things! higher-level patterns, outcomes of interactions, responses to enviro change, etc] OR could simplify to "use traits to predict [stuff]" but i do think it's important to acknowledge that the traits are often pretty distal to the processes they are supposed to capture
%emwMar6 -- I would make the Angert lab change -- you can save your point about 'traits are often pretty distal to the processes they are supposed to capture'  for the discussion
could simplify to "use traits to predict [stuff]" but i do think it's important to acknowledge that the traits are often pretty distal to the processes they are supposed to capture
\begin{enumerate}
\item Many leaf and wood traits follow predictable gradients in their trait values, having associations that range from acquisitive (fast) growth strategies to more conservative (slow) growth strategies.
%Angert lab meeting:  predictable in what ways? Across spp? Across environments?
\item Collectively, these trait relationships led to the development of the leaf economic spectrum and the wood economic spectrum
\item These frameworks have been built into decades of research linking functional traits to species responses to abiotic and biotic factors and community assembly
\item But these frameworks have limitations ... 
\item one is that they don't predict how variable are traits .... this is sometimes used to explain why phenology not incl. in traits % Note: try holding back point that phenology is NOT variable, once you include cues
% themselves can be highly variable, both across and within spp, highlighting the need for trait research that spans diverse species across geographic scales---Violle paper ‘viva la variability. 
\item Integrating phenology within functional traits could both advance the functional trait framework and potentially help explain why species have different phenologies ... %emwMar6 -- 'the biotic community' they interact with is too much -- I think 'growth strategies' can cover competition, and you can get more into biotic stuff in the discussion.... And then I made more edits to merge 6 & 7, so moving this comment up. 
\end{enumerate}

\item As outlined above, early to late season differences in selection pressures --- where early season species have high access to resources but risk tissue loss and damage, whereas late season species experience a highly competitive but less environmentally risky environment -- would predict variation in leaf and wood traits with phenology. 
\begin{enumerate}
\item Shrubs and other woody understory species that tend to budburst earlier--- soil resources and light availability is greatest--- would have traits associated with acquisitive growth: shorter, with leaf traits favourable to higher light availability and tolerance of late spring frost (high SLA, high LNC)  %low forcing, chilling, and photoperiod 
\item Canopy species that budburst later must be better competitors, and so would have traits associated with conservative growth: taller---requiring greater wood densities, low SLA and LNC. % high forcing, chilling, and photoperiod requirements % early-late
\item This acquisitive to conservative continuum for early to late species may also predict seed size to increase with later budburst. 
\item This gradient of early-late = driven by cues: early species are low force/chill/photo, later species are high chill or photo or both and likely higher forcing % may be short para on its own
\end{enumerate}




% Save some of below for discussion? And cut otherwise this paragraph
%\item Previous studies of budburst in woody trees have shown 3 cues are most important for spp responses:
%\begin{enumerate}
%\item Chilling - the period of cold temperatures from late fall to late winter, % releases buds from dormancy % skip dormancy in this paper!
%\item Forcing - the occurrence of warm temperatures in spring % that initiate bud development 
%\item Photoperiod - daylength
%\item While field observations of phenology are highly variable — under controlled environments and set cues, bb is highly predictable
%\item This suggests there is potential to use phenological data from controlled environment studies to identifying the relationship between species cue responses and traits
%\end{enumerate}

\item We used available trait data from trait databases and bb data from the OSPREE database of controlled environment studies of woody plant species to test for associations between budburst responses to environmental cues and common functional traits
%Angert lab meeting:  need to better define chilling, forcing, photoperiod  %emwMar6  -- yes, we'll have to see how working out 'This gradient of early-late = driven by cues' goes ... we should have these well (and quickly defined) for the reader BEFORE we get here. 
\begin{enumerate}
\item Focus on the effects of forcing, chilling, and photoperiod cues and four commonly measured traits — SLA, LNC, height, \& seed mass
\item Our model attributes phenological variation (day of bb) to species’ trait values while including residual variation from species (partial-pooling).
\item When traits explain a significant portion of the variation, spp will explain only a small amount — may be able to predict spp growth strategies and phenological responses from trait values.
\end{enumerate}

\end{enumerate}

Angert lab meeting comments about methods and results
\begin{enumerate}
\item Generally thought intro was too long, suggested combining paragraphs 1 \& 2, 6 \& 7 %emwMar6 -- 1 & 2 should be short, so they don't have to go together. 6 & 7 are a place we should save some space. See edits above. 
\item Fredi pointed out we never mention freezing resistance %emwMar6 -- we mention false springs
\item Natalie suggested an appendix for how we cleaned the data and a table of how many species/studies used to estimate the parameters in the model %emwMar6 -- sure on the cleaning data and being clear about how much data is in each model (one model = all same data for all parameters in it ...)
\item Was not clear that the trait data and the OSPREE data were different studies, maybe stress this more %emwMar6 -- meh, seems fine to me. 
\item Colours should be changed - difficult to see the pale bands in fig 2 and the purple crosses in Fig 3
\item Indifference to having the conceptual fig above the fig vs in the intro 
\item possibly add a figure of the response to cue on the y and species on the y - show the variation in response to cue by spp %emwMar6 -- this is a repetitive plot but does suggest we need a version of Fig 3 with all species labeled in the supp, which I think I suggested before. Let's make that happen. 
\item possibly add a figure of the trait data to show that it does correlate with phenology the way we would predict %emwMar6 -- I would not add new figures now; we need to finish the paper. 
\item People questioned why we used species that are not the most extreme as the examples in our figures (ie the green is sometimes in the cloud of points). %emwMar6 -- Did you point out that they are extreme in some and that we use the same species everywhere so they did not end up extreme in all?
\end{enumerate}

%Need to fit in into intro, not sure where:
%\begin{enumerate}
%\item Cues address phenological variability
%\item Be sure to clearly set up acquisitive vs. conservative
%\end{enumerate}

%Stuff we had, but could cut:
%\begin{enumerate}
%\item details of phenological responses - ectodormancy transition to endodormancy -- Cutting this, too much other content
%\item detailed definition of forcing, chilling, photoperiod
%\end{enumerate}

{\bf Move to discussion or such ...}
\item Timing of plant phenological events define species' temporal niche = the partitioning of resources across species over time (Gotelli \& Granves 1996 - ch5). % Try to keep as general idea
\begin{enumerate} 
\item temporal niche differences determine the abiotic environment during growth and biotic interactions -- for example, competitive landscape and pressures from herbivory/disease. % keep this short
%Angert lab meeting: several people felt bringing up herbivore damage and disease was tangential and confusing for a paper focused on plant phenology. ... %emwMar6 -- I think this will work better in the discussion where you can flesh it out more. 
\item Distribution of temporal niche within community influences its potential invasibility--- invasive spp tend to be early bb with the can fill vacant niche space early in the season. 
%20FebDL: I moved this to the end of the paragraph
% \item This diversity of selective pressures (biotic and abiotic) is likely to correlate with diversity in other plant traits as well % skip traits later -- but love this phrasing!
% \item A more holistic framework of the drivers of species temporal niches differences = important --- climate change is changing environmental conditions and habitats = shifting phenologies % may not need this
\end{enumerate}

%emwMar6 -- mv below points may work in discussion 

Despite the lack of integration between functional trait and phenological research, both are likely to shape species growth strategies

more constant light compensates for their slower overall growth rates.  %emwMar6 -- too complex for intro

Associations are intuitive but few studies have tested for similar gradients in growth strategies in phenological events across diverse species


\end{document}


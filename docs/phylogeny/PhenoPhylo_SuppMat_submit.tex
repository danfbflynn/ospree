\documentclass[11pt]{article}
\usepackage[top=1.00in, bottom=1.0in, left=1.1in, right=1.1in]{geometry}
\renewcommand{\baselinestretch}{1.1}
\usepackage{graphicx}
\usepackage{natbib}
\usepackage{textcomp, gensymb} 
\usepackage{amsmath}
\usepackage[utf8]{inputenc}
\usepackage{lineno}
\usepackage{longtable}
\usepackage{caption}
\usepackage{xr-hyper} %refer to Fig.s in another document
\usepackage{hyperref}
\externaldocument{PhenoPhylo_ms_submit_wbib} 
\externaldocument{PhenoPhylo_ExtDat} 

%\urlstyle{same}

\def\labelitemi{--}
\renewcommand{\baselinestretch}{1.2}
\parindent=0pt
\parskip=5pt

\usepackage{Sweave}
\begin{document}
%\SweaveOpts{concordance=TRUE}


%\bibliographystyle{naturemag}% 

%\title{Supplementary Information\\
%Phylogenetic estimates of species-level phenology improve ecological forecasting}
%\maketitle


\renewcommand\contentsname{Table of contents}
\tableofcontents




\pagenumbering{arabic}
%$^*$Corresponding author: ignacio.moralesc@uah.es\\
\renewcommand{\thetable}{S\arabic{table}}
\renewcommand{\thefigure}{S\arabic{figure}}
\renewcommand{\labelitemi}{$-$}
\setkeys{Gin}{width=0.8\textwidth}









\clearpage


%%%%%%%%%%%%%%%%%%%%%%%%%%%%%%%
% Extended Methods
%%%%%%%%%%%%%%%%%%%%%%%%%%%%%%%

\section{Extended Methods}

\subsection{Details on data: sources and structure}

We provide several additional details here on how our dataset is structured, including the data sources and the geographical bias of the data. First, in Fig. \ref{fig:raw2ddata} we show the distribution of the response and predictor variables. Second, we show a map (Fig. \ref{fig:mapstudylocations}) of the locations where woody plant samples were collected. As our data are geographically limited to mainly temperate Europe and North America, further experimental research on the phenology of extra-temperate woody species will be important for determining the extent to which the results in this work hold in other biomes and climates. Such data would also provide a more representative picture of how cues affect phenology in different parts of the world and would allow addressing questions such as, is variability in cue sensitivity across species larger in temperate than tropical latitudes? Third, we provide a table (\ref{tab:ref}) identifying the original sources from which the data were compiled. The data can be accessed at \url{https://knb.ecoinformatics.org/view/doi:10.5063/F18P5XZF}.


  
\subsection{Interpretation of $\lambda_j$ and $\sigma_j^2$ on slopes and intercepts}

Most current phylogenetic regression approaches aimed at controlling for phylogenetic non-independence of analysis units \citep[i.e. usually species, see][]{revell2010phylogenetic} assume the $\lambda$ scaling parameter is constant across the full set of predictors in the model. Thus, $\lambda$ is estimated as a single parameter based on one single residual term VCV matrix. While useful for correcting for phylogenetic non-independence, this approach does not allow for different tempos and modes of evolution across different predictors. 

In models with multiple cues, species responses to all cues are estimated as similarly phylogenetically structured, but this may not be the case. For example, in a PGLS model with three cues, it would be possible to have a high (close to 1) value of $\lambda$, due to either a strong phylogenetic signal in the response, but no phylogenetic structuring in the cues, or one or more predictors being strongly phylogenetically structured. In the latter case, phylogenetic structuring of responses to cues could be correlated (i.e., responses to cues evolving in a correlated fashion) or uncorrelated (i.e., independent evolution of responses to cues). Distinguishing among these different situations is challenging, in part because the phylogeny is constant across all data, and also because in practice multiple process affect the evolution of traits that each leave a signature of some kind in the data. Perhaps, in part, due to the complexity of the problem, most modern approaches take the conservative approach of focusing on whether model residuals are phylogenetically structured (i.e. in PGLS) or the amount of model variance attributable to the phylogeny and independent from other sources of variation \citep[i.e., in PMM, see][]{housworth2004phylogenetic}.

Because we are specifically interested in estimating the phylogenetic structure of each cue, our approach explicitly partitions variance into specific components relative to the model intercept and predictor (cue) slopes (see equation \ref{eqfive}). The multivariate normal distributions of the intercept and slope terms each include a variance term (see equation \ref{phybetas}), modelled with a $\lambda$ scaling parameter. The interpretation of $\lambda$s in our models are analogous to Pagel's $\lambda$  \citep{pagel1999inferring} parameter \citep{housworth2004phylogenetic}, constrained to range from 0 to 1, with values of 0 indicating no phylogenetic signal, and values of 1 phylogenetic signal consistent with Brownian motion evolution (BM). 


While the lambdas estimated through our fitting process will resemble those of lambdas estimated from non-phylogenetically informed models, our approach gives a number of benefits. First, the uncertainty associated with estimating parameters across shared data (the phylogeny) is directly incorporated into the fitting process itself: our posterior estimates are joint across that shared data and the uncertainty it introduces to our fitting process. Second, we expect our cues and their evolution to both be correlated, and assessing both simultaneously allows uncertainty in our ability to unpick precise evolutionary process is, again, incorporated into our uncertainty estimates. Third, this approach adjusts our partial pooling (`random effect' of species) based on evolutionary distance, more strongly pooling closely related species, and only weakly pooling distantly related species \citep[see Gaussian process models in][]{BDA}. This is particularly important for the practising ecologist who, unlike an evolutionary biologist, is not interested in controlling for past evolution \emph{per se}, but is interested in using that past information to predict slopes for (un)measured species on the basis of their evolutionary history.

A traditional interpretation of $\sigma^2$ values under Brownian Motion evolution, is an `evolutionary rate' or phenotypic accumulation over time \citep{revell2008phylogenetic}. In PGLS, $\sigma_\epsilon^2$ is estimated for the model error term, which parameterises a multivariate normal with a VCV matrix given by $\sigma_\epsilon^2$$\boldsymbol{\Sigma_i}$. Here, similar to our approach to $\lambda$, we estimate four $\sigma^2$ values, corresponding to each model parameter. In our particular case (i.e., modelling a phenological response to three environmental cues), $\sigma_\alpha^2$ for the intercept could be interpreted as the phenological variation across species accumulated along evolution independently from the cues, and is perhaps most comparable to the interpretation of lambda in a PGLS as measuring phylogenetic signal in the residuals. The $\sigma_{\beta_{chill}}^2$, $\sigma_{\beta_{force}}^2$, and $\sigma_{\beta_{photo}}^2$, corresponding to model slopes, would represent the phylogenetic variance linked to species responses to each of the modelled cues. This is, the variability in how species shift their phenology responding to temperature and light, accumulated along the evolutionary process and considered in concert. 

\subsection{Does accounting for phylogenetic relationships affect forecasts?}
We forecasted estimated shifts in species phenologies from our phylogenetic and non-phylogenetic models for two species with overlapping European ranges to show the impact of differences across models in a well (\emph{Betula pendula}, $n=311$) versus poorly sampled  (\emph{Acer campestre}, $n=6$) species. For this, we first fit our phylogenetic and non-phylogenetic models using natural (i.e., not z-scored) units. Second, we projected fitted models to the European geographic range of each species using two climate scenarios within species distributions: a scenario of historical climate (1980-2016) and a scenario of 2$^{\circ}$C of warming. These projections yield four predictions for each species: phylogenetic-historical, phylogenetic-warmer, non-phylogenetic-historical and non-phylogenetic-warmer. Third, for each species we compared phylogenetic vs. non-phylogenetic models (see Fig. \ref{fig:pmmvshmm}). Finally, for each species we quantified how phenological shifts expected due to warming differ as a result of using a phylogenetic model instead of a non-phylogenetic model.  

Species distributional data were extracted from published distributional maps \citep{caudullo2017chorological}. We extracted climate data corresponding to all grid cells contained within each species' range from daily gridded meteorological datasets. Specifically, we extracted minimum and maximum daily temperatures from the E-OBS dataset v.25 at 0.25 latitudinal degrees (\url{https://cds.climate.copernicus.eu/cdsapp#!/dataset/insitu-gridded-observations-europe?tab=overview}; last accessed in May 2023). Daily temperatures were used to compute both forcing (mean daily temperature from March 1st through April 30th) and chilling (Utah units from from 1 September through 30 April). Utah units were calculated using the chillR package in R. We used yearly values of forcing and chilling computed for each location (i.e., grid cell) as inputs in the models to predict date of budburst under each scenario for each 12637 locations for \emph{Betula pendula} and 7537 locations for \emph{Acer campestre}. Lastly, we compared model forecasts from phylogenetic and non-phylogenetic models (i.e., by calculating the difference PMM-HMM) and quantified the change in forecasted phenological shifts due to warming resulting from using non-phylogenetic models instead of phylogenetic ones (i.e., calculating the difference $[PMM_{historical} -PMM_{warming}]-[HMM_{historical}-HMM_{warming}]$).

Our forecasted bias from HMM compared to PMM (Fig. \ref{fig:forecast}) would likely extend to many other species, based on shifts in estimated responses to temperature and daylength (model coefficients) across our studied species. Across all 191 species, accounting for phylogenetic structuring shifted many species estimates (Fig. \ref{fig:correls_angio}). Not accounting for phylogeny (i.e., assuming $\lambda$ = 0 as done in HMM) biased model coefficients on average, particularly so for forcing and somewhat less for chilling (Fig. \ref{fig:correls_angio}). Specifically, species sensitivities to forcing and chilling were underestimated on average (model slopes shifted by 7.2\% and 3.7\%, respectively). Sensitivities to photoperiod, which showed weak phylogenetic signal were not biased in non-phylogenetic models (Fig. \ref{fig:correls_angio}), likely associated to their low estimated $\lambda$ values. 

However, as explained in the main text, these biases do not apply homogeneously to all species. Over represented species (i.e., high number of observations) suffer little to no bias if a non-phylogenetic model is used and underrepresented species can experience large shifts when phylogenetic relationships are ignored (see Fig. \ref{fig:pmmvshmm}). Interestingly the bias in forecasts for underrepresented species does not distribute homogeneously across the geography, indicating that ignoring phylogeny can lead to biased forecasts for these species, more so in particular regions (coinciding with coldest locations in our example; Fig. \ref{fig:pmmvshmm}b).\\ 


\subsection{Leave-One-Clade-Out cross validation}

As an additional test to compare our presented phylogenetic mixed model (PMM) versus more traditional hierarchical models (HMM), we developed a cross-validation approach in which we left out data from different clades and assessed model performance. We then tested how well each model predicted held-out responses, and also tested the stability in modelled estimates of species responses to a given cue. Specifically, our Leave-One-Clade-Out cross validation followed these steps:

\begin{enumerate}
\item Iteratively subset the dataset leaving out one genus at a time. For our approach here, we report results from leaving out (iteratively) the 25 genera with the highest number of observations and at least two species. 
\item Run both PMM and HMM for each subset of data recording the posterior distribution of predicted values. 
\item To test how well subset models predict observed responses, we compare the observed response of left out (held out) species against the posterior means of predicted responses for those species, inferred by the subset version of the models that excluded those species. 
\item To test the stability of model coefficients, we compare model coefficients (i.e., slopes) estimated by the subset version of the models and the full models, only for species left in.
\end{enumerate} 

Traditional cross-validation methods generally leave out a randomly selected proportion of the data (e.g., 20\% in 5-fold cross validation) assuming sampling is completely random (that is, that researchers gathering data would be equally likely to select each observation) and thus ignores potentially important structure in the data that may covary with sampling. In contrast, our method leverages phylogenetic structure, dropping out data in a way that may more accurately reflect differences across differently sampled data. Because of the presence of phylogenetic structure in ecological trait data and the tendency for such datasets to sample across taxa unevenly, we often omit sets of taxa (e.g., clades) sharing similar trait values across different natural sampling regimes. Our results show that the outcomes of this structured omission varies for each model. In HMM, excluding a well-sampled clade would tend to skew parameter estimates more (because it is likely to shift the grand mean towards which all species pool) than in PMM, where the partial pooling co-varies with phylogeny and thus species-level estimates are more stable. 

Given the computational resources of leave-one-out  cross-validation, we used a version of the PMM and HMM models that only included an intercept and forcing response (slope). Our results showed that predicted responses by the PMM model were more correlated with the observed responses of held out species than predicted responses by the HMM model (Fig. \ref{fig:LOCO_obsvspred}). Results also supported a higher stability of model coefficients when estimated by PMM compared to by HMM (Fig. \ref{fig:LOCO_slopescors}). 

The algorithms and code to develop these analyses are available in GitHub \url{https://github.com/MoralesCastilla/PhenoPhyloMM/}.


\clearpage



%\bibliography{phylorefs}
\begin{thebibliography}{10}
%\expandafter\ifx\csname url\endcsname\relax
%  \def\url#1{\texttt{#1}}\fi
%\expandafter\ifx\csname urlprefix\endcsname\relax\def\urlprefix{URL }\fi
%\providecommand{\bibinfo}[2]{#2}
%\providecommand{\eprint}[2][]{\url{#2}}

\bibitem{revell2010phylogenetic}
\bibinfo{author}{Revell, L.~J.}
\newblock \bibinfo{title}{Phylogenetic signal and linear regression on species
  data}.
\newblock \emph{\bibinfo{journal}{Methods in Ecology and Evolution}}
  \textbf{\bibinfo{volume}{1}}, \bibinfo{pages}{319--329}
  (\bibinfo{year}{2010}).

\bibitem{housworth2004phylogenetic}
\bibinfo{author}{Housworth, E.~A.}, \bibinfo{author}{Martins, E.~P.} \&
  \bibinfo{author}{Lynch, M.}
\newblock \bibinfo{title}{The phylogenetic mixed model}.
\newblock \emph{\bibinfo{journal}{The American Naturalist}}
  \textbf{\bibinfo{volume}{163}}, \bibinfo{pages}{84--96}
  (\bibinfo{year}{2004}).

\bibitem{pagel1999inferring}
\bibinfo{author}{Pagel, M.}
\newblock \bibinfo{title}{Inferring the historical patterns of biological
  evolution}.
\newblock \emph{\bibinfo{journal}{Nature}} \textbf{\bibinfo{volume}{401}},
  \bibinfo{pages}{877--884} (\bibinfo{year}{1999}).

\bibitem{BDA}
\bibinfo{author}{Gelman, A.} \emph{et~al.}
\newblock \emph{\bibinfo{title}{Bayesian Data Analysis}}
  (\bibinfo{publisher}{CRC Press}, \bibinfo{address}{New York},
  \bibinfo{year}{2014}), \bibinfo{edition}{3rd} edn.

\bibitem{revell2008phylogenetic}
\bibinfo{author}{Revell, L.~J.}, \bibinfo{author}{Harmon, L.~J.} \&
  \bibinfo{author}{Collar, D.~C.}
\newblock \bibinfo{title}{Phylogenetic signal, evolutionary process, and rate}.
\newblock \emph{\bibinfo{journal}{Systematic biology}}
  \textbf{\bibinfo{volume}{57}}, \bibinfo{pages}{591--601}
  (\bibinfo{year}{2008}).

\bibitem{caudullo2017chorological}
\bibinfo{author}{Caudullo, G.}, \bibinfo{author}{Welk, E.} \&
  \bibinfo{author}{San-Miguel-Ayanz, J.}
\newblock \bibinfo{title}{Chorological maps for the main european woody
  species}.
\newblock \emph{\bibinfo{journal}{Data in brief}}
  \textbf{\bibinfo{volume}{12}}, \bibinfo{pages}{662--666}
  (\bibinfo{year}{2017}).

\bibitem{zohner2016}
\bibinfo{author}{Zohner, C.~M.}, \bibinfo{author}{Benito, B.~M.},
  \bibinfo{author}{Svenning, J.~C.} \& \bibinfo{author}{Renner, S.~S.}
\newblock \bibinfo{title}{Day length unlikely to constrain climate-driven
  shifts in leaf-out times of northern woody plants}.
\newblock \emph{\bibinfo{journal}{Nature Climate Change}}
  \textbf{\bibinfo{volume}{6}}, \bibinfo{pages}{1120--1123}
  (\bibinfo{year}{2016}).

\bibitem{Laube:2014a}
\bibinfo{author}{Laube, J.} \emph{et~al.}
\newblock \bibinfo{title}{Chilling outweighs photoperiod in preventing
  precocious spring development}.
\newblock \emph{\bibinfo{journal}{Global Change Biology}}
  \textbf{\bibinfo{volume}{20}}, \bibinfo{pages}{170--182}
  (\bibinfo{year}{2014}).

\bibitem{flynn2018}
\bibinfo{author}{Flynn, D. F.~B.} \& \bibinfo{author}{Wolkovich, E.~M.}
\newblock \bibinfo{title}{Temperature and photoperiod drive spring phenology
  across all species in a temperate forest community}.
\newblock \emph{\bibinfo{journal}{New Phytologist}}
  \textbf{\bibinfo{volume}{219}}, \bibinfo{pages}{1353--1362}
  (\bibinfo{year}{2018}).

\bibitem{Basler:2012}
\bibinfo{author}{Basler, D.} \& \bibinfo{author}{K{\"o}rner, C.}
\newblock \bibinfo{title}{Photoperiod sensitivity of bud burst in 14 temperate
  forest tree species}.
\newblock \emph{\bibinfo{journal}{Agricultural and Forest Meteorology}}
  \textbf{\bibinfo{volume}{165}}, \bibinfo{pages}{73--81}
  (\bibinfo{year}{2012}).

\bibitem{Basler:2014aa}
\bibinfo{author}{Basler, D.} \& \bibinfo{author}{K{\"o}rner, C.}
\newblock \bibinfo{title}{Photoperiod and temperature responses of bud swelling
  and bud burst in four temperate forest tree species}.
\newblock \emph{\bibinfo{journal}{Tree Physiology}}
  \textbf{\bibinfo{volume}{34}}, \bibinfo{pages}{377--388}
  (\bibinfo{year}{2014}).

\bibitem{malyshev2018}
\bibinfo{author}{Malyshev, A.~V.}, \bibinfo{author}{Henry, H.~A.},
  \bibinfo{author}{Bolte, A.}, \bibinfo{author}{Khan, M. A.~A.} \&
  \bibinfo{author}{Kreyling, J.}
\newblock \bibinfo{title}{Temporal photoperiod sensitivity and forcing
  requirements for budburst in temperate tree seedlings}.
\newblock \emph{\bibinfo{journal}{Agricultural and Forest Meteorology}}
  \textbf{\bibinfo{volume}{248}}, \bibinfo{pages}{82--90}
  (\bibinfo{year}{2018}).

\bibitem{nanninga17}
\bibinfo{author}{Nanninga, C.}, \bibinfo{author}{Buyarski, C.~R.},
  \bibinfo{author}{Pretorius, A.~M.} \& \bibinfo{author}{Montgomery, R.~A.}
\newblock \bibinfo{title}{Increased exposure to chilling advances the time to
  budburst in north american tree species}.
\newblock \emph{\bibinfo{journal}{Tree Physiology}}
  \textbf{\bibinfo{volume}{37}}, \bibinfo{pages}{1727--1738}
  (\bibinfo{year}{2017}).

\bibitem{Webb:1977}
\bibinfo{author}{Webb, D.~P.}
\newblock \bibinfo{title}{Root regeneration and bud dormancy of sugar maple,
  silver maple, and white ash seedlings: effects of chilling}.
\newblock \emph{\bibinfo{journal}{Forest Science}}
  \textbf{\bibinfo{volume}{23}}, \bibinfo{pages}{474--483}
  (\bibinfo{year}{1977}).

\bibitem{Heide:1993}
\bibinfo{author}{Heide, O.}
\newblock \bibinfo{title}{Daylength and thermal time responses of budburst
  during dormancy release in some northern deciduous trees}.
\newblock \emph{\bibinfo{journal}{Physiologia Plantarum}}
  \textbf{\bibinfo{volume}{88}}, \bibinfo{pages}{531--540}
  (\bibinfo{year}{1993}).

\bibitem{Myking:1998aa}
\bibinfo{author}{Myking, T.}
\newblock \bibinfo{title}{Interrelations between respiration and dormancy in
  buds of three hardwood species with different chilling requirements for
  dormancy release}.
\newblock \emph{\bibinfo{journal}{Trees}} \textbf{\bibinfo{volume}{12}},
  \bibinfo{pages}{224--229} (\bibinfo{year}{1998}).

\bibitem{Calme:1994aa}
\bibinfo{author}{Calm{\'e}, S.}, \bibinfo{author}{Bigras, F.~J.},
  \bibinfo{author}{Margolis, H.~A.} \& \bibinfo{author}{H{\'e}bert, C.}
\newblock \bibinfo{title}{Frost tolerance and bud dormancy of container-grown
  yellow birch, red oak and sugar maple seedlings}.
\newblock \emph{\bibinfo{journal}{Tree Physiology}}
  \textbf{\bibinfo{volume}{14}}, \bibinfo{pages}{1313--1325}
  (\bibinfo{year}{1994}).

\bibitem{man17}
\bibinfo{author}{Man, R.}, \bibinfo{author}{Lu, P.} \& \bibinfo{author}{Dang,
  Q.-L.}
\newblock \bibinfo{title}{Insufficient chilling effects vary among boreal tree
  species and chilling duration}.
\newblock \emph{\bibinfo{journal}{Frontiers in plant Science}}
  \textbf{\bibinfo{volume}{8}}, \bibinfo{pages}{1354} (\bibinfo{year}{2017}).

\bibitem{Laube:2014b}
\bibinfo{author}{Laube, J.}, \bibinfo{author}{Sparks, T.~H.},
  \bibinfo{author}{Estrella, N.} \& \bibinfo{author}{Menzel, A.}
\newblock \bibinfo{title}{Does humidity trigger tree phenology? {Proposal} for
  an air humidity based framework for bud development in spring}.
\newblock \emph{\bibinfo{journal}{New Phytologist}}
  \textbf{\bibinfo{volume}{202}}, \bibinfo{pages}{350--355}
  (\bibinfo{year}{2014}).

\bibitem{Li:2005aa}
\bibinfo{author}{Li, C.} \emph{et~al.}
\newblock \bibinfo{title}{{Differential responses of silver birch (\emph{Betula
  pendula}) ecotypes to short-day photoperiod and low temperature}}.
\newblock \emph{\bibinfo{journal}{Tree physiology}}
  \textbf{\bibinfo{volume}{25}}, \bibinfo{pages}{1563--1569}
  (\bibinfo{year}{2005}).

\bibitem{Linkosalo:2006aa}
\bibinfo{author}{Linkosalo, T.} \& \bibinfo{author}{Lechowicz, M.~J.}
\newblock \bibinfo{title}{{Twilight far-red treatment advances leaf bud burst
  of silver birch (\emph{Betula pendula})}}.
\newblock \emph{\bibinfo{journal}{Tree Physiology}}
  \textbf{\bibinfo{volume}{26}}, \bibinfo{pages}{1249--1256}
  (\bibinfo{year}{2006}).

\bibitem{Myking:1995}
\bibinfo{author}{Myking, T.} \& \bibinfo{author}{Heide, O.}
\newblock \bibinfo{title}{{Dormancy release and chilling requirement of buds of
  latitudinal ecotypes of \textit{Betula pendula} and \textit{B. pubescens}}}.
\newblock \emph{\bibinfo{journal}{Tree Physiology}}
  \textbf{\bibinfo{volume}{15}}, \bibinfo{pages}{697--704}
  (\bibinfo{year}{1995}).

\bibitem{Rinne:1997aa}
\bibinfo{author}{Rinne, P.}, \bibinfo{author}{H{\"a}nninen, H.},
  \bibinfo{author}{Kaikuranta, P.}, \bibinfo{author}{Jalonen, J.} \&
  \bibinfo{author}{Repo, T.}
\newblock \bibinfo{title}{{Freezing exposure releases bud dormancy in
  \emph{Betula pubescens} and \emph{B. pendula}}}.
\newblock \emph{\bibinfo{journal}{Plant, Cell \& Environment}}
  \textbf{\bibinfo{volume}{20}}, \bibinfo{pages}{1199--1204}
  (\bibinfo{year}{1997}).

\bibitem{Skuterud:1994aa}
\bibinfo{author}{Skuterud, R.} \& \bibinfo{author}{Dietrichson, J.}
\newblock \bibinfo{title}{{Budburst in detached birch shoots (\emph{Betula
  pendula}) of different varieties winter-stored in darkness at three different
  temperatures.}}
\newblock \emph{\bibinfo{journal}{Silva Fennica}}
  \textbf{\bibinfo{volume}{28}}, \bibinfo{pages}{223--224}
  (\bibinfo{year}{1994}).

\bibitem{Caffarra:2011a}
\bibinfo{author}{Caffarra, A.} \& \bibinfo{author}{Donnelly, A.}
\newblock \bibinfo{title}{The ecological significance of phenology in four
  different tree species: effects of light and temperature on bud burst}.
\newblock \emph{\bibinfo{journal}{International Journal of Biometeorology}}
  \textbf{\bibinfo{volume}{55}}, \bibinfo{pages}{711--721}
  (\bibinfo{year}{2011}).

\bibitem{Caffarra:2011b}
\bibinfo{author}{Caffarra, A.}, \bibinfo{author}{Donnelly, A.},
  \bibinfo{author}{Chuine, I.} \& \bibinfo{author}{Jones, M.~B.}
\newblock \bibinfo{title}{{Modelling the timing of \textit{Betula pubescens}
  bud-burst. I. Temperature and photoperiod: A conceptual model}}.
\newblock \emph{\bibinfo{journal}{Climate Research}}
  \textbf{\bibinfo{volume}{46}}, \bibinfo{pages}{147} (\bibinfo{year}{2011}).

\bibitem{Myking:1997aa}
\bibinfo{author}{Myking, T.}
\newblock \bibinfo{title}{{Effects of constant and fluctuating temperature on
  time to budburst in \emph{Betula pubescens} and its relation to bud
  respiration}}.
\newblock \emph{\bibinfo{journal}{Trees}} \textbf{\bibinfo{volume}{12}},
  \bibinfo{pages}{107--112} (\bibinfo{year}{1997}).

\bibitem{Rinne:1994}
\bibinfo{author}{Rinne, P.}, \bibinfo{author}{Saarelainen, A.} \&
  \bibinfo{author}{Junttila, O.}
\newblock \bibinfo{title}{{Growth cessation and bud dormancy in relation to ABA
  level in seedlings and coppice shoots of \emph{Betula pubescens} as affected
  by a short photoperiod, water stress and chilling}}.
\newblock \emph{\bibinfo{journal}{Physiologia Plantarum}}
  \textbf{\bibinfo{volume}{90}}, \bibinfo{pages}{451--458}
  (\bibinfo{year}{1994}).

\bibitem{Heide:1993a}
\bibinfo{author}{Heide, O.}
\newblock \bibinfo{title}{{Dormancy release in beech buds (\textit{Fagus
  sylvatica}) requires both chilling and long days}}.
\newblock \emph{\bibinfo{journal}{Physiologia Plantarum}}
  \textbf{\bibinfo{volume}{89}}, \bibinfo{pages}{187--191}
  (\bibinfo{year}{1993}).

\bibitem{vitra17}
\bibinfo{author}{Vitra, A.}, \bibinfo{author}{Lenz, A.} \&
  \bibinfo{author}{Vitasse, Y.}
\newblock \bibinfo{title}{Frost hardening and dehardening potential in
  temperate trees from winter to budburst}.
\newblock \emph{\bibinfo{journal}{New Phytologist}}
  \textbf{\bibinfo{volume}{216}}, \bibinfo{pages}{113--123}
  (\bibinfo{year}{2017}).

\bibitem{Falusi:1990aa}
\bibinfo{author}{Falusi, M.} \& \bibinfo{author}{Calamassi, R.}
\newblock \bibinfo{title}{{Bud dormancy in beech (\textit{Fagus sylvatica} L.).
  Effect of chilling and photoperiod on dormancy release of beech seedlings}}.
\newblock \emph{\bibinfo{journal}{Tree Physiology}}
  \textbf{\bibinfo{volume}{6}}, \bibinfo{pages}{429--438}
  (\bibinfo{year}{1990}).

\bibitem{Falusi:1996aa}
\bibinfo{author}{Falusi, M.} \& \bibinfo{author}{Calamassi, R.}
\newblock \bibinfo{title}{{Geographic variation and bud dormancy in beech
  seedlings (\emph{Fagus sylvatica} L)}}.
\newblock In \emph{\bibinfo{booktitle}{Annales des Sciences foresti{\`e}res}},
  vol.~\bibinfo{volume}{53}, \bibinfo{pages}{967--979} (\bibinfo{publisher}{EDP
  Sciences}, \bibinfo{year}{1996}).

\bibitem{Falusi:1997aa}
\bibinfo{author}{Falusi, M.} \& \bibinfo{author}{Calamassi, R.}
\newblock \bibinfo{title}{{Bud dormancy in \emph{Fagus sylvatica} L. II. The
  evolution of dormancy in seedlings and one-node cuttings}}.
\newblock \emph{\bibinfo{journal}{Plant Biosystems-An International Journal
  Dealing with all Aspects of Plant Biology}} \textbf{\bibinfo{volume}{131}},
  \bibinfo{pages}{143--148} (\bibinfo{year}{1997}).

\bibitem{Falusi:2003aa}
\bibinfo{author}{Falusi, M.} \& \bibinfo{author}{Calamassi, R.}
\newblock \bibinfo{title}{{Dormancy of \emph{Fagus sylvatica} L. buds III.
  Temperature and hormones in the evolution of dormancy in one-node cuttings}}.
\newblock \emph{\bibinfo{journal}{Plant Biosystems-An International Journal
  Dealing with all Aspects of Plant Biology}} \textbf{\bibinfo{volume}{137}},
  \bibinfo{pages}{185--191} (\bibinfo{year}{2003}).

\bibitem{Thielges:1976aa}
\bibinfo{author}{Thielges, B.} \& \bibinfo{author}{Beck, R.}
\newblock \bibinfo{title}{{Control of bud break and its inheritance in
  \emph{Populus deltoides}}}.
\newblock \emph{\bibinfo{journal}{Tree Physiology and Yield Improvement}}
  \textbf{\bibinfo{volume}{14}}, \bibinfo{pages}{253--259}
  (\bibinfo{year}{1976}).

\bibitem{Sanz-Perez:2010aa}
\bibinfo{author}{Sanz-P{\'e}rez, V.} \& \bibinfo{author}{Castro-D{\'\i}ez, P.}
\newblock \bibinfo{title}{Summer water stress and shade alter bud size and
  budburst date in three {Mediterranean \emph{Quercus}} species}.
\newblock \emph{\bibinfo{journal}{Trees}} \textbf{\bibinfo{volume}{24}},
  \bibinfo{pages}{89--97} (\bibinfo{year}{2010}).

\bibitem{Sanz-Perez:2009aa}
\bibinfo{author}{Sanz-Perez, V.}, \bibinfo{author}{Castro-Diez, P.} \&
  \bibinfo{author}{Valladares, F.}
\newblock \bibinfo{title}{Differential and interactive effects of temperature
  and photoperiod on budburst and carbon reserves in two co-occurring
  {M}editerranean oaks}.
\newblock \emph{\bibinfo{journal}{Plant Biology}}
  \textbf{\bibinfo{volume}{11}}, \bibinfo{pages}{142--51}
  (\bibinfo{year}{2009}).

\bibitem{Morin:2010aa}
\bibinfo{author}{Morin, X.}, \bibinfo{author}{Roy, J.},
  \bibinfo{author}{Soni{\'e}, L.} \& \bibinfo{author}{Chuine, I.}
\newblock \bibinfo{title}{{Changes in leaf phenology of three European oak
  species in response to experimental climate change}}.
\newblock \emph{\bibinfo{journal}{New Phytologist}}
  \textbf{\bibinfo{volume}{186}}, \bibinfo{pages}{900--910}
  (\bibinfo{year}{2010}).

\bibitem{Ghelardini:2010aa}
\bibinfo{author}{Ghelardini, L.}, \bibinfo{author}{Santini, A.},
  \bibinfo{author}{Black-Samuelsson, S.}, \bibinfo{author}{Myking, T.} \&
  \bibinfo{author}{Falusi, M.}
\newblock \bibinfo{title}{Bud dormancy release in elm ( {\emph{ulmus}} spp.)
  clones---a case study of photoperiod and temperature responses}.
\newblock \emph{\bibinfo{journal}{Tree physiology}}
  \textbf{\bibinfo{volume}{30}}, \bibinfo{pages}{264--274}
  (\bibinfo{year}{2010}).

\end{thebibliography}


%%%%%%%%%%%%%%%%%%%%%%%%%%%%%%%
% Supporting Figures and Tables
%%%%%%%%%%%%%%%%%%%%%%%%%%%%%%%
\clearpage
\section{Supplementary Tables}



\begingroup\footnotesize
\begin{longtable}{p{0.3\textwidth}p{0.15\textwidth}p{0.40\textwidth}}
\caption{Species and references for data included in the phylogenetic model using Utah chill units.} \\ 
  \hline
\emph{Species} & \emph{Num.Studies} & \emph{Reference} \\ 
  \hline \endhead  \hline
\emph{Acer barbinerve} &   1 & \emph{\citep{zohner2016}} \\ 
  \emph{Acer campestre} &   1 & \emph{\citep{zohner2016}} \\ 
  \emph{Acer ginnala} &   1 & \emph{\citep{zohner2016}} \\ 
  \emph{Acer negundo} &   1 & \emph{\citep{Laube:2014a}} \\ 
  \emph{Acer pensylvanicum} &   1 & \emph{\citep{flynn2018}} \\ 
  \emph{Acer platanoides} &   1 & \emph{\citep{zohner2016}} \\ 
  \emph{Acer pseudoplatanus} &   4 & \emph{\citep{Basler:2012,Basler:2014aa,Laube:2014a,malyshev2018}} \\ 
  \emph{Acer rubrum} &   2 & \emph{\citep{flynn2018,nanninga17}} \\ 
  \emph{Acer saccharinum} &   1 & \emph{\citep{Webb:1977}} \\ 
  \emph{Acer tataricum} &   1 & \emph{\citep{Laube:2014a}} \\ 
  \emph{Aesculus flava} &   1 & \emph{\citep{zohner2016}} \\ 
  \emph{Aesculus hippocastanum} &   3 & \emph{\citep{Basler:2012,Laube:2014a,zohner2016}} \\ 
  \emph{Aesculus parviflora} &   1 & \emph{\citep{zohner2016}} \\ 
  \emph{Alnus glutinosa} &   2 & \emph{\citep{Heide:1993,Myking:1998aa}} \\ 
  \emph{Alnus incana} &   3 & \emph{\citep{flynn2018,Heide:1993,zohner2016}} \\ 
  \emph{Alnus maximowiczii} &   1 & \emph{\citep{zohner2016}} \\ 
  \emph{Amelanchier florida} &   1 & \emph{\citep{zohner2016}} \\ 
  \emph{Amelanchier laevis} &   1 & \emph{\citep{zohner2016}} \\ 
  \emph{Amorpha fruticosa} &   1 & \emph{\citep{Laube:2014a}} \\ 
  \emph{Aronia melanocarpa} &   2 & \emph{\citep{flynn2018,zohner2016}} \\ 
  \emph{Berberis dielsiana} &   1 & \emph{\citep{zohner2016}} \\ 
  \emph{Betula alleghaniensis} &   2 & \emph{\citep{Calme:1994aa,flynn2018}} \\ 
  \emph{Betula lenta} &   2 & \emph{\citep{flynn2018,zohner2016}} \\ 
  \emph{Betula nana} &   1 & \emph{\citep{zohner2016}} \\ 
  \emph{Betula papyrifera} &   3 & \emph{\citep{flynn2018,man17,nanninga17}} \\ 
  \emph{Betula pendula} &  10 & \emph{\citep{Basler:2012,Heide:1993,Laube:2014a,Laube:2014b,Li:2005aa,Linkosalo:2006aa,Myking:1995,Myking:1998aa,Rinne:1997aa,Skuterud:1994aa}} \\ 
  \emph{Betula populifolia} &   1 & \emph{\citep{zohner2016}} \\ 
  \emph{Betula pubescens} &   6 & \emph{\citep{Caffarra:2011a,Caffarra:2011b,Heide:1993,Myking:1995,Myking:1997aa,Rinne:1994}} \\ 
  \emph{Buddleja albiflora} &   1 & \emph{\citep{zohner2016}} \\ 
  \emph{Buddleja alternifolia} &   1 & \emph{\citep{zohner2016}} \\ 
  \emph{Buddleja davidii} &   1 & \emph{\citep{zohner2016}} \\ 
  \emph{Caragana pygmaea} &   1 & \emph{\citep{zohner2016}} \\ 
  \emph{Carpinus betulus} &   4 & \emph{\citep{Heide:1993a,Laube:2014a,vitra17,zohner2016}} \\ 
  \emph{Carpinus laxiflora} &   1 & \emph{\citep{zohner2016}} \\ 
  \emph{Carpinus monbeigiana} &   1 & \emph{\citep{zohner2016}} \\ 
  \emph{Carya cordiformis} &   1 & \emph{\citep{zohner2016}} \\ 
  \emph{Carya laciniosa} &   1 & \emph{\citep{zohner2016}} \\ 
  \emph{Celtis caucasica} &   1 & \emph{\citep{zohner2016}} \\ 
  \emph{Celtis laevigata} &   1 & \emph{\citep{zohner2016}} \\ 
  \emph{Celtis occidentalis} &   1 & \emph{\citep{zohner2016}} \\ 
  \emph{Cephalanthus occidentalis} &   1 & \emph{\citep{zohner2016}} \\ 
  \emph{Cercidiphyllum japonicum} &   1 & \emph{\citep{zohner2016}} \\ 
  \emph{Cercidiphyllum magnificum} &   1 & \emph{\citep{zohner2016}} \\ 
  \emph{Cercis canadensis} &   1 & \emph{\citep{zohner2016}} \\ 
  \emph{Cercis chinensis} &   1 & \emph{\citep{zohner2016}} \\ 
  \emph{Cladrastis lutea} &   1 & \emph{\citep{zohner2016}} \\ 
  \emph{Cornus alba} &   2 & \emph{\citep{Laube:2014a,zohner2016}} \\ 
  \emph{Cornus kousa} &   1 & \emph{\citep{zohner2016}} \\ 
  \emph{Cornus mas} &   2 & \emph{\citep{Laube:2014a,Laube:2014b}} \\ 
  \emph{Corylopsis sinensis} &   1 & \emph{\citep{zohner2016}} \\ 
  \emph{Corylopsis spicata} &   1 & \emph{\citep{zohner2016}} \\ 
  \emph{Corylus cornuta} &   1 & \emph{\citep{flynn2018}} \\ 
  \emph{Decaisnea fargesii} &   1 & \emph{\citep{zohner2016}} \\ 
  \emph{Deutzia gracilis} &   1 & \emph{\citep{zohner2016}} \\ 
  \emph{Deutzia scabra} &   1 & \emph{\citep{zohner2016}} \\ 
  \emph{Elaeagnus ebbingei} &   1 & \emph{\citep{zohner2016}} \\ 
  \emph{Eleutherococcus senticosus} &   1 & \emph{\citep{zohner2016}} \\ 
  \emph{Eleutherococcus setchuenensis} &   1 & \emph{\citep{zohner2016}} \\ 
  \emph{Eleutherococcus sieboldianus} &   1 & \emph{\citep{zohner2016}} \\ 
  \emph{Euonymus europaeus} &   1 & \emph{\citep{zohner2016}} \\ 
  \emph{Euonymus latifolius} &   1 & \emph{\citep{zohner2016}} \\ 
  \emph{Fagus crenata} &   1 & \emph{\citep{zohner2016}} \\ 
  \emph{Fagus engleriana} &   1 & \emph{\citep{zohner2016}} \\ 
  \emph{Fagus grandifolia} &   1 & \emph{\citep{flynn2018}} \\ 
  \emph{Fagus orientalis} &   1 & \emph{\citep{zohner2016}} \\ 
  \emph{Fagus sylvatica} &  12 & \emph{\citep{Basler:2012,Basler:2014aa,Caffarra:2011a,Falusi:1990aa,Falusi:1996aa,Falusi:1997aa,Falusi:2003aa,Heide:1993a,Laube:2014a,malyshev2018,vitra17,zohner2016}} \\ 
  \emph{Forsythia ovata} &   1 & \emph{\citep{zohner2016}} \\ 
  \emph{Forsythia suspensa} &   1 & \emph{\citep{zohner2016}} \\ 
  \emph{Fraxinus americana} &   1 & \emph{\citep{Webb:1977}} \\ 
  \emph{Fraxinus chinensis} &   1 & \emph{\citep{Laube:2014a}} \\ 
  \emph{Fraxinus excelsior} &   2 & \emph{\citep{Basler:2012,Laube:2014a}} \\ 
  \emph{Fraxinus latifolia} &   1 & \emph{\citep{zohner2016}} \\ 
  \emph{Fraxinus nigra} &   1 & \emph{\citep{flynn2018}} \\ 
  \emph{Fraxinus ornus} &   1 & \emph{\citep{zohner2016}} \\ 
  \emph{Fraxinus pennsylvanica} &   1 & \emph{\citep{Laube:2014a}} \\ 
  \emph{Hamamelis japonica} &   1 & \emph{\citep{zohner2016}} \\ 
  \emph{Hamamelis vernalis} &   1 & \emph{\citep{zohner2016}} \\ 
  \emph{Hamamelis virginiana} &   1 & \emph{\citep{flynn2018}} \\ 
  \emph{Heptacodium miconioides} &   1 & \emph{\citep{zohner2016}} \\ 
  \emph{Hibiscus syriacus} &   1 & \emph{\citep{zohner2016}} \\ 
  \emph{Hydrangea arborescens} &   1 & \emph{\citep{zohner2016}} \\ 
  \emph{Hydrangea involucrata} &   1 & \emph{\citep{zohner2016}} \\ 
  \emph{Hydrangea serrata} &   1 & \emph{\citep{zohner2016}} \\ 
  \emph{Ilex mucronata} &   1 & \emph{\citep{flynn2018}} \\ 
  \emph{Juglans ailantifolia} &   1 & \emph{\citep{Laube:2014a}} \\ 
  \emph{Juglans cinerea} &   1 & \emph{\citep{Laube:2014a}} \\ 
  \emph{Kalmia angustifolia} &   1 & \emph{\citep{flynn2018}} \\ 
  \emph{Ligustrum tschonoskii} &   1 & \emph{\citep{zohner2016}} \\ 
  \emph{Liquidambar orientalis} &   1 & \emph{\citep{zohner2016}} \\ 
  \emph{Liquidambar styraciflua} &   1 & \emph{\citep{zohner2016}} \\ 
  \emph{Liriodendron tulipifera} &   1 & \emph{\citep{zohner2016}} \\ 
  \emph{Lonicera alpigena} &   1 & \emph{\citep{zohner2016}} \\ 
  \emph{Lonicera caerulea} &   1 & \emph{\citep{zohner2016}} \\ 
  \emph{Lonicera canadensis} &   1 & \emph{\citep{flynn2018}} \\ 
  \emph{Lonicera maximowiczii} &   1 & \emph{\citep{zohner2016}} \\ 
  \emph{Lyonia ligustrina} &   1 & \emph{\citep{flynn2018}} \\ 
  \emph{Nothofagus antarctica} &   1 & \emph{\citep{zohner2016}} \\ 
  \emph{Nyssa sylvatica} &   1 & \emph{\citep{flynn2018}} \\ 
  \emph{Oemleria cerasiformis} &   1 & \emph{\citep{zohner2016}} \\ 
  \emph{Orixa japonica} &   1 & \emph{\citep{zohner2016}} \\ 
  \emph{Ostrya carpinifolia} &   1 & \emph{\citep{zohner2016}} \\ 
  \emph{Ostrya virginiana} &   1 & \emph{\citep{zohner2016}} \\ 
  \emph{Paeonia rockii} &   1 & \emph{\citep{zohner2016}} \\ 
  \emph{Parrotia persica} &   1 & \emph{\citep{zohner2016}} \\ 
  \emph{Parrotiopsis jaquemontiana} &   1 & \emph{\citep{zohner2016}} \\ 
  \emph{Photinia villosa} &   1 & \emph{\citep{zohner2016}} \\ 
  \emph{Populus deltoides} &   1 & \emph{\citep{Thielges:1976aa}} \\ 
  \emph{Populus grandidentata} &   1 & \emph{\citep{flynn2018}} \\ 
  \emph{Populus koreana} &   1 & \emph{\citep{zohner2016}} \\ 
  \emph{Populus tremula} &   3 & \emph{\citep{Heide:1993,Laube:2014a,Laube:2014b}} \\ 
  \emph{Prinsepia sinensis} &   1 & \emph{\citep{zohner2016}} \\ 
  \emph{Prinsepia uniflora} &   1 & \emph{\citep{zohner2016}} \\ 
  \emph{Prunus padus} &   3 & \emph{\citep{Heide:1993,Myking:1998aa,zohner2016}} \\ 
  \emph{Prunus pensylvanica} &   1 & \emph{\citep{flynn2018}} \\ 
  \emph{Prunus serotina} &   1 & \emph{\citep{Laube:2014a}} \\ 
  \emph{Prunus serrulata} &   1 & \emph{\citep{zohner2016}} \\ 
  \emph{Prunus tenella} &   1 & \emph{\citep{zohner2016}} \\ 
  \emph{Ptelea trifoliata} &   1 & \emph{\citep{zohner2016}} \\ 
  \emph{Pyrus elaeagnifolia} &   1 & \emph{\citep{zohner2016}} \\ 
  \emph{Pyrus ussuriensis} &   1 & \emph{\citep{zohner2016}} \\ 
  \emph{Quercus alba} &   1 & \emph{\citep{flynn2018}} \\ 
  \emph{Quercus bicolor} &   1 & \emph{\citep{Laube:2014a}} \\ 
  \emph{Quercus coccifera} &   1 & \emph{\citep{Sanz-Perez:2010aa}} \\ 
  \emph{Quercus ellipsoidalis} &   1 & \emph{\citep{nanninga17}} \\ 
  \emph{Quercus faginea} &   2 & \emph{\citep{Sanz-Perez:2009aa,Sanz-Perez:2010aa}} \\ 
  \emph{Quercus ilex} &   3 & \emph{\citep{Morin:2010aa,Sanz-Perez:2009aa,Sanz-Perez:2010aa}} \\ 
  \emph{Quercus petraea} &   3 & \emph{\citep{Basler:2012,Basler:2014aa,vitra17}} \\ 
  \emph{Quercus pubescens} &   1 & \emph{\citep{Morin:2010aa}} \\ 
  \emph{Quercus robur} &   5 & \emph{\citep{Laube:2014a,Laube:2014b,malyshev2018,Morin:2010aa,zohner2016}} \\ 
  \emph{Quercus rubra} &   3 & \emph{\citep{Calme:1994aa,flynn2018,Laube:2014a}} \\ 
  \emph{Quercus shumardii} &   1 & \emph{\citep{zohner2016}} \\ 
  \emph{Quercus velutina} &   1 & \emph{\citep{flynn2018}} \\ 
  \emph{Rhamnus alpina} &   1 & \emph{\citep{zohner2016}} \\ 
  \emph{Rhamnus cathartica} &   2 & \emph{\citep{nanninga17,zohner2016}} \\ 
  \emph{Rhamnus frangula} &   1 & \emph{\citep{flynn2018}} \\ 
  \emph{Rhododendron canadense} &   1 & \emph{\citep{zohner2016}} \\ 
  \emph{Rhododendron dauricum} &   1 & \emph{\citep{zohner2016}} \\ 
  \emph{Rhododendron mucronulatum} &   1 & \emph{\citep{zohner2016}} \\ 
  \emph{Rhododendron prinophyllum} &   1 & \emph{\citep{flynn2018}} \\ 
  \emph{Ribes alpinum} &   1 & \emph{\citep{zohner2016}} \\ 
  \emph{Ribes divaricatum} &   1 & \emph{\citep{zohner2016}} \\ 
  \emph{Ribes glaciale} &   1 & \emph{\citep{zohner2016}} \\ 
  \emph{Robinia pseudoacacia} &   2 & \emph{\citep{Laube:2014a,Laube:2014b}} \\ 
  \emph{Rosa hugonis} &   1 & \emph{\citep{zohner2016}} \\ 
  \emph{Rosa majalis} &   1 & \emph{\citep{zohner2016}} \\ 
  \emph{Salix gracilistyla} &   1 & \emph{\citep{zohner2016}} \\ 
  \emph{Salix repens} &   1 & \emph{\citep{zohner2016}} \\ 
  \emph{Salix smithiana} &   1 & \emph{\citep{Caffarra:2011a}} \\ 
  \emph{Sambucus pubens} &   1 & \emph{\citep{zohner2016}} \\ 
  \emph{Sambucus tigranii} &   1 & \emph{\citep{zohner2016}} \\ 
  \emph{Sinowilsonia henryi} &   1 & \emph{\citep{zohner2016}} \\ 
  \emph{Sorbus aria} &   1 & \emph{\citep{zohner2016}} \\ 
  \emph{Sorbus aucuparia} &   2 & \emph{\citep{Basler:2012,Heide:1993}} \\ 
  \emph{Sorbus commixta} &   1 & \emph{\citep{zohner2016}} \\ 
  \emph{Sorbus decora} &   1 & \emph{\citep{zohner2016}} \\ 
  \emph{Sorbus torminalis} &   1 & \emph{\citep{malyshev2018}} \\ 
  \emph{Spiraea canescens} &   1 & \emph{\citep{zohner2016}} \\ 
  \emph{Spiraea chamaedryfolia} &   1 & \emph{\citep{zohner2016}} \\ 
  \emph{Spiraea japonica} &   1 & \emph{\citep{zohner2016}} \\ 
  \emph{Spirea alba} &   1 & \emph{\citep{flynn2018}} \\ 
  \emph{Stachyurus praecox} &   1 & \emph{\citep{zohner2016}} \\ 
  \emph{Stachyurus sinensis} &   1 & \emph{\citep{zohner2016}} \\ 
  \emph{Symphoricarpos albus} &   2 & \emph{\citep{Laube:2014a,Laube:2014b}} \\ 
  \emph{Syringa josikaea} &   1 & \emph{\citep{zohner2016}} \\ 
  \emph{Syringa reticulata} &   1 & \emph{\citep{zohner2016}} \\ 
  \emph{Syringa villosa} &   1 & \emph{\citep{zohner2016}} \\ 
  \emph{Syringa vulgaris} &   3 & \emph{\citep{Basler:2012,Laube:2014a,Laube:2014b}} \\ 
  \emph{Tilia cordata} &   3 & \emph{\citep{Basler:2012,Caffarra:2011a,malyshev2018}} \\ 
  \emph{Tilia dasystyla} &   1 & \emph{\citep{zohner2016}} \\ 
  \emph{Tilia japonica} &   1 & \emph{\citep{zohner2016}} \\ 
  \emph{Tilia platyphyllos} &   1 & \emph{\citep{zohner2016}} \\ 
  \emph{Toona sinensis} &   1 & \emph{\citep{zohner2016}} \\ 
  \emph{Ulmus americana} &   1 & \emph{\citep{zohner2016}} \\ 
  \emph{Ulmus glabra} &   1 & \emph{\citep{Ghelardini:2010aa}} \\ 
  \emph{Ulmus laevis} &   1 & \emph{\citep{zohner2016}} \\ 
  \emph{Ulmus macrocarpa} &   1 & \emph{\citep{Ghelardini:2010aa}} \\ 
  \emph{Ulmus minor} &   1 & \emph{\citep{Ghelardini:2010aa}} \\ 
  \emph{Ulmus parvifolia} &   1 & \emph{\citep{Ghelardini:2010aa}} \\ 
  \emph{Ulmus pumila} &   1 & \emph{\citep{Ghelardini:2010aa}} \\ 
  \emph{Ulmus villosa} &   1 & \emph{\citep{Ghelardini:2010aa}} \\ 
  \emph{Vaccinium myrtilloides} &   1 & \emph{\citep{flynn2018}} \\ 
  \emph{Viburnum betulifolium} &   1 & \emph{\citep{zohner2016}} \\ 
  \emph{Viburnum buddleifolium} &   1 & \emph{\citep{zohner2016}} \\ 
  \emph{Viburnum carlesii} &   1 & \emph{\citep{zohner2016}} \\ 
  \emph{Viburnum cassinoides} &   1 & \emph{\citep{flynn2018}} \\ 
  \emph{Viburnum lantanoides} &   1 & \emph{\citep{flynn2018}} \\ 
  \emph{Viburnum opulus} &   1 & \emph{\citep{zohner2016}} \\ 
  \emph{Viburnum plicatum} &   1 & \emph{\citep{zohner2016}} \\ 
  \emph{Weigela coraeensis} &   1 & \emph{\citep{zohner2016}} \\ 
  \emph{Weigela florida} &   1 & \emph{\citep{zohner2016}} \\ 
  \emph{Weigela maximowiczii} &   1 & \emph{\citep{zohner2016}} \\ 
  \hline
\label{tab:ref}
\end{longtable}
\endgroup


\clearpage



% latex table generated in R 4.0.4 by xtable 1.8-4 package
% Sat Jun 15 02:27:55 2024
\begin{table}[ht]
\centering
\caption{Model parameters estimated for 191 tree species including mean, standard deviation (sd), 2.5\%, 50\%, and 97.5\% uncertainty intervals (z-scored model, thus predictors are directly comparable to one another), alongside model diagnostics.} 
\label{tab:modelanglamb}
\begingroup\footnotesize
\begin{tabular}{lrrrrrrr}
  \hline
Parameter & mean & sd & 2.5\% & 50\% & 97.5\% & $n_{eff}$ & Rhat \\ 
  \hline
$\mu_{\alpha}$ & 30.63 & 3.41 & 23.94 & 30.66 & 37.26 & 12315.84 & 1.00 \\ 
  $\mu_{\beta_{force}}$ & -6.12 & 2.11 & -10.24 & -6.15 & -1.85 & 3989.87 & 1.00 \\ 
  $\mu_{\beta_{chill}}$ & -6.86 & 2.18 & -10.98 & -6.91 & -2.39 & 7444.80 & 1.00 \\ 
  $\mu_{\beta_{photo}}$ & -1.22 & 0.77 & -2.73 & -1.22 & 0.36 & 2482.96 & 1.00 \\ 
  $\lambda_{\alpha}$ & 0.34 & 0.10 & 0.16 & 0.34 & 0.55 & 7668.82 & 1.00 \\ 
  $\lambda_{\beta_{force}}$ & 0.65 & 0.20 & 0.22 & 0.67 & 0.97 & 630.96 & 1.01 \\ 
  $\lambda_{\beta_{chill}}$ & 0.54 & 0.15 & 0.25 & 0.55 & 0.82 & 1834.14 & 1.00 \\ 
  $\lambda_{\beta_{photo}}$ & 0.40 & 0.24 & 0.03 & 0.38 & 0.88 & 672.39 & 1.00 \\ 
  $\sigma_{\alpha}$ & 15.99 & 1.15 & 13.98 & 15.91 & 18.47 & 6970.37 & 1.00 \\ 
  $\sigma_{\beta_{force}}$ & 5.80 & 1.01 & 4.06 & 5.70 & 8.01 & 1043.34 & 1.00 \\ 
  $\sigma_{\beta_{chill}}$ & 7.10 & 0.88 & 5.53 & 7.04 & 8.99 & 1767.13 & 1.00 \\ 
  $\sigma_{\beta_{photo}}$ & 2.36 & 0.41 & 1.61 & 2.34 & 3.23 & 636.82 & 1.01 \\ 
  $\sigma_y$ & 12.58 & 0.18 & 12.24 & 12.58 & 12.93 & 10904.90 & 1.00 \\ 
   \hline
\end{tabular}
\endgroup
\end{table} \clearpage \pagebreak 
% latex table generated in R 4.0.4 by xtable 1.8-4 package
% Sat Jun 15 02:27:55 2024
\begin{table}[ht]
\centering
\caption{Model parameters for non-phylogenetic model ($\lambda$ = 0) estimated for 191 tree species including mean, standard deviation (sd), 2.5\%, 50\%, and 97.5\% uncertainty intervals (z-scored model, thus predictors are directly comparable to one another), alongside model diagnostics.} 
\label{tab:modlamb0}
\begingroup\footnotesize
\begin{tabular}{lrrrrrrr}
  \hline
Parameter & mean & sd & 2.5\% & 50\% & 97.5\% & $n_{eff}$ & Rhat \\ 
  \hline
$\mu_{\alpha}$ & 31.79 & 1.28 & 29.29 & 31.77 & 34.35 & 13779.62 & 1.00 \\ 
  $\mu_{\beta_{force}}$ & -7.46 & 0.89 & -9.19 & -7.46 & -5.71 & 2960.28 & 1.00 \\ 
  $\mu_{\beta_{chill}}$ & -8.75 & 0.81 & -10.29 & -8.76 & -7.11 & 6051.59 & 1.00 \\ 
  $\mu_{\beta_{photo}}$ & -1.21 & 0.46 & -2.10 & -1.20 & -0.29 & 2175.88 & 1.00 \\ 
  $\sigma_{\alpha}$ & 16.35 & 1.00 & 14.46 & 16.31 & 18.41 & 10178.43 & 1.00 \\ 
  $\sigma_{\beta_{force}}$ & 5.20 & 0.82 & 3.76 & 5.15 & 6.93 & 677.74 & 1.00 \\ 
  $\sigma_{\beta_{chill}}$ & 6.84 & 0.78 & 5.40 & 6.80 & 8.46 & 1815.10 & 1.00 \\ 
  $\sigma_{\beta_{photo}}$ & 2.27 & 0.35 & 1.61 & 2.25 & 2.99 & 649.15 & 1.00 \\ 
  $\sigma_y$ & 12.57 & 0.18 & 12.23 & 12.57 & 12.94 & 12887.31 & 1.00 \\ 
   \hline
\end{tabular}
\endgroup
\end{table} \clearpage \pagebreak 
% latex table generated in R 4.0.4 by xtable 1.8-4 package
% Sat Jun 15 02:27:55 2024
\begin{table}[ht]
\centering
\caption{Model parameters estimated for 191 tree species including mean, standard deviation (sd), 2.5\%, 50\%, and 97.5\% uncertainty intervals (z-scored model, thus predictors are directly comparable to one another), alongside model diagnostics. This model uses Chill Portions instead of chilling Utah units.} 
\label{tab:modelanglambchillports}
\begingroup\footnotesize
\begin{tabular}{lrrrrrrr}
  \hline
Parameter & mean & sd & 2.5\% & 50\% & 97.5\% & $n_{eff}$ & Rhat \\ 
  \hline
$\mu_{\alpha}$ & 30.07 & 3.46 & 23.26 & 30.09 & 37.05 & 13454.54 & 1.00 \\ 
  $\mu_{\beta_{force}}$ & -6.21 & 2.03 & -10.14 & -6.25 & -2.07 & 3322.83 & 1.00 \\ 
  $\mu_{\beta_{chill}}$ & -5.81 & 1.98 & -9.58 & -5.86 & -1.77 & 8346.15 & 1.00 \\ 
  $\mu_{\beta_{photo}}$ & -1.33 & 0.77 & -2.81 & -1.36 & 0.24 & 2598.50 & 1.00 \\ 
  $\lambda_{\alpha}$ & 0.35 & 0.10 & 0.17 & 0.35 & 0.56 & 7138.11 & 1.00 \\ 
  $\lambda_{\beta_{force}}$ & 0.66 & 0.21 & 0.20 & 0.68 & 0.97 & 428.34 & 1.00 \\ 
  $\lambda_{\beta_{chill}}$ & 0.52 & 0.13 & 0.26 & 0.52 & 0.77 & 2850.91 & 1.00 \\ 
  $\lambda_{\beta_{photo}}$ & 0.46 & 0.25 & 0.04 & 0.45 & 0.93 & 335.00 & 1.03 \\ 
  $\sigma_{\alpha}$ & 16.00 & 1.14 & 13.96 & 15.93 & 18.43 & 6926.49 & 1.00 \\ 
  $\sigma_{\beta_{force}}$ & 5.44 & 0.98 & 3.78 & 5.35 & 7.57 & 919.27 & 1.00 \\ 
  $\sigma_{\beta_{chill}}$ & 6.89 & 0.75 & 5.56 & 6.85 & 8.47 & 2579.79 & 1.00 \\ 
  $\sigma_{\beta_{photo}}$ & 2.36 & 0.40 & 1.64 & 2.33 & 3.21 & 754.25 & 1.01 \\ 
  $\sigma_y$ & 12.08 & 0.17 & 11.76 & 12.08 & 12.42 & 9826.08 & 1.00 \\ 
   \hline
\end{tabular}
\endgroup
\end{table} \clearpage \pagebreak 
% latex table generated in R 4.0.4 by xtable 1.8-4 package
% Sat Jun 15 02:27:55 2024
\begin{table}[ht]
\centering
\caption{Model parameters for non-phylogenetic model ($\lambda$ = 0) estimated for 191 tree species including mean, standard deviation (sd), 2.5\%, 50\%, and 97.5\% uncertainty intervals (z-scored model, thus predictors are directly comparable to one another), alongside model diagnostics. This model uses Chill Portions instead of chilling Utah units.} 
\label{tab:modlamb0chillports}
\begingroup\footnotesize
\begin{tabular}{lrrrrrrr}
  \hline
Parameter & mean & sd & 2.5\% & 50\% & 97.5\% & $n_{eff}$ & Rhat \\ 
  \hline
$\mu_{\alpha}$ & 31.32 & 1.25 & 28.82 & 31.32 & 33.77 & 13796.71 & 1.00 \\ 
  $\mu_{\beta_{force}}$ & -7.39 & 0.83 & -8.99 & -7.38 & -5.75 & 3253.91 & 1.00 \\ 
  $\mu_{\beta_{chill}}$ & -7.35 & 0.69 & -8.71 & -7.36 & -5.99 & 7607.11 & 1.00 \\ 
  $\mu_{\beta_{photo}}$ & -1.36 & 0.45 & -2.23 & -1.36 & -0.47 & 2528.36 & 1.00 \\ 
  $\sigma_{\alpha}$ & 16.50 & 1.00 & 14.66 & 16.45 & 18.58 & 11407.70 & 1.00 \\ 
  $\sigma_{\beta_{force}}$ & 4.91 & 0.76 & 3.57 & 4.85 & 6.56 & 706.89 & 1.00 \\ 
  $\sigma_{\beta_{chill}}$ & 6.51 & 0.65 & 5.33 & 6.49 & 7.87 & 2590.38 & 1.00 \\ 
  $\sigma_{\beta_{photo}}$ & 2.20 & 0.34 & 1.56 & 2.20 & 2.90 & 540.06 & 1.01 \\ 
  $\sigma_y$ & 12.08 & 0.17 & 11.76 & 12.08 & 12.42 & 14353.28 & 1.00 \\ 
   \hline
\end{tabular}
\endgroup
\end{table} \clearpage \pagebreak 




% latex table generated in R 4.0.4 by xtable 1.8-4 package
% Sat Jun 15 02:27:55 2024
\begingroup\footnotesize
\begin{longtable}{lrrrrrrrrr}
\caption{Estimated sensitivities of 191 tree species to three environmental cues: chilling ($\beta_{chill,j}$), forcing ($\beta_{force,j}$) and photoperiod ($\beta_{photo,j}$), along with their corresponding 2.5\% (low) and 97.5\% (up) uncertainty intervals (UI). Values correspond to the full model accounting for phylogenetic relationships.} \\ 
  \hline
\emph{Species} & \emph{$\beta_{chill,j}$} & \emph{lowUI} & \emph{upUI} & \emph{$\beta_{force,j}$} & \emph{lowUI} & \emph{upUI} & \emph{$\beta_{photo,j}$} & \emph{lowUI} & \emph{upUI} \\ 
  \hline
\endhead
\hline
\multicolumn{10}{l}{\footnotesize Continued on next page}
\endfoot
\endlastfoot
 \hline
\emph{Populus deltoides} & -15.16 & -25.05 & -5.82 & -6.43 & -13.73 & 1.82 & -0.97 & -5.39 & 3.23 \\ 
  \emph{Populus koreana} & -8.65 & -18.98 & 1.84 & -7.47 & -15.61 & 1.11 & -0.87 & -5.03 & 3.22 \\ 
  \emph{Populus tremula} & -1.94 & -7.35 & 3.60 & -7.41 & -10.54 & -4.23 & 0.05 & -2.11 & 2.28 \\ 
  \emph{Populus grandidentata} & -14.08 & -21.56 & -6.90 & -9.98 & -15.90 & -4.64 & -1.18 & -5.12 & 2.65 \\ 
  \emph{Euonymus europaeus} & -3.61 & -15.78 & 8.88 & -6.23 & -17.68 & 5.35 & -1.15 & -5.80 & 3.39 \\ 
  \emph{Euonymus latifolius} & 1.12 & -9.12 & 11.97 & -6.32 & -17.69 & 5.22 & -1.01 & -5.32 & 3.41 \\ 
  \emph{Nothofagus antarctica} & -10.16 & -22.40 & 1.95 & -6.62 & -17.37 & 4.52 & -1.63 & -6.10 & 2.79 \\ 
  \emph{Carya cordiformis} & -15.52 & -25.34 & -6.01 & -4.51 & -14.91 & 6.16 & -2.65 & -7.00 & 1.57 \\ 
  \emph{Carya laciniosa} & -17.48 & -27.47 & -7.95 & -4.03 & -14.28 & 6.79 & -1.72 & -6.04 & 2.60 \\ 
  \emph{Alnus incana} & -1.73 & -5.59 & 2.12 & -10.18 & -14.56 & -5.96 & -0.52 & -2.82 & 1.82 \\ 
  \emph{Alnus glutinosa} & -5.85 & -12.14 & 0.49 & -10.25 & -15.37 & -5.24 & -0.60 & -3.06 & 1.93 \\ 
  \emph{Alnus maximowiczii} & -5.35 & -14.90 & 4.33 & -7.32 & -16.89 & 2.55 & -1.05 & -5.26 & 3.15 \\ 
  \emph{Betula nana} & -5.76 & -14.88 & 3.60 & -5.15 & -13.97 & 3.61 & -0.92 & -4.90 & 3.08 \\ 
  \emph{Betula pendula} & -4.33 & -5.44 & -3.23 & -1.38 & -3.58 & 0.76 & -0.42 & -1.41 & 0.59 \\ 
  \emph{Betula pubescens} & -2.13 & -3.47 & -0.83 & -6.64 & -9.29 & -4.11 & -0.75 & -1.85 & 0.33 \\ 
  \emph{Betula populifolia} & -5.30 & -14.54 & 4.00 & -5.48 & -14.46 & 3.60 & -0.86 & -4.88 & 3.16 \\ 
  \emph{Betula papyrifera} & -27.28 & -33.48 & -21.06 & -5.69 & -10.02 & -1.32 & 1.25 & -2.24 & 5.13 \\ 
  \emph{Betula alleghaniensis} & -11.23 & -18.21 & -4.10 & -6.99 & -10.69 & -3.42 & -1.47 & -5.48 & 2.47 \\ 
  \emph{Betula lenta} & -3.33 & -10.33 & 3.85 & -4.84 & -12.37 & 2.89 & -0.98 & -5.16 & 2.98 \\ 
  \emph{Corylus cornuta} & -6.20 & -17.60 & 5.56 & -7.24 & -15.02 & 0.13 & -1.35 & -5.69 & 2.86 \\ 
  \emph{Ostrya carpinifolia} & -5.52 & -14.52 & 3.75 & -4.92 & -14.08 & 3.93 & -0.92 & -4.91 & 3.01 \\ 
  \emph{Ostrya virginiana} & -11.38 & -20.74 & -2.10 & -4.75 & -14.01 & 4.43 & -0.75 & -4.80 & 3.43 \\ 
  \emph{Carpinus laxiflora} & -8.99 & -19.72 & 1.53 & -4.74 & -13.68 & 3.95 & -0.91 & -5.22 & 3.32 \\ 
  \emph{Carpinus betulus} & -12.30 & -18.99 & -5.82 & -2.59 & -7.41 & 2.38 & -0.59 & -4.06 & 3.02 \\ 
  \emph{Carpinus monbeigiana} & -5.66 & -14.75 & 3.82 & -4.73 & -13.83 & 4.11 & -0.89 & -4.97 & 3.20 \\ 
  \emph{Rosa majalis} & -4.26 & -14.32 & 5.85 & -7.15 & -17.85 & 3.38 & -0.94 & -5.20 & 3.37 \\ 
  \emph{Rosa hugonis} & -6.88 & -18.69 & 5.08 & -7.23 & -17.86 & 3.75 & -1.03 & -5.49 & 3.50 \\ 
  \emph{Aronia melanocarpa} & -2.38 & -10.96 & 6.08 & -5.07 & -12.57 & 2.62 & -0.74 & -4.72 & 3.33 \\ 
  \emph{Photinia villosa} & -2.92 & -11.68 & 5.92 & -5.86 & -14.58 & 2.79 & -0.76 & -4.64 & 3.30 \\ 
  \emph{Spiraea japonica} & -3.74 & -12.85 & 5.42 & -6.66 & -16.49 & 3.07 & -0.71 & -4.74 & 3.41 \\ 
  \emph{Spiraea chamaedryfolia} & -4.02 & -13.48 & 5.21 & -6.72 & -16.37 & 3.13 & -0.73 & -4.89 & 3.58 \\ 
  \emph{Spiraea canescens} & -5.16 & -16.13 & 5.65 & -6.78 & -16.48 & 2.85 & -0.79 & -5.14 & 3.64 \\ 
  \emph{Prunus tenella} & -3.89 & -15.10 & 7.59 & -7.19 & -16.37 & 2.29 & -0.93 & -5.27 & 3.29 \\ 
  \emph{Prunus serrulata} & -4.63 & -15.76 & 6.60 & -7.08 & -15.83 & 2.23 & -0.78 & -4.88 & 3.43 \\ 
  \emph{Prunus pensylvanica} & -3.85 & -14.58 & 7.26 & -7.01 & -14.53 & 0.66 & -0.91 & -5.02 & 3.29 \\ 
  \emph{Prunus serotina} & -7.40 & -16.97 & 2.22 & -6.26 & -14.41 & 2.36 & -0.83 & -4.90 & 3.31 \\ 
  \emph{Prunus padus} & -1.76 & -7.09 & 3.46 & -10.44 & -14.27 & -6.74 & -0.96 & -2.96 & 1.03 \\ 
  \emph{Prinsepia uniflora} & -5.04 & -16.35 & 6.67 & -6.97 & -16.85 & 3.27 & -0.72 & -4.89 & 3.49 \\ 
  \emph{Prinsepia sinensis} & -5.07 & -17.10 & 6.91 & -7.03 & -16.88 & 3.01 & -0.80 & -5.04 & 3.55 \\ 
  \emph{Oemleria cerasiformis} & -4.93 & -15.87 & 6.42 & -6.84 & -16.51 & 3.03 & -0.75 & -4.99 & 3.60 \\ 
  \emph{Ulmus minor} & -16.20 & -20.78 & -11.59 & -10.39 & -14.29 & -6.49 & -2.57 & -6.32 & 0.98 \\ 
  \emph{Ulmus glabra} & -18.61 & -25.10 & -12.12 & -10.25 & -17.35 & -2.78 & -1.49 & -5.16 & 2.38 \\ 
  \emph{Ulmus macrocarpa} & -17.08 & -23.91 & -10.39 & -10.51 & -18.27 & -2.82 & -1.49 & -5.27 & 2.54 \\ 
  \emph{Ulmus pumila} & -12.48 & -17.10 & -7.94 & -10.68 & -14.62 & -6.65 & -1.81 & -5.35 & 1.76 \\ 
  \emph{Ulmus parvifolia} & -15.59 & -20.21 & -10.92 & -9.42 & -13.49 & -5.28 & -2.39 & -6.08 & 1.18 \\ 
  \emph{Ulmus laevis} & -16.26 & -25.01 & -7.62 & -9.63 & -17.46 & -1.30 & -1.64 & -5.64 & 2.43 \\ 
  \emph{Ulmus americana} & -13.53 & -22.09 & -4.68 & -9.81 & -17.57 & -1.43 & -1.59 & -5.51 & 2.44 \\ 
  \emph{Ulmus villosa} & -15.74 & -20.40 & -11.06 & -11.92 & -15.90 & -8.04 & -1.94 & -5.59 & 1.65 \\ 
  \emph{Caragana pygmaea} & -5.09 & -16.95 & 6.79 & -4.36 & -14.86 & 6.22 & -1.14 & -5.42 & 3.32 \\ 
  \emph{Robinia pseudoacacia} & -5.49 & -15.64 & 4.60 & -0.16 & -5.50 & 5.18 & -0.98 & -5.30 & 3.51 \\ 
  \emph{Amorpha fruticosa} & -1.98 & -12.24 & 8.75 & -3.88 & -13.42 & 5.70 & -1.04 & -5.35 & 3.42 \\ 
  \emph{Cercis chinensis} & -0.91 & -10.52 & 9.36 & -5.45 & -16.53 & 5.76 & -1.86 & -6.37 & 2.54 \\ 
  \emph{Cercis canadensis} & -4.02 & -13.75 & 5.85 & -5.15 & -15.95 & 6.47 & -1.45 & -6.00 & 3.10 \\ 
  \emph{Tilia japonica} & -2.93 & -11.73 & 6.09 & -8.03 & -16.42 & 0.99 & -1.26 & -5.59 & 3.04 \\ 
  \emph{Tilia cordata} & -3.70 & -8.16 & 0.93 & -9.91 & -14.05 & -5.82 & -2.55 & -6.15 & 0.92 \\ 
  \emph{Tilia dasystyla} & -7.53 & -17.00 & 1.52 & -7.95 & -16.41 & 1.18 & -1.28 & -5.47 & 2.95 \\ 
  \emph{Tilia platyphyllos} & -4.30 & -13.23 & 4.64 & -7.95 & -16.40 & 1.33 & -1.27 & -5.57 & 3.13 \\ 
  \emph{Hibiscus syriacus} & 0.01 & -9.30 & 10.00 & -7.50 & -17.14 & 2.41 & -1.26 & -5.54 & 3.08 \\ 
  \emph{Aesculus flava} & -14.90 & -24.42 & -5.70 & -1.49 & -10.89 & 7.74 & -1.45 & -5.58 & 2.77 \\ 
  \emph{Aesculus parviflora} & -12.63 & -21.95 & -3.69 & -2.00 & -11.47 & 7.49 & -2.00 & -6.23 & 2.02 \\ 
  \emph{Aesculus hippocastanum} & -8.33 & -16.34 & -0.34 & 1.57 & -4.06 & 7.72 & -1.33 & -5.34 & 2.74 \\ 
  \emph{Toona sinensis} & -3.77 & -13.34 & 6.19 & -4.78 & -15.53 & 6.37 & -1.31 & -5.70 & 2.94 \\ 
  \emph{Orixa japonica} & -5.76 & -15.83 & 4.15 & -4.75 & -15.49 & 6.38 & -1.21 & -5.45 & 3.25 \\ 
  \emph{Ptelea trifoliata} & -4.44 & -14.20 & 5.45 & -4.60 & -15.22 & 6.67 & -1.23 & -5.57 & 3.14 \\ 
  \emph{Ribes divaricatum} & -11.44 & -23.49 & 0.20 & -6.57 & -18.06 & 4.68 & -1.25 & -5.82 & 3.30 \\ 
  \emph{Ribes glaciale} & -7.39 & -17.37 & 2.48 & -6.69 & -18.49 & 4.88 & -0.96 & -5.39 & 3.52 \\ 
  \emph{Ribes alpinum} & -9.60 & -21.69 & 2.07 & -6.65 & -18.27 & 4.94 & -1.07 & -5.56 & 3.50 \\ 
  \emph{Hamamelis virginiana} & -8.19 & -20.19 & 3.97 & -9.99 & -19.60 & -0.73 & -1.60 & -6.28 & 3.08 \\ 
  \emph{Hamamelis vernalis} & -12.75 & -22.86 & -2.86 & -7.46 & -18.49 & 3.77 & -1.08 & -5.47 & 3.33 \\ 
  \emph{Hamamelis japonica} & -8.81 & -18.66 & 1.17 & -7.47 & -18.61 & 3.42 & -1.11 & -5.53 & 3.31 \\ 
  \emph{Sinowilsonia henryi} & -4.90 & -15.08 & 5.12 & -7.03 & -18.68 & 4.55 & -1.22 & -5.58 & 3.34 \\ 
  \emph{Corylopsis sinensis} & -7.56 & -19.20 & 4.40 & -7.10 & -18.43 & 3.99 & -1.33 & -5.75 & 3.21 \\ 
  \emph{Corylopsis spicata} & -7.91 & -20.00 & 4.03 & -7.06 & -18.29 & 4.06 & -1.42 & -5.81 & 3.03 \\ 
  \emph{Liquidambar styraciflua} & -7.91 & -17.96 & 1.79 & -6.68 & -17.99 & 4.77 & -1.51 & -5.92 & 2.83 \\ 
  \emph{Liquidambar orientalis} & -2.27 & -12.08 & 7.89 & -6.81 & -18.31 & 4.47 & -1.21 & -5.61 & 3.23 \\ 
  \emph{Cercidiphyllum japonicum} & -9.19 & -20.49 & 2.23 & -6.65 & -17.89 & 4.96 & -1.24 & -5.71 & 3.26 \\ 
  \emph{Cercidiphyllum magnificum} & -9.91 & -19.98 & -0.03 & -6.58 & -17.94 & 4.98 & -1.30 & -5.65 & 3.18 \\ 
  \emph{Parrotia persica} & -8.06 & -20.72 & 4.46 & -6.62 & -17.88 & 4.72 & -1.01 & -5.32 & 3.49 \\ 
  \emph{Paeonia rockii} & -6.41 & -18.74 & 6.05 & -6.63 & -18.19 & 5.11 & -1.17 & -5.66 & 3.44 \\ 
  \emph{Syringa villosa} & -5.04 & -15.49 & 5.54 & -4.17 & -12.82 & 4.14 & -0.89 & -5.19 & 3.46 \\ 
  \emph{Syringa vulgaris} & -4.55 & -13.61 & 4.73 & -1.79 & -6.08 & 2.68 & -0.93 & -5.22 & 3.18 \\ 
  \emph{Syringa josikaea} & -4.34 & -14.73 & 6.34 & -4.07 & -12.75 & 4.47 & -0.96 & -5.36 & 3.41 \\ 
  \emph{Syringa reticulata} & -4.51 & -15.06 & 5.98 & -4.12 & -12.75 & 4.24 & -0.88 & -5.17 & 3.50 \\ 
  \emph{Ligustrum tschonoskii} & -4.45 & -14.78 & 5.75 & -4.13 & -12.79 & 4.31 & -0.88 & -5.20 & 3.49 \\ 
  \emph{Forsythia ovata} & -4.34 & -14.99 & 6.24 & -4.68 & -14.36 & 4.83 & -0.88 & -5.23 & 3.38 \\ 
  \emph{Forsythia suspensa} & -4.40 & -13.57 & 5.01 & -4.73 & -14.22 & 4.57 & -0.83 & -5.14 & 3.55 \\ 
  \emph{Cephalanthus occidentalis} & -2.38 & -12.22 & 7.91 & -5.37 & -16.10 & 5.42 & -1.11 & -5.46 & 3.35 \\ 
  \emph{Viburnum buddleifolium} & -10.26 & -21.58 & 0.26 & -4.75 & -13.53 & 4.42 & -0.23 & -4.45 & 4.30 \\ 
  \emph{Viburnum carlesii} & -7.53 & -18.46 & 2.95 & -5.33 & -14.31 & 3.30 & -0.86 & -5.13 & 3.60 \\ 
  \emph{Viburnum cassinoides} & -4.06 & -11.31 & 3.10 & -4.63 & -10.55 & 1.19 & -1.07 & -4.93 & 2.89 \\ 
  \emph{Viburnum lantanoides} & -8.43 & -15.76 & -1.23 & -7.04 & -12.94 & -1.20 & -1.27 & -5.27 & 2.77 \\ 
  \emph{Viburnum plicatum} & -9.13 & -20.55 & 1.66 & -5.00 & -13.89 & 4.25 & -0.49 & -4.90 & 4.04 \\ 
  \emph{Viburnum opulus} & -6.20 & -15.54 & 3.42 & -5.27 & -14.19 & 3.75 & -0.80 & -5.10 & 3.44 \\ 
  \emph{Viburnum betulifolium} & -7.78 & -18.78 & 3.06 & -5.35 & -14.50 & 3.79 & -0.99 & -5.29 & 3.27 \\ 
  \emph{Weigela coraeensis} & -4.55 & -14.12 & 4.95 & -4.97 & -15.03 & 5.46 & -0.70 & -4.94 & 3.65 \\ 
  \emph{Weigela florida} & -5.05 & -16.05 & 6.17 & -5.02 & -15.07 & 5.15 & -0.72 & -5.01 & 3.64 \\ 
  \emph{Weigela maximowiczii} & -5.99 & -15.36 & 3.50 & -4.92 & -15.24 & 5.49 & -0.67 & -4.88 & 3.66 \\ 
  \emph{Heptacodium miconioides} & -4.27 & -15.74 & 7.38 & -5.05 & -14.97 & 4.95 & -0.78 & -5.20 & 3.75 \\ 
  \emph{Symphoricarpos albus} & -6.84 & -16.49 & 2.74 & -3.29 & -8.35 & 1.96 & -0.74 & -5.02 & 3.60 \\ 
  \emph{Lonicera maximowiczii} & -4.60 & -16.20 & 6.94 & -5.01 & -14.15 & 4.06 & -0.71 & -4.96 & 3.80 \\ 
  \emph{Lonicera alpigena} & -4.83 & -16.11 & 6.80 & -4.91 & -14.25 & 4.31 & -0.61 & -4.86 & 3.79 \\ 
  \emph{Lonicera canadensis} & -4.56 & -16.34 & 7.44 & -5.48 & -13.30 & 1.79 & -0.87 & -5.28 & 3.53 \\ 
  \emph{Lonicera caerulea} & -4.96 & -17.41 & 7.31 & -5.19 & -14.44 & 3.80 & -0.77 & -5.21 & 3.86 \\ 
  \emph{Eleutherococcus sieboldianus} & -3.08 & -12.87 & 6.67 & -5.51 & -16.16 & 5.23 & -0.75 & -5.07 & 3.58 \\ 
  \emph{Eleutherococcus setchuenensis} & -3.74 & -15.08 & 7.72 & -5.47 & -16.29 & 5.56 & -0.86 & -5.32 & 3.72 \\ 
  \emph{Eleutherococcus senticosus} & -3.20 & -12.73 & 6.71 & -5.45 & -16.37 & 5.49 & -0.72 & -4.90 & 3.73 \\ 
  \emph{Ilex mucronata} & -2.88 & -10.53 & 4.87 & -6.00 & -12.28 & 0.16 & -1.91 & -6.17 & 2.25 \\ 
  \emph{Rhododendron prinophyllum} & -3.02 & -14.99 & 9.25 & -11.17 & -20.26 & -2.67 & -0.97 & -5.49 & 3.70 \\ 
  \emph{Rhododendron canadense} & 0.00 & -9.69 & 9.99 & -10.20 & -19.93 & -0.41 & -0.78 & -4.94 & 3.73 \\ 
  \emph{Rhododendron dauricum} & -1.90 & -13.20 & 9.95 & -10.15 & -20.05 & 0.24 & -0.66 & -5.05 & 3.89 \\ 
  \emph{Rhododendron mucronulatum} & -2.54 & -14.20 & 9.05 & -10.38 & -20.08 & -0.54 & -0.78 & -5.24 & 3.88 \\ 
  \emph{Kalmia angustifolia} & -2.93 & -14.96 & 9.56 & -12.80 & -21.36 & -4.83 & -1.53 & -6.00 & 2.93 \\ 
  \emph{Vaccinium myrtilloides} & -1.69 & -13.89 & 10.77 & -9.51 & -17.61 & -1.70 & -0.80 & -5.17 & 3.86 \\ 
  \emph{Lyonia ligustrina} & -3.56 & -15.92 & 9.13 & -11.82 & -20.72 & -3.04 & -0.99 & -5.57 & 3.76 \\ 
  \emph{Nyssa sylvatica} & -6.22 & -19.01 & 6.72 & -8.05 & -17.75 & 1.31 & -1.44 & -6.02 & 3.16 \\ 
  \emph{Cornus alba} & -4.43 & -14.38 & 5.63 & 2.72 & -4.19 & 9.86 & -1.48 & -5.75 & 2.74 \\ 
  \emph{Cornus kousa} & -2.92 & -12.91 & 6.78 & -2.84 & -12.93 & 7.21 & -1.30 & -5.77 & 3.02 \\ 
  \emph{Cornus mas} & -4.83 & -14.88 & 5.25 & -2.17 & -8.45 & 4.40 & -1.15 & -5.51 & 3.30 \\ 
  \emph{Hydrangea arborescens} & -8.72 & -18.61 & 0.95 & -5.11 & -16.13 & 6.25 & -0.91 & -5.30 & 3.50 \\ 
  \emph{Hydrangea involucrata} & -8.38 & -17.92 & 1.10 & -5.18 & -16.24 & 6.18 & -0.96 & -5.14 & 3.46 \\ 
  \emph{Hydrangea serrata} & -7.90 & -19.38 & 3.71 & -5.24 & -16.55 & 6.05 & -0.95 & -5.30 & 3.43 \\ 
  \emph{Deutzia gracilis} & -5.33 & -16.70 & 6.09 & -5.43 & -16.20 & 5.60 & -0.87 & -5.24 & 3.65 \\ 
  \emph{Deutzia scabra} & -3.84 & -13.80 & 6.13 & -5.30 & -16.00 & 5.63 & -0.87 & -5.16 & 3.50 \\ 
  \emph{Decaisnea fargesii} & -8.07 & -21.12 & 4.95 & -6.02 & -18.26 & 5.96 & -1.05 & -5.68 & 3.54 \\ 
  \emph{Berberis dielsiana} & -5.03 & -18.26 & 8.40 & -6.23 & -18.11 & 6.06 & -0.98 & -5.72 & 3.80 \\ 
  \emph{Liriodendron tulipifera} & -12.23 & -26.09 & 1.05 & -6.28 & -18.58 & 5.93 & -1.57 & -6.32 & 3.20 \\ 
  \emph{Acer pseudoplatanus} & -10.59 & -16.87 & -4.06 & -9.06 & -12.19 & -5.96 & -1.30 & -4.58 & 2.05 \\ 
  \emph{Acer saccharinum} & -7.95 & -12.05 & -3.92 & -3.33 & -10.91 & 4.47 & -1.20 & -5.57 & 3.07 \\ 
  \emph{Acer rubrum} & -15.89 & -22.63 & -9.17 & -0.34 & -5.01 & 4.46 & -0.42 & -4.11 & 3.53 \\ 
  \emph{Acer barbinerve} & -9.88 & -20.06 & 0.71 & -4.03 & -12.65 & 4.64 & -1.19 & -5.29 & 2.92 \\ 
  \emph{Acer negundo} & -11.88 & -21.14 & -2.52 & -2.89 & -10.66 & 5.24 & -1.30 & -5.54 & 2.85 \\ 
  \emph{Acer pensylvanicum} & -9.42 & -16.45 & -2.35 & -5.80 & -11.67 & -0.04 & -1.97 & -5.94 & 1.83 \\ 
  \emph{Acer platanoides} & -10.35 & -19.35 & -1.41 & -3.77 & -12.18 & 4.69 & -1.21 & -5.25 & 2.80 \\ 
  \emph{Acer campestre} & -11.25 & -20.38 & -2.26 & -3.71 & -12.17 & 5.09 & -1.10 & -5.10 & 3.04 \\ 
  \emph{Acer tataricum} & -9.05 & -18.10 & 0.14 & -2.60 & -10.40 & 5.71 & -1.18 & -5.31 & 3.01 \\ 
  \emph{Acer ginnala} & -9.18 & -19.83 & 1.23 & -3.93 & -12.38 & 4.92 & -1.22 & -5.43 & 3.05 \\ 
  \emph{Amelanchier laevis} & -5.13 & -15.80 & 6.01 & -5.88 & -14.63 & 2.85 & -0.91 & -5.08 & 3.31 \\ 
  \emph{Amelanchier florida} & -6.02 & -15.05 & 2.89 & -5.86 & -14.55 & 3.07 & -0.72 & -4.73 & 3.36 \\ 
  \emph{Buddleja davidii} & -3.03 & -15.36 & 9.46 & -5.45 & -15.85 & 5.04 & -0.75 & -5.13 & 3.84 \\ 
  \emph{Buddleja alternifolia} & -2.79 & -14.91 & 9.62 & -5.45 & -15.69 & 5.26 & -0.83 & -5.19 & 3.75 \\ 
  \emph{Buddleja albiflora} & -3.35 & -15.38 & 8.66 & -5.48 & -16.42 & 4.90 & -0.79 & -5.21 & 3.92 \\ 
  \emph{Celtis laevigata} & -12.47 & -22.42 & -2.60 & -7.60 & -18.51 & 3.47 & -1.60 & -5.81 & 2.64 \\ 
  \emph{Celtis occidentalis} & -7.70 & -19.65 & 3.89 & -7.48 & -18.18 & 3.43 & -1.40 & -5.77 & 3.03 \\ 
  \emph{Celtis caucasica} & -10.86 & -20.81 & -1.13 & -7.54 & -18.07 & 2.98 & -1.29 & -5.60 & 3.02 \\ 
  \emph{Cladrastis lutea} & -9.69 & -19.88 & 0.48 & -4.68 & -15.83 & 6.43 & -1.93 & -6.40 & 2.32 \\ 
  \emph{Elaeagnus ebbingei} & -7.93 & -20.04 & 4.44 & -8.29 & -19.52 & 2.52 & -1.09 & -5.42 & 3.40 \\ 
  \emph{Fagus crenata} & -13.85 & -22.96 & -4.61 & -7.94 & -16.59 & 0.62 & -3.49 & -8.02 & 1.21 \\ 
  \emph{Fagus engleriana} & -13.94 & -22.98 & -5.04 & -7.93 & -16.88 & 0.24 & -3.39 & -7.90 & 1.36 \\ 
  \emph{Fagus grandifolia} & -13.77 & -21.02 & -6.50 & -8.83 & -14.77 & -2.89 & -4.52 & -8.88 & -0.31 \\ 
  \emph{Fagus orientalis} & -16.38 & -25.68 & -7.38 & -8.06 & -16.89 & 0.26 & -4.00 & -8.52 & 0.42 \\ 
  \emph{Fagus sylvatica} & -14.31 & -15.89 & -12.74 & -2.91 & -4.92 & -0.92 & -9.37 & -11.91 & -6.71 \\ 
  \emph{Fraxinus excelsior} & -6.70 & -15.49 & 1.96 & -5.66 & -13.56 & 1.70 & -1.34 & -5.59 & 2.76 \\ 
  \emph{Fraxinus ornus} & -11.44 & -20.97 & -2.69 & -4.07 & -12.57 & 4.71 & -0.88 & -4.99 & 3.28 \\ 
  \emph{Fraxinus nigra} & -6.30 & -16.53 & 4.27 & -7.89 & -15.98 & -0.64 & -1.49 & -5.81 & 2.64 \\ 
  \emph{Fraxinus pennsylvanica} & -4.76 & -13.81 & 4.38 & -4.04 & -11.90 & 4.08 & -1.06 & -5.25 & 3.18 \\ 
  \emph{Fraxinus americana} & -7.23 & -11.63 & -2.76 & -3.87 & -11.48 & 4.02 & -0.98 & -5.27 & 3.39 \\ 
  \emph{Fraxinus latifolia} & -7.66 & -16.43 & 1.07 & -4.63 & -13.19 & 3.81 & -1.45 & -5.61 & 2.74 \\ 
  \emph{Fraxinus chinensis} & -5.94 & -14.92 & 2.99 & -4.40 & -12.35 & 3.60 & -0.95 & -5.04 & 3.16 \\ 
  \emph{Juglans cinerea} & -11.11 & -20.91 & -1.25 & -2.91 & -12.33 & 7.19 & -1.45 & -5.75 & 2.70 \\ 
  \emph{Juglans ailantifolia} & -11.61 & -21.18 & -2.29 & -2.85 & -12.14 & 7.07 & -1.51 & -5.90 & 2.77 \\ 
  \emph{Parrotiopsis jaquemontiana} & -5.83 & -15.63 & 3.89 & -7.40 & -18.44 & 3.63 & -1.11 & -5.55 & 3.32 \\ 
  \emph{Pyrus ussuriensis} & -6.05 & -16.71 & 4.34 & -5.98 & -14.58 & 2.52 & -0.73 & -4.74 & 3.31 \\ 
  \emph{Pyrus elaeagnifolia} & -6.92 & -15.89 & 1.86 & -5.85 & -14.48 & 3.13 & -0.80 & -4.83 & 3.31 \\ 
  \emph{Quercus faginea} & -20.99 & -32.47 & -10.85 & -12.86 & -18.58 & -7.15 & -3.69 & -8.03 & 0.37 \\ 
  \emph{Quercus bicolor} & -15.08 & -23.84 & -6.01 & -9.69 & -16.58 & -1.46 & -2.31 & -6.48 & 1.89 \\ 
  \emph{Quercus alba} & -14.33 & -24.51 & -3.88 & -12.84 & -19.68 & -5.59 & -2.69 & -6.74 & 1.39 \\ 
  \emph{Quercus coccifera} & -21.14 & -32.27 & -10.79 & -12.91 & -20.09 & -5.50 & -2.50 & -6.73 & 1.85 \\ 
  \emph{Quercus rubra} & -18.25 & -24.21 & -12.29 & -11.54 & -15.11 & -7.89 & -2.56 & -6.10 & 1.07 \\ 
  \emph{Quercus ellipsoidalis} & -17.47 & -27.04 & -7.96 & -13.15 & -20.62 & -5.69 & -2.63 & -6.89 & 1.63 \\ 
  \emph{Quercus velutina} & -14.69 & -25.24 & -3.81 & -13.13 & -20.18 & -6.08 & -2.47 & -6.65 & 1.80 \\ 
  \emph{Quercus shumardii} & -17.04 & -25.91 & -8.41 & -12.12 & -19.62 & -4.18 & -2.40 & -6.36 & 1.79 \\ 
  \emph{Quercus ilex} & -23.29 & -34.81 & -13.14 & -17.73 & -24.27 & -11.90 & -2.57 & -6.69 & 1.78 \\ 
  \emph{Quercus petraea} & -16.47 & -22.58 & -10.32 & -12.45 & -15.88 & -9.04 & -2.64 & -6.62 & 1.38 \\ 
  \emph{Quercus pubescens} & -14.74 & -25.34 & -3.62 & -16.42 & -24.65 & -9.56 & -2.33 & -6.61 & 2.28 \\ 
  \emph{Quercus robur} & -13.18 & -19.40 & -6.83 & -11.66 & -15.22 & -8.02 & 0.28 & -3.49 & 4.35 \\ 
  \emph{Rhamnus cathartica} & -4.58 & -12.27 & 3.22 & -11.90 & -22.38 & -2.51 & -1.93 & -6.12 & 2.18 \\ 
  \emph{Rhamnus alpina} & -11.53 & -21.67 & -2.00 & -8.90 & -19.55 & 1.61 & -1.14 & -5.42 & 3.13 \\ 
  \emph{Rhamnus frangula} & -6.98 & -18.76 & 4.88 & -10.67 & -19.83 & -1.62 & -1.29 & -5.82 & 3.19 \\ 
  \emph{Salix gracilistyla} & -7.38 & -18.54 & 3.62 & -6.15 & -15.42 & 3.21 & -1.03 & -5.34 & 3.26 \\ 
  \emph{Salix smithiana} & -5.00 & -11.20 & 1.20 & -4.71 & -10.10 & 0.96 & -1.98 & -6.28 & 2.08 \\ 
  \emph{Salix repens} & -9.73 & -19.17 & -0.53 & -6.02 & -14.86 & 3.35 & -1.16 & -5.36 & 3.15 \\ 
  \emph{Sambucus tigranii} & -8.12 & -20.14 & 3.49 & -5.33 & -15.53 & 5.07 & -0.83 & -5.23 & 3.56 \\ 
  \emph{Sambucus pubens} & -8.30 & -20.17 & 3.35 & -5.33 & -15.75 & 5.11 & -0.79 & -5.26 & 3.70 \\ 
  \emph{Sorbus aucuparia} & -5.67 & -15.28 & 4.26 & -5.60 & -10.21 & -0.98 & -0.80 & -4.24 & 2.58 \\ 
  \emph{Sorbus torminalis} & -8.15 & -17.44 & 0.37 & -6.21 & -14.71 & 2.47 & -0.83 & -4.46 & 2.95 \\ 
  \emph{Sorbus aria} & -8.86 & -18.00 & 0.08 & -5.67 & -14.08 & 2.86 & -0.86 & -4.74 & 3.09 \\ 
  \emph{Sorbus decora} & -6.40 & -15.18 & 2.22 & -5.78 & -14.19 & 2.78 & -0.85 & -4.88 & 3.17 \\ 
  \emph{Sorbus commixta} & -5.84 & -14.68 & 3.21 & -5.88 & -14.37 & 2.63 & -0.77 & -4.82 & 3.39 \\ 
  \emph{Spirea alba} & -2.58 & -14.70 & 10.38 & -5.51 & -13.84 & 2.80 & -0.99 & -5.28 & 3.46 \\ 
  \emph{Stachyurus praecox} & -6.22 & -18.40 & 6.27 & -6.02 & -16.95 & 5.21 & -1.08 & -5.46 & 3.37 \\ 
  \emph{Stachyurus sinensis} & -6.14 & -16.90 & 4.62 & -5.65 & -17.01 & 5.99 & -0.97 & -5.47 & 3.62 \\ 
  \hline
\label{tab:tablesupp3}
\end{longtable}
\endgroup \clearpage \pagebreak 
% latex table generated in R 4.0.4 by xtable 1.8-4 package
% Sat Jun 15 02:27:55 2024
\begingroup\footnotesize
\begin{longtable}{lrrrrrrrrr}
\caption{Estimated sensitivities of 191 tree species to three environmental cues: chilling ($\beta_{chill,j}$), forcing ($\beta_{force,j}$) and photoperiod ($\beta_{photo,j}$), along with their corresponding 2.5\% (low) and 97.5\% (up) uncertainty intervals (UI). Values correspond to the non phylogenetic model.} \\ 
  \hline
\emph{Species} & \emph{$\beta_{chill,j}$} & \emph{lowUI} & \emph{upUI} & \emph{$\beta_{force,j}$} & \emph{lowUI} & \emph{upUI} & \emph{$\beta_{photo,j}$} & \emph{lowUI} & \emph{upUI} \\ 
  \hline
\endhead
\hline
\multicolumn{10}{l}{\footnotesize Continued on next page}
\endfoot
\endlastfoot
 \hline
\emph{Populus deltoides} & -18.11 & -28.51 & -7.77 & -6.23 & -15.95 & 4.03 & -1.26 & -5.84 & 3.32 \\ 
  \emph{Populus koreana} & -7.92 & -20.17 & 4.43 & -7.42 & -18.36 & 3.81 & -1.10 & -5.60 & 3.35 \\ 
  \emph{Populus tremula} & -0.58 & -6.21 & 5.09 & -7.44 & -10.70 & -4.20 & 0.08 & -2.10 & 2.32 \\ 
  \emph{Populus grandidentata} & -15.62 & -23.79 & -7.77 & -11.33 & -17.49 & -5.03 & -1.32 & -5.36 & 2.76 \\ 
  \emph{Euonymus europaeus} & -8.27 & -20.53 & 4.41 & -7.49 & -17.98 & 2.90 & -1.18 & -5.65 & 3.40 \\ 
  \emph{Euonymus latifolius} & 0.39 & -10.11 & 10.84 & -7.43 & -18.03 & 3.43 & -0.92 & -5.14 & 3.40 \\ 
  \emph{Nothofagus antarctica} & -10.50 & -23.34 & 2.22 & -7.46 & -17.74 & 3.05 & -1.38 & -5.87 & 3.14 \\ 
  \emph{Carya cordiformis} & -14.04 & -24.40 & -3.93 & -7.85 & -18.59 & 2.95 & -2.62 & -7.25 & 1.74 \\ 
  \emph{Carya laciniosa} & -17.27 & -27.99 & -6.99 & -6.61 & -17.22 & 4.39 & -1.16 & -5.49 & 3.25 \\ 
  \emph{Alnus incana} & -1.59 & -5.55 & 2.37 & -10.38 & -14.59 & -6.15 & -0.55 & -2.95 & 1.88 \\ 
  \emph{Alnus glutinosa} & -6.51 & -13.31 & 0.43 & -10.42 & -15.83 & -5.14 & -0.61 & -3.13 & 1.88 \\ 
  \emph{Alnus maximowiczii} & -6.18 & -16.56 & 4.39 & -7.15 & -17.62 & 3.58 & -1.21 & -5.73 & 3.36 \\ 
  \emph{Betula nana} & -6.39 & -16.31 & 3.94 & -7.15 & -17.95 & 3.49 & -1.13 & -5.50 & 3.18 \\ 
  \emph{Betula pendula} & -4.32 & -5.43 & -3.24 & -1.34 & -3.51 & 0.83 & -0.43 & -1.41 & 0.55 \\ 
  \emph{Betula pubescens} & -2.11 & -3.42 & -0.79 & -6.88 & -9.54 & -4.20 & -0.78 & -1.88 & 0.31 \\ 
  \emph{Betula populifolia} & -5.25 & -15.53 & 5.25 & -7.38 & -18.12 & 3.16 & -1.08 & -5.38 & 3.30 \\ 
  \emph{Betula papyrifera} & -29.62 & -35.81 & -23.48 & -5.80 & -10.59 & -0.96 & 1.37 & -2.37 & 5.36 \\ 
  \emph{Betula alleghaniensis} & -9.92 & -17.11 & -2.60 & -6.93 & -10.75 & -3.03 & -2.03 & -6.13 & 1.95 \\ 
  \emph{Betula lenta} & -2.29 & -9.47 & 5.25 & -5.99 & -14.69 & 2.79 & -1.23 & -5.54 & 3.13 \\ 
  \emph{Corylus cornuta} & -6.37 & -19.26 & 7.06 & -9.25 & -17.67 & -1.12 & -1.62 & -5.98 & 2.70 \\ 
  \emph{Ostrya carpinifolia} & -4.10 & -14.33 & 6.22 & -7.40 & -18.07 & 3.12 & -1.08 & -5.33 & 3.33 \\ 
  \emph{Ostrya virginiana} & -13.26 & -23.48 & -3.08 & -6.97 & -17.59 & 3.71 & -0.85 & -5.04 & 3.54 \\ 
  \emph{Carpinus laxiflora} & -9.40 & -22.06 & 3.28 & -7.33 & -18.04 & 3.19 & -1.13 & -5.57 & 3.27 \\ 
  \emph{Carpinus betulus} & -13.38 & -20.38 & -6.35 & -2.12 & -7.22 & 3.01 & -0.62 & -4.38 & 3.08 \\ 
  \emph{Carpinus monbeigiana} & -4.43 & -14.44 & 6.14 & -7.23 & -17.98 & 3.60 & -1.05 & -5.41 & 3.16 \\ 
  \emph{Rosa majalis} & -4.90 & -14.82 & 5.20 & -7.49 & -18.17 & 2.80 & -0.91 & -5.30 & 3.54 \\ 
  \emph{Rosa hugonis} & -8.24 & -20.58 & 4.29 & -7.48 & -17.98 & 2.93 & -1.01 & -5.44 & 3.50 \\ 
  \emph{Aronia melanocarpa} & -2.26 & -12.02 & 8.08 & -5.30 & -13.53 & 3.41 & -1.01 & -5.18 & 3.30 \\ 
  \emph{Photinia villosa} & -3.69 & -13.75 & 6.68 & -7.29 & -17.69 & 3.12 & -0.99 & -5.24 & 3.28 \\ 
  \emph{Spiraea japonica} & -5.62 & -15.79 & 4.68 & -7.23 & -17.59 & 3.41 & -0.90 & -5.24 & 3.44 \\ 
  \emph{Spiraea chamaedryfolia} & -6.08 & -16.17 & 4.00 & -7.35 & -17.96 & 3.50 & -0.90 & -5.25 & 3.53 \\ 
  \emph{Spiraea canescens} & -8.77 & -21.15 & 3.53 & -7.52 & -17.64 & 3.02 & -1.05 & -5.46 & 3.37 \\ 
  \emph{Prunus tenella} & -6.68 & -18.91 & 6.02 & -7.50 & -18.67 & 3.50 & -1.11 & -5.55 & 3.30 \\ 
  \emph{Prunus serrulata} & -8.11 & -20.34 & 4.47 & -7.26 & -17.51 & 2.93 & -0.94 & -5.47 & 3.52 \\ 
  \emph{Prunus pensylvanica} & -6.31 & -18.71 & 6.34 & -6.83 & -14.73 & 1.34 & -1.14 & -5.66 & 3.25 \\ 
  \emph{Prunus serotina} & -10.58 & -21.02 & -0.14 & -5.49 & -14.91 & 4.23 & -1.09 & -5.41 & 3.21 \\ 
  \emph{Prunus padus} & -2.17 & -7.46 & 3.09 & -10.81 & -14.58 & -6.99 & -1.02 & -3.02 & 0.93 \\ 
  \emph{Prinsepia uniflora} & -7.80 & -20.33 & 4.82 & -7.62 & -18.52 & 3.05 & -0.96 & -5.31 & 3.54 \\ 
  \emph{Prinsepia sinensis} & -8.08 & -21.71 & 5.47 & -7.73 & -18.30 & 2.92 & -1.03 & -5.45 & 3.47 \\ 
  \emph{Oemleria cerasiformis} & -7.71 & -19.80 & 4.48 & -7.38 & -17.76 & 2.99 & -0.89 & -5.29 & 3.44 \\ 
  \emph{Ulmus minor} & -15.63 & -20.50 & -10.78 & -9.99 & -14.32 & -5.72 & -2.53 & -6.44 & 1.32 \\ 
  \emph{Ulmus glabra} & -17.95 & -25.56 & -10.49 & -8.23 & -17.95 & 1.42 & -0.95 & -5.10 & 3.26 \\ 
  \emph{Ulmus macrocarpa} & -16.04 & -23.67 & -8.80 & -8.41 & -18.61 & 1.26 & -0.98 & -5.12 & 3.21 \\ 
  \emph{Ulmus pumila} & -11.49 & -16.11 & -6.71 & -10.23 & -14.42 & -6.11 & -1.42 & -5.28 & 2.43 \\ 
  \emph{Ulmus parvifolia} & -15.06 & -19.83 & -10.22 & -8.70 & -12.86 & -4.46 & -2.26 & -6.03 & 1.52 \\ 
  \emph{Ulmus laevis} & -14.85 & -25.33 & -4.77 & -7.03 & -17.89 & 3.92 & -1.09 & -5.44 & 3.11 \\ 
  \emph{Ulmus americana} & -10.02 & -20.18 & 0.50 & -7.25 & -17.66 & 3.23 & -1.08 & -5.43 & 3.28 \\ 
  \emph{Ulmus villosa} & -15.17 & -19.92 & -10.37 & -11.93 & -16.26 & -7.68 & -1.64 & -5.41 & 2.16 \\ 
  \emph{Caragana pygmaea} & -6.71 & -18.99 & 6.14 & -7.62 & -17.95 & 2.61 & -1.00 & -5.50 & 3.49 \\ 
  \emph{Robinia pseudoacacia} & -6.86 & -17.21 & 3.40 & -0.60 & -5.78 & 4.61 & -0.75 & -5.06 & 3.72 \\ 
  \emph{Amorpha fruticosa} & -2.88 & -13.18 & 7.81 & -6.49 & -15.80 & 3.00 & -0.89 & -5.06 & 3.46 \\ 
  \emph{Cercis chinensis} & -1.56 & -11.77 & 9.06 & -7.60 & -17.89 & 2.95 & -1.74 & -6.21 & 2.63 \\ 
  \emph{Cercis canadensis} & -6.58 & -16.76 & 3.84 & -7.05 & -17.67 & 3.67 & -1.16 & -5.52 & 3.17 \\ 
  \emph{Tilia japonica} & -4.13 & -14.09 & 6.20 & -7.27 & -17.82 & 3.35 & -0.92 & -5.13 & 3.38 \\ 
  \emph{Tilia cordata} & -4.07 & -8.65 & 0.70 & -10.14 & -14.18 & -6.13 & -2.80 & -6.50 & 0.72 \\ 
  \emph{Tilia dasystyla} & -11.94 & -21.87 & -2.13 & -7.08 & -17.24 & 3.43 & -0.97 & -5.30 & 3.48 \\ 
  \emph{Tilia platyphyllos} & -6.54 & -16.83 & 3.82 & -7.15 & -17.61 & 3.34 & -0.95 & -5.30 & 3.35 \\ 
  \emph{Hibiscus syriacus} & -0.59 & -11.02 & 10.10 & -7.26 & -17.56 & 3.34 & -1.01 & -5.34 & 3.55 \\ 
  \emph{Aesculus flava} & -16.49 & -27.49 & -6.45 & -6.64 & -16.97 & 4.11 & -1.19 & -5.56 & 3.16 \\ 
  \emph{Aesculus parviflora} & -12.49 & -22.79 & -2.37 & -7.72 & -18.11 & 3.24 & -2.03 & -6.34 & 2.29 \\ 
  \emph{Aesculus hippocastanum} & -6.13 & -14.54 & 2.27 & 0.38 & -5.66 & 6.53 & -1.15 & -5.18 & 2.96 \\ 
  \emph{Toona sinensis} & -4.10 & -14.20 & 6.30 & -7.25 & -17.75 & 3.20 & -1.17 & -5.46 & 3.17 \\ 
  \emph{Orixa japonica} & -6.50 & -16.39 & 3.67 & -7.19 & -18.04 & 3.52 & -1.13 & -5.59 & 3.42 \\ 
  \emph{Ptelea trifoliata} & -5.10 & -15.40 & 5.25 & -7.18 & -17.65 & 3.19 & -1.13 & -5.31 & 3.00 \\ 
  \emph{Ribes divaricatum} & -13.67 & -26.45 & -1.51 & -7.57 & -17.79 & 2.87 & -1.34 & -5.84 & 3.10 \\ 
  \emph{Ribes glaciale} & -6.68 & -16.53 & 3.71 & -7.52 & -18.12 & 3.24 & -0.93 & -5.24 & 3.38 \\ 
  \emph{Ribes alpinum} & -10.17 & -22.66 & 2.37 & -7.55 & -18.28 & 2.77 & -1.09 & -5.59 & 3.36 \\ 
  \emph{Hamamelis virginiana} & -8.61 & -21.51 & 4.26 & -10.91 & -20.31 & -2.12 & -1.71 & -6.07 & 2.63 \\ 
  \emph{Hamamelis vernalis} & -14.59 & -24.87 & -4.51 & -6.98 & -17.43 & 3.98 & -0.99 & -5.36 & 3.38 \\ 
  \emph{Hamamelis japonica} & -9.15 & -19.59 & 1.08 & -7.27 & -17.57 & 2.99 & -1.06 & -5.29 & 3.44 \\ 
  \emph{Sinowilsonia henryi} & -5.24 & -15.18 & 4.75 & -7.49 & -18.03 & 2.98 & -1.18 & -5.49 & 3.06 \\ 
  \emph{Corylopsis sinensis} & -8.51 & -21.07 & 3.86 & -7.55 & -18.23 & 3.10 & -1.28 & -5.87 & 3.26 \\ 
  \emph{Corylopsis spicata} & -9.18 & -22.03 & 3.41 & -7.43 & -17.82 & 3.06 & -1.39 & -5.73 & 2.99 \\ 
  \emph{Liquidambar styraciflua} & -11.04 & -21.08 & -0.87 & -7.33 & -17.56 & 2.99 & -1.53 & -5.94 & 2.77 \\ 
  \emph{Liquidambar orientalis} & -1.35 & -11.39 & 9.46 & -7.50 & -17.89 & 2.68 & -1.10 & -5.42 & 3.36 \\ 
  \emph{Cercidiphyllum japonicum} & -9.68 & -22.12 & 2.60 & -7.37 & -17.85 & 2.64 & -1.21 & -5.57 & 3.14 \\ 
  \emph{Cercidiphyllum magnificum} & -10.65 & -20.75 & -0.52 & -7.40 & -18.14 & 3.44 & -1.26 & -5.51 & 3.06 \\ 
  \emph{Parrotia persica} & -9.39 & -21.92 & 2.89 & -7.29 & -17.70 & 3.04 & -0.96 & -5.35 & 3.61 \\ 
  \emph{Paeonia rockii} & -7.59 & -19.89 & 4.62 & -7.50 & -17.90 & 2.67 & -1.11 & -5.54 & 3.45 \\ 
  \emph{Syringa villosa} & -7.83 & -20.07 & 4.59 & -7.56 & -17.94 & 3.00 & -0.96 & -5.47 & 3.67 \\ 
  \emph{Syringa vulgaris} & -6.30 & -16.36 & 3.99 & -1.65 & -6.09 & 2.84 & -1.10 & -5.37 & 3.20 \\ 
  \emph{Syringa josikaea} & -6.60 & -19.15 & 6.33 & -7.45 & -18.02 & 2.72 & -1.06 & -5.44 & 3.39 \\ 
  \emph{Syringa reticulata} & -6.89 & -19.47 & 5.72 & -7.39 & -17.73 & 3.23 & -0.99 & -5.25 & 3.46 \\ 
  \emph{Ligustrum tschonoskii} & -6.63 & -19.20 & 6.28 & -7.60 & -18.30 & 3.08 & -0.97 & -5.36 & 3.50 \\ 
  \emph{Forsythia ovata} & -6.77 & -19.55 & 5.87 & -7.44 & -17.72 & 2.73 & -1.05 & -5.50 & 3.29 \\ 
  \emph{Forsythia suspensa} & -6.12 & -16.36 & 4.09 & -7.35 & -17.43 & 3.19 & -0.91 & -5.26 & 3.49 \\ 
  \emph{Cephalanthus occidentalis} & -4.96 & -14.92 & 5.38 & -7.24 & -17.89 & 3.24 & -1.20 & -5.44 & 3.08 \\ 
  \emph{Viburnum buddleifolium} & -14.54 & -27.32 & -2.15 & -6.42 & -16.75 & 4.49 & -0.33 & -4.81 & 4.32 \\ 
  \emph{Viburnum carlesii} & -8.95 & -21.76 & 3.97 & -7.51 & -18.21 & 3.24 & -1.24 & -5.64 & 3.15 \\ 
  \emph{Viburnum cassinoides} & -3.71 & -11.38 & 3.89 & -5.17 & -11.27 & 1.16 & -1.47 & -5.57 & 2.67 \\ 
  \emph{Viburnum lantanoides} & -9.24 & -16.57 & -1.67 & -8.47 & -14.62 & -2.48 & -1.61 & -5.72 & 2.45 \\ 
  \emph{Viburnum plicatum} & -12.20 & -25.12 & 0.40 & -6.78 & -16.97 & 3.77 & -0.68 & -5.11 & 3.94 \\ 
  \emph{Viburnum opulus} & -6.92 & -17.13 & 3.34 & -7.13 & -17.59 & 3.54 & -1.07 & -5.31 & 3.32 \\ 
  \emph{Viburnum betulifolium} & -9.67 & -21.96 & 2.29 & -7.35 & -18.04 & 2.93 & -1.35 & -5.71 & 3.06 \\ 
  \emph{Weigela coraeensis} & -6.56 & -16.50 & 3.63 & -7.28 & -17.73 & 3.12 & -0.97 & -5.39 & 3.55 \\ 
  \emph{Weigela florida} & -7.68 & -20.09 & 4.64 & -7.44 & -17.87 & 2.82 & -1.02 & -5.43 & 3.48 \\ 
  \emph{Weigela maximowiczii} & -8.92 & -19.08 & 1.20 & -7.17 & -17.88 & 3.39 & -0.93 & -5.25 & 3.48 \\ 
  \emph{Heptacodium miconioides} & -6.82 & -19.04 & 5.69 & -7.43 & -18.06 & 3.16 & -1.06 & -5.51 & 3.45 \\ 
  \emph{Symphoricarpos albus} & -9.52 & -19.78 & 0.77 & -3.21 & -8.44 & 2.01 & -0.99 & -5.31 & 3.30 \\ 
  \emph{Lonicera maximowiczii} & -7.39 & -19.31 & 5.44 & -7.68 & -18.06 & 2.94 & -0.99 & -5.42 & 3.58 \\ 
  \emph{Lonicera alpigena} & -7.78 & -20.33 & 4.51 & -7.44 & -18.17 & 3.04 & -0.91 & -5.30 & 3.73 \\ 
  \emph{Lonicera canadensis} & -5.93 & -18.81 & 7.24 & -6.98 & -14.79 & 0.81 & -1.10 & -5.56 & 3.39 \\ 
  \emph{Lonicera caerulea} & -8.21 & -21.52 & 5.31 & -7.69 & -18.25 & 2.78 & -1.07 & -5.56 & 3.48 \\ 
  \emph{Eleutherococcus sieboldianus} & -5.41 & -15.51 & 4.71 & -7.38 & -17.99 & 3.12 & -0.93 & -5.20 & 3.20 \\ 
  \emph{Eleutherococcus setchuenensis} & -7.57 & -20.49 & 5.35 & -7.35 & -17.75 & 2.96 & -1.11 & -5.57 & 3.36 \\ 
  \emph{Eleutherococcus senticosus} & -5.49 & -15.94 & 5.09 & -7.28 & -17.96 & 3.46 & -0.85 & -5.16 & 3.48 \\ 
  \emph{Ilex mucronata} & -3.79 & -11.32 & 3.87 & -6.56 & -12.55 & -0.46 & -1.95 & -6.21 & 2.13 \\ 
  \emph{Rhododendron prinophyllum} & -8.09 & -20.84 & 4.67 & -9.26 & -18.33 & -0.45 & -1.15 & -5.60 & 3.14 \\ 
  \emph{Rhododendron canadense} & -2.62 & -12.72 & 7.74 & -7.23 & -17.42 & 3.14 & -0.97 & -5.29 & 3.32 \\ 
  \emph{Rhododendron dauricum} & -6.80 & -19.00 & 5.68 & -7.19 & -17.73 & 3.19 & -0.78 & -5.16 & 3.84 \\ 
  \emph{Rhododendron mucronulatum} & -7.40 & -20.33 & 5.50 & -7.54 & -18.22 & 3.36 & -0.93 & -5.24 & 3.42 \\ 
  \emph{Kalmia angustifolia} & -7.32 & -19.73 & 5.25 & -12.11 & -20.86 & -4.08 & -1.88 & -6.36 & 2.60 \\ 
  \emph{Vaccinium myrtilloides} & -5.98 & -18.81 & 6.75 & -7.41 & -15.38 & 0.43 & -0.96 & -5.29 & 3.52 \\ 
  \emph{Lyonia ligustrina} & -8.15 & -21.11 & 4.80 & -10.43 & -20.28 & -1.38 & -1.13 & -5.67 & 3.42 \\ 
  \emph{Nyssa sylvatica} & -8.31 & -20.67 & 4.49 & -9.52 & -18.51 & -0.80 & -1.39 & -5.88 & 3.14 \\ 
  \emph{Cornus alba} & -6.41 & -16.04 & 3.37 & 1.92 & -5.09 & 9.28 & -1.53 & -5.67 & 2.62 \\ 
  \emph{Cornus kousa} & -4.68 & -14.61 & 5.57 & -7.39 & -17.85 & 2.97 & -1.23 & -5.54 & 3.10 \\ 
  \emph{Cornus mas} & -6.61 & -16.98 & 3.82 & -3.77 & -10.09 & 2.70 & -1.00 & -5.49 & 3.49 \\ 
  \emph{Hydrangea arborescens} & -10.58 & -20.66 & -0.37 & -7.14 & -17.55 & 3.55 & -1.00 & -5.24 & 3.33 \\ 
  \emph{Hydrangea involucrata} & -10.15 & -20.14 & -0.07 & -7.34 & -18.13 & 3.41 & -1.09 & -5.52 & 3.30 \\ 
  \emph{Hydrangea serrata} & -9.89 & -22.57 & 2.80 & -7.40 & -17.99 & 3.12 & -1.05 & -5.61 & 3.29 \\ 
  \emph{Deutzia gracilis} & -7.60 & -20.50 & 4.96 & -7.50 & -18.00 & 2.96 & -0.97 & -5.22 & 3.39 \\ 
  \emph{Deutzia scabra} & -5.07 & -15.20 & 5.25 & -7.38 & -17.61 & 2.97 & -0.96 & -5.33 & 3.45 \\ 
  \emph{Decaisnea fargesii} & -9.88 & -22.56 & 2.67 & -7.29 & -18.05 & 3.42 & -1.04 & -5.49 & 3.53 \\ 
  \emph{Berberis dielsiana} & -6.59 & -19.04 & 6.31 & -7.55 & -18.15 & 3.04 & -1.03 & -5.46 & 3.38 \\ 
  \emph{Liriodendron tulipifera} & -13.64 & -26.59 & -1.43 & -7.48 & -17.99 & 2.87 & -1.54 & -6.05 & 2.97 \\ 
  \emph{Acer pseudoplatanus} & -9.70 & -16.44 & -2.92 & -9.89 & -12.97 & -6.82 & -1.04 & -4.56 & 2.52 \\ 
  \emph{Acer saccharinum} & -7.59 & -11.67 & -3.57 & -5.92 & -14.96 & 3.57 & -1.10 & -5.65 & 3.47 \\ 
  \emph{Acer rubrum} & -16.72 & -23.81 & -9.65 & -0.57 & -5.33 & 4.18 & -0.11 & -3.91 & 3.89 \\ 
  \emph{Acer barbinerve} & -8.41 & -20.80 & 4.09 & -7.50 & -17.88 & 3.24 & -1.09 & -5.44 & 3.39 \\ 
  \emph{Acer negundo} & -11.74 & -22.34 & -1.06 & -5.75 & -15.29 & 4.11 & -1.25 & -5.65 & 3.01 \\ 
  \emph{Acer pensylvanicum} & -8.87 & -16.22 & -1.40 & -7.87 & -13.96 & -1.91 & -2.06 & -6.16 & 1.94 \\ 
  \emph{Acer platanoides} & -9.84 & -20.42 & 0.22 & -7.16 & -17.34 & 2.95 & -1.08 & -5.44 & 3.22 \\ 
  \emph{Acer campestre} & -10.97 & -21.18 & -0.77 & -7.11 & -17.69 & 3.50 & -0.95 & -5.30 & 3.32 \\ 
  \emph{Acer tataricum} & -7.79 & -18.03 & 2.46 & -5.40 & -14.64 & 4.07 & -1.15 & -5.40 & 3.14 \\ 
  \emph{Acer ginnala} & -7.24 & -19.58 & 5.24 & -7.49 & -18.33 & 3.43 & -1.12 & -5.68 & 3.44 \\ 
  \emph{Amelanchier laevis} & -8.10 & -21.11 & 5.19 & -7.37 & -17.58 & 3.14 & -1.20 & -5.72 & 3.31 \\ 
  \emph{Amelanchier florida} & -8.42 & -18.53 & 1.88 & -7.19 & -17.84 & 3.50 & -0.98 & -5.22 & 3.36 \\ 
  \emph{Buddleja davidii} & -6.63 & -18.75 & 5.87 & -7.44 & -17.88 & 3.29 & -0.89 & -5.27 & 3.52 \\ 
  \emph{Buddleja alternifolia} & -6.14 & -18.70 & 6.71 & -7.50 & -17.98 & 2.95 & -0.99 & -5.35 & 3.43 \\ 
  \emph{Buddleja albiflora} & -6.94 & -19.44 & 5.67 & -7.57 & -17.97 & 2.86 & -0.92 & -5.35 & 3.61 \\ 
  \emph{Celtis laevigata} & -12.78 & -22.93 & -2.58 & -7.23 & -17.80 & 3.21 & -1.47 & -5.78 & 2.81 \\ 
  \emph{Celtis occidentalis} & -6.21 & -18.44 & 6.36 & -7.09 & -17.45 & 3.03 & -1.26 & -5.69 & 3.24 \\ 
  \emph{Celtis caucasica} & -10.47 & -20.58 & -0.58 & -7.13 & -17.53 & 3.27 & -1.10 & -5.50 & 3.24 \\ 
  \emph{Cladrastis lutea} & -12.14 & -22.62 & -1.82 & -7.49 & -17.74 & 3.05 & -1.87 & -6.32 & 2.42 \\ 
  \emph{Elaeagnus ebbingei} & -7.84 & -20.50 & 4.78 & -7.59 & -18.36 & 3.17 & -0.94 & -5.27 & 3.50 \\ 
  \emph{Fagus crenata} & -9.96 & -20.36 & 0.39 & -7.46 & -17.89 & 2.82 & -1.39 & -5.68 & 2.90 \\ 
  \emph{Fagus engleriana} & -10.08 & -20.42 & -0.09 & -7.30 & -17.89 & 3.21 & -1.29 & -5.55 & 3.13 \\ 
  \emph{Fagus grandifolia} & -12.35 & -20.21 & -4.58 & -9.72 & -16.07 & -3.46 & -3.14 & -7.46 & 0.95 \\ 
  \emph{Fagus orientalis} & -14.05 & -24.38 & -3.82 & -7.67 & -17.96 & 2.80 & -2.12 & -6.56 & 2.13 \\ 
  \emph{Fagus sylvatica} & -14.21 & -15.81 & -12.65 & -2.65 & -4.64 & -0.66 & -9.51 & -12.05 & -7.03 \\ 
  \emph{Fraxinus excelsior} & -8.32 & -18.52 & 1.93 & -8.54 & -17.44 & 0.30 & -1.54 & -5.89 & 2.75 \\ 
  \emph{Fraxinus ornus} & -16.69 & -27.23 & -6.40 & -6.19 & -16.38 & 4.61 & -0.79 & -5.22 & 3.63 \\ 
  \emph{Fraxinus nigra} & -8.33 & -21.58 & 4.78 & -11.81 & -20.23 & -3.87 & -1.66 & -6.24 & 2.85 \\ 
  \emph{Fraxinus pennsylvanica} & -5.37 & -15.49 & 5.02 & -6.09 & -15.47 & 3.72 & -1.14 & -5.50 & 3.25 \\ 
  \emph{Fraxinus americana} & -7.61 & -12.04 & -3.03 & -5.48 & -14.63 & 4.13 & -1.11 & -5.73 & 3.57 \\ 
  \emph{Fraxinus latifolia} & -10.54 & -20.86 & -0.45 & -7.26 & -17.74 & 3.42 & -1.68 & -6.06 & 2.61 \\ 
  \emph{Fraxinus chinensis} & -7.27 & -17.55 & 3.05 & -6.59 & -15.87 & 3.19 & -0.98 & -5.29 & 3.43 \\ 
  \emph{Juglans cinerea} & -8.76 & -19.59 & 2.18 & -6.49 & -15.82 & 2.86 & -1.05 & -5.29 & 3.25 \\ 
  \emph{Juglans ailantifolia} & -9.49 & -20.35 & 1.28 & -6.42 & -15.56 & 2.82 & -1.17 & -5.49 & 3.22 \\ 
  \emph{Parrotiopsis jaquemontiana} & -5.61 & -15.90 & 4.64 & -7.37 & -17.60 & 2.86 & -1.06 & -5.36 & 3.28 \\ 
  \emph{Pyrus ussuriensis} & -8.49 & -21.12 & 4.01 & -7.50 & -17.79 & 3.05 & -1.01 & -5.40 & 3.50 \\ 
  \emph{Pyrus elaeagnifolia} & -9.48 & -19.57 & 0.58 & -7.28 & -18.02 & 3.53 & -1.07 & -5.26 & 3.22 \\ 
  \emph{Quercus faginea} & -18.47 & -33.56 & -5.41 & -11.03 & -17.86 & -4.45 & -3.11 & -7.94 & 1.24 \\ 
  \emph{Quercus bicolor} & -8.94 & -19.24 & 1.46 & -6.42 & -15.74 & 3.46 & -1.21 & -5.63 & 3.23 \\ 
  \emph{Quercus alba} & -9.45 & -22.53 & 3.34 & -11.13 & -20.45 & -2.23 & -1.97 & -6.64 & 2.55 \\ 
  \emph{Quercus coccifera} & -18.01 & -32.31 & -4.98 & -8.90 & -18.92 & 0.86 & -1.25 & -5.73 & 3.27 \\ 
  \emph{Quercus rubra} & -16.88 & -23.01 & -10.62 & -10.78 & -14.60 & -6.94 & -2.17 & -6.10 & 1.72 \\ 
  \emph{Quercus ellipsoidalis} & -13.69 & -25.01 & -2.52 & -7.55 & -17.79 & 2.46 & -1.22 & -5.78 & 3.27 \\ 
  \emph{Quercus velutina} & -9.87 & -23.30 & 3.02 & -11.53 & -21.17 & -2.22 & -1.60 & -6.14 & 2.87 \\ 
  \emph{Quercus shumardii} & -12.36 & -22.59 & -2.46 & -7.18 & -17.39 & 3.45 & -1.31 & -5.42 & 3.02 \\ 
  \emph{Quercus ilex} & -22.78 & -36.98 & -10.07 & -18.72 & -25.67 & -12.04 & -1.40 & -5.94 & 3.15 \\ 
  \emph{Quercus petraea} & -14.19 & -21.05 & -7.52 & -12.70 & -16.43 & -9.14 & -1.64 & -5.91 & 2.50 \\ 
  \emph{Quercus pubescens} & -5.83 & -19.07 & 7.80 & -15.90 & -26.67 & -6.44 & -0.98 & -5.62 & 3.61 \\ 
  \emph{Quercus robur} & -10.11 & -16.66 & -3.51 & -11.38 & -15.34 & -7.53 & 1.99 & -1.51 & 5.79 \\ 
  \emph{Rhamnus cathartica} & -4.04 & -11.63 & 3.79 & -11.39 & -21.57 & -1.84 & -1.89 & -6.25 & 2.37 \\ 
  \emph{Rhamnus alpina} & -13.28 & -23.56 & -3.29 & -6.84 & -17.38 & 3.86 & -0.92 & -5.18 & 3.57 \\ 
  \emph{Rhamnus frangula} & -7.45 & -20.31 & 5.55 & -9.97 & -19.23 & -0.83 & -1.15 & -5.51 & 3.37 \\ 
  \emph{Salix gracilistyla} & -7.77 & -20.03 & 5.07 & -7.52 & -17.74 & 2.88 & -0.98 & -5.36 & 3.50 \\ 
  \emph{Salix smithiana} & -4.74 & -11.00 & 1.55 & -4.61 & -10.32 & 0.93 & -2.27 & -6.68 & 2.05 \\ 
  \emph{Salix repens} & -11.68 & -21.88 & -1.57 & -7.11 & -17.62 & 3.43 & -1.11 & -5.42 & 3.22 \\ 
  \emph{Sambucus tigranii} & -10.43 & -22.84 & 1.55 & -7.40 & -17.95 & 3.27 & -1.05 & -5.56 & 3.38 \\ 
  \emph{Sambucus pubens} & -10.68 & -23.03 & 1.36 & -7.42 & -17.88 & 3.17 & -1.03 & -5.39 & 3.43 \\ 
  \emph{Sorbus aucuparia} & -8.39 & -20.03 & 3.44 & -6.35 & -11.41 & -1.50 & -1.11 & -4.81 & 2.58 \\ 
  \emph{Sorbus torminalis} & -10.89 & -21.41 & -0.75 & -8.01 & -18.33 & 2.39 & -1.09 & -5.06 & 2.89 \\ 
  \emph{Sorbus aria} & -13.07 & -23.69 & -2.74 & -6.98 & -17.30 & 3.44 & -1.08 & -5.35 & 3.24 \\ 
  \emph{Sorbus decora} & -8.57 & -18.81 & 1.94 & -7.34 & -17.48 & 2.93 & -1.09 & -5.39 & 3.30 \\ 
  \emph{Sorbus commixta} & -7.51 & -17.66 & 2.70 & -7.38 & -17.75 & 3.10 & -1.01 & -5.41 & 3.50 \\ 
  \emph{Spirea alba} & -5.95 & -18.47 & 6.94 & -6.51 & -14.74 & 1.51 & -1.05 & -5.49 & 3.46 \\ 
  \emph{Stachyurus praecox} & -7.90 & -20.29 & 4.77 & -7.55 & -17.73 & 2.86 & -1.05 & -5.54 & 3.43 \\ 
  \emph{Stachyurus sinensis} & -7.56 & -18.17 & 3.53 & -7.03 & -17.41 & 3.35 & -0.99 & -5.43 & 3.44 \\ 
  \hline
\label{tab:tablesupp2}
\end{longtable}
\endgroup \clearpage \pagebreak 


% latex table generated in R 4.0.4 by xtable 1.8-4 package
% Sat Jun 15 02:27:55 2024
\begingroup\footnotesize
\begin{longtable}{lrrrrrrr}
\caption{Model parameters estimated for 158 tree species not affected by polytomies in the phylogenetic tree, including mean, standard deviation (sd), 2.5\%, 50\%, and 97.5\% uncertainty intervals (z-scored model, thus predictors are directly comparable to one another), alongside model diagnostics. Note that most species from well-represented genera (e.g., Quercus, Fagus, Acer), are absent from this analysis.} \\ 
  \hline
\emph{Parameter} & \emph{mean} & \emph{sd} & \emph{2.5\%} & \emph{50\%} & \emph{97.5\%} & \emph{$n_{eff}$} & \emph{Rhat} \\ 
  \hline
\endhead
\hline
\multicolumn{10}{l}{\footnotesize Continued on next page}
\endfoot
\endlastfoot
 \hline
\emph{$\mu_{\alpha}$} & 30.01 & 3.75 & 22.73 & 30.00 & 37.26 & 12779.08 & 1.00 \\ 
  \emph{$\mu_{\beta_{force}}$} & -5.98 & 1.82 & -9.54 & -6.00 & -2.28 & 3683.05 & 1.00 \\ 
  \emph{$\mu_{\beta_{chill}}$} & -6.20 & 2.09 & -10.28 & -6.22 & -1.90 & 7016.47 & 1.00 \\ 
  \emph{$\mu_{\beta_{photo}}$} & -1.30 & 0.89 & -3.05 & -1.31 & 0.45 & 2898.12 & 1.00 \\ 
  \emph{$\lambda_{\alpha}$} & 0.64 & 0.12 & 0.38 & 0.65 & 0.85 & 4353.95 & 1.00 \\ 
  \emph{$\lambda_{\beta_{force}}$} & 0.14 & 0.12 & 0.00 & 0.11 & 0.45 & 3127.79 & 1.00 \\ 
  \emph{$\lambda_{\beta_{chill}}$} & 0.53 & 0.16 & 0.19 & 0.54 & 0.82 & 2083.87 & 1.00 \\ 
  \emph{$\lambda_{\beta_{photo}}$} & 0.58 & 0.28 & 0.04 & 0.61 & 0.98 & 193.31 & 1.02 \\ 
  \emph{$\sigma_{\alpha}$} & 14.01 & 1.35 & 11.66 & 13.91 & 16.90 & 7157.78 & 1.00 \\ 
  \emph{$\sigma_{\beta_{force}}$} & 8.17 & 1.01 & 6.42 & 8.09 & 10.37 & 1490.78 & 1.00 \\ 
  \emph{$\sigma_{\beta_{chill}}$} & 6.87 & 0.86 & 5.35 & 6.82 & 8.67 & 2211.74 & 1.00 \\ 
  \emph{$\sigma_{\beta_{photo}}$} & 2.44 & 0.41 & 1.71 & 2.41 & 3.33 & 1045.71 & 1.01 \\ 
  \emph{$\sigma_y$} & 12.53 & 0.19 & 12.17 & 12.52 & 12.90 & 11427.98 & 1.00 \\ 
  \hline
\label{tab:modelnopolyt}
\end{longtable}
\endgroup \clearpage \pagebreak 

\section{Supplementary Figures}

\begin{figure}
  \begin{center}
  \includegraphics[width=16cm]{figures/Figs_SuppMat_1.pdf}
  \caption{\textbf{Phenological sensitivity to three environmental cues across 191 woody species estimated by a Phylogenetic Mixed Model.} The environmental cues are chilling (a), forcing (b) and photoperiod (c) measured as change in days to budburst per standardized unit ($z$-transformation) of the cues. The database used to fit the Phylogenetic Mixed Model comprised 44 studies, 191 species (all named in this version) and 2940 observations. The same phylogenetic tree is shown in each panel, colored according to an estimation of ancestral character states, being the states at the tips the species sensitivities to a cue, as estimated by our hierarchical phylogenetic model. Species sensitivities are shown as mean values $+/-$ 50\% uncertainty intervals. Note that the color scale varies in each panel. Total tree depth is 81 My.}
    \label{fig:muplot191names}
    \end{center}
\end{figure}
\clearpage



\begin{figure}
  \begin{center}
  \includegraphics[width=16cm]{figures/Figs_SuppMat_2.pdf} 
  \caption{\textbf{Distribution of data.} Diagrams showing the distribution of data, including response (days to budburst) and predictor variables (forcing in $^\circ$C, chilling in Utah units, and photoperiod in hours of light per day) in our dataset. The upper panels show distribution of experimental treatments for forcing, chilling and photoperiod ordered vertically with respect the phylogenetic tree. Colors quantify the response variable. The lower panels show response vs. predictor variables with colors indicating the frequency at which a treatment-response is found in the dataset.}
  \label{fig:raw2ddata}
  \end{center}
\end{figure}
\clearpage


\begin{figure}
  \begin{center}
  \includegraphics[width=14cm]{figures/Figs_SuppMat_3.pdf}
  \caption{\textbf{Geographic distribution of underlying data.} Map showing locations from which woody plant samples were taken for experiments in meta-analytic dataset, which is a subset of the OSPREE dataset. Dots in the map show the locations, with different colors for each unique paper. The base hillshade map was made with Natural Earth.}  
  \label{fig:mapstudylocations}
  \end{center}
\end{figure}
\clearpage

\clearpage
\begin{figure}
  \begin{center}
  \includegraphics[width=16cm]{figures/Figs_SuppMat_4.pdf}
  \caption{\textbf{Correlations among estimated sensitivities to the environmental cues.} Diagrams compare forcing vs. chilling (a,d), forcing vs. photoperiod (b,e) and chilling vs. photoperiod (c,f). Upper panels show correlations among estimated sensitivities by the phylogenetic model and lower panels show results for the non-phylogenetic model. The database used to fit both models comprised 44 studies, 191 species and 2940 observations.}
  \label{fig:suppcorrelsens}
  \end{center}
\end{figure}


\clearpage
\begin{figure}
  \begin{center}
  \includegraphics[width=14cm]{figures/Figs_SuppMat_5.pdf}
  \caption{\textbf{Marginal plots contrasting the posterior distributions (in red) of estimated sensitivities.} Posterior distributions show sensitivities  to chilling (a), forcing (b) and photoperiod (c) against prior distributions (blue). Lower panels show marginal plots for the posterior distributions of phylogenetic parameter $\lambda$ fitted for chilling (d), forcing (e) and photoperiod (f). See \emph{Stan code} section below for details on prior and parameter specification of the models.}
  \label{fig:marginalplots}
  \end{center}
\end{figure}


\clearpage
\begin{figure}
  \begin{center}
  \includegraphics[width=14cm]{figures/Figs_SuppMat_6.pdf}
  \caption{\textbf{Estimates of sigma parameter for simulated data.} Estimates of sigma were computed using our phylogenetic model (PMM) with simulated responses to forcing and photoperiod generated from a delta model of evolution (fastBM in phytools package), with delta set to 2 for forcing and to 0.5 for photoperiod (intercept and chilling generated from lambda model with a lambda of 1).}
  \label{fig:burstmodels}
  \end{center}
\end{figure}



\clearpage
\section*{Stan code}

\subsection*{Annotated example code}
We wrote our model using the Stan programming language. The model code consists of four program blocks. Below we have included a breakdown of our code structure for a simplified example model with a single intercept and slope and annotations for the different components of our model code. Additional annotations within the Stan code are denoted by a double forward slash.\\

The Stan code begins with a function block in which the phylogenetic variance-covariance matrices are generated. Below the function defines the correlation matrix with lambda ($\lambda$) and sigma ($\sigma$) parameters, and generates a matrix of the correct size and scaling.\\

\begin{verbatim}
functions {
  //correlation matrix with lambda and sigma
  matrix lambda_vcv(matrix vcv, real lambda, real sigma){ 
  matrix[rows(vcv),cols(vcv)] local_vcv; //blank matrix - size vcv
  local_vcv = vcv * lambda; 
  
  for(i in 1:rows(local_vcv))
      local_vcv[i,i] = vcv[i,i]; 
      return(quad_form_diag(local_vcv, rep_vector(sigma, rows(vcv))));
  }
}
\end{verbatim}

The data block declares the data type and includes associated dimensions and restrictions to the data. In the below example, this includes defining the overall size of the dataset, the number of species used for partial pooling, the response and predictor variables, and finally the phylogeny. \\

\begin{verbatim}
data {
  int<lower=1> N; //sample size
  int<lower=1> n_sp; //number of species
  int<lower=1, upper=n_sp> sp[N]; //species
  vector[N] y; // response
  vector[N] x1; // predictor
  matrix[n_sp,n_sp]Vphy;     // phylogeny
}
\end{verbatim}

The parameters block defines the model parameters included in our model. In our simple model we include phylogenetic effects on both the intercept and slope, with lambda and sigma parameters for each, and their respective root trait values. This is in addition to parameters for the intercept, slope, and error.\\

\begin{verbatim}
parameters {
  real<lower=0> sigma_y; //error    
  real<lower=0, upper=1> lam_interceptsa; //lambda for the intercept       
  real<lower=0> sigma_interceptsa; //sigma for the intercept
  real<lower=0, upper=1> lam_interceptsb; //lambda for predictor    
  real<lower=0> sigma_interceptsb; //sigma for forcing
  
  vector[n_sp] b; // slope of predictor effect
  real b_z; // root trait value for predictor    
  vector[n_sp] a; // intercept
  real a_z; // root trait value for the intercept

}
\end{verbatim}

Finally the model block is where we define the model structure, sampling statements, and prior distributions for our parameters. We define the prior distribution for each of the parameters listed above.\\

\begin{verbatim}
model {
  real yhat[N]; //predicted y value from below model
       
  // phylogenetic matrices generated by function model block
  matrix[n_sp,n_sp] vcv_a;     
  matrix[n_sp,n_sp] vcv_b;    
  
  // linear regression model with one intercept and one predictor
  for(i in 1:N){
    yhat[i] = a[sp[i]] + b[sp[i]] * x1[i];
}
			     	
  vcv_a = cholesky_decompose(lambda_vcv(Vphy, lam_interceptsa, sigma_interceptsa));
  vcv_b = cholesky_decompose(lambda_vcv(Vphy, lam_interceptsb, sigma_interceptsb));
 
  a ~ multi_normal_cholesky(rep_vector(a_z,n_sp), vcv_a); 
  b ~ multi_normal_cholesky(rep_vector(b_z,n_sp), vcv_b); 
  
  y ~ normal(yhat, sigma_y);

  // Priors 
  a_z ~ normal(60, 10); 
  b_z ~ normal(0, 10);  
  
  lam_interceptsa ~ beta(1, 1);
  lam_interceptsb ~ beta(1, 1);
  
  sigma_interceptsa ~ normal(30, 20);
  sigma_interceptsb ~ normal(1, 5);
   
  sigma_y ~ normal(10, 10);
  
}
\end{verbatim}

\subsection*{Full model Stan code}
Below is the full model used to conduct our analyses. Our full model includes a single intercept and three predictor parameters, but consists of the four program blocks outlined above. Parameters listed in the parameter block are defined in the Bayesian hierarchical phylogenetic model section of the main text and in equations \ref{modely} to \ref{phybetas}.

\begin{verbatim}
functions {
  matrix lambda_vcv(matrix vcv, real lambda, real sigma){
    matrix[rows(vcv),cols(vcv)] local_vcv; 
    local_vcv = vcv * lambda;
    for(i in 1:rows(local_vcv))
      local_vcv[i,i] = vcv[i,i];
      return(quad_form_diag(local_vcv, rep_vector(sigma, rows(vcv))));
  }
}

data {
  int<lower=1> N;
  int<lower=1> n_sp;
  int<lower=1, upper=n_sp> sp[N];
  vector[N] y; 	// response
  vector[N] x1; // predictor (forcing)
  vector[N] x2; // predictor (chilling)
  vector[N] x3; // predictor (photoperiod)
  matrix[n_sp,n_sp]Vphy;  // phylogeny
}

parameters {
  real<lower=0> sigma_y;    
  real<lower=0, upper=1> lam_interceptsa;       
  real<lower=0> sigma_interceptsa;
  real<lower=0, upper=1> lam_interceptsbf;       
  real<lower=0> sigma_interceptsbf;   
  real<lower=0, upper=1> lam_interceptsbc;       
  real<lower=0> sigma_interceptsbc; 
  real<lower=0, upper=1> lam_interceptsbp;       
  real<lower=0> sigma_interceptsbp; 
  vector[n_sp] b_force; // slope of forcing effect
  real b_zf;
  vector[n_sp] b_chill; // slope of chilling effect
  real b_zc;
  vector[n_sp] b_photo; // slope of photo effect
  real b_zp;
  vector[n_sp] a; // intercept
  real a_z;
}

model {
  real yhat[N];
       
  matrix[n_sp,n_sp] vcv_a; 
  matrix[n_sp,n_sp] vcv_bf;     
  matrix[n_sp,n_sp] vcv_bc;    
  matrix[n_sp,n_sp] vcv_bp;     

  // linear regression model with one intercept and three predictors
  for(i in 1:N){
    yhat[i] = 
	  a[sp[i]] + b_force[sp[i]] * x1[i] + b_chill[sp[i]] * x2[i] + b_photo[sp[i]] * x3[i];
  }
			     	
	vcv_a = cholesky_decompose(lambda_vcv(Vphy, lam_interceptsa, sigma_interceptsa));
  vcv_bf = cholesky_decompose(lambda_vcv(Vphy, lam_interceptsbf, sigma_interceptsbf));
  vcv_bc = cholesky_decompose(lambda_vcv(Vphy, lam_interceptsbc, sigma_interceptsbc));
  vcv_bp = cholesky_decompose(lambda_vcv(Vphy, lam_interceptsbp, sigma_interceptsbp));


  a ~ multi_normal_cholesky(rep_vector(a_z,n_sp), vcv_a); 
  b_force ~ multi_normal_cholesky(rep_vector(b_zf,n_sp), vcv_bf); 
  b_chill ~ multi_normal_cholesky(rep_vector(b_zc,n_sp), vcv_bc);
  b_photo ~ multi_normal_cholesky(rep_vector(b_zp,n_sp),vcv_bp);
  
  y ~ normal(yhat, sigma_y);
  
  //Priors
  a_z ~ normal(30, 10); 
  b_zf ~ normal(-2, 10); 
  b_zc ~ normal(-2, 10);
  b_zp ~ normal(0, 5); 

  lam_interceptsa ~ beta(1, 1);
  lam_interceptsbf ~ beta(1, 1);
  lam_interceptsbc~ beta(1, 1);
  lam_interceptsbp ~ beta(1, 1);

  sigma_interceptsa ~ normal(30, 20);
  sigma_interceptsbf ~ normal(1, 5);
  sigma_interceptsbc ~ normal(1, 5);
  sigma_interceptsbp ~ normal(1, 5);
    
  sigma_y ~ normal(10, 10); 
}
\end{verbatim}



\end{document}

\documentclass[11pt]{article}
\usepackage[top=1.00in, bottom=1.0in, left=1.1in, right=1.1in]{geometry}
\usepackage{graphicx}
\usepackage{natbib}
\usepackage{amsmath}
\usepackage{gensymb}
\usepackage{lineno}
\usepackage{xcolor}
\usepackage{xr-hyper}
\usepackage[parfill]{parskip}
\usepackage{pifont}
\usepackage{amssymb}
\usepackage{array}
\usepackage{nth}
\usepackage{siunitx}

\externaldocument{..//PhenoPhylo_ms}
% Cross reference line numbers with main text: see https://github.com/temporalecologylab/labgit/wiki/LaTeX#cross-referencing-
\newcommand{\R}[1]{\label{#1}\linelabel{#1}}
\newcommand{\lr}[1]{line~\lineref{#1}}
% \usepackage{hyperref}
\setlength\parindent{0pt}


\begin{document}
Editor and reviewer comments (we provide below the full context of each review) are in \emph{italics}, while our responses are in regular text. 
%\textcolor{blue}{Comments by Lizzie (that need to be deleted) are in blue.}
\textcolor{teal}{Comments by Nacho (that need to be deleted).}
\\ % and all in-text citations generally cross-reference to the main text.

{\bf Editor's comments:} \\

% \textcolor{blue}{The editor actually gave a good hint of what they want below. We need to justify our choices (but perhaps not too much beyond that, meaning no new data or such) and deal with the `true' parameter comment.}\\

\emph{The referees’ reports seem to be quite clear. Naturally, we will need you to address all of the points raised. We draw your attention specifically to the need to provide more methodological details, and justifications of the choices made in the analyses, as well as the need to compare the modelling done to `true' parameters.}\\

We review below, point-by-point, all the changes made in responses to the reviewers' comments. In particular we have provided extensive additional methodological details, and rephrased details that were present before, but perhaps not clear enough. We also now provide additional tests to compare the phylogenetic and non-phylogenetic models using a leave-one-out approach (see Appendix SXX), provide more details on data sources and their distribution (see Supporting Table SXX, and Figures SXX-XX), and re-analyze data using a different metric of chilling (see Figures SXX-XX). We thank the reviewers for their constructive feedback.\\
%IMC - not sure we need the last bit

\emph{Please note that the reviewers have also highlighted that sharing of the code and data will be an important part in ensuring that this paper has broad impact and use for the community; therefore please do let me know if you think at this point that you will not share the data/code on publication of the work.}\\

Our labs are committed to open and reproducible science and share all possible data and code. The full dataset is already available on KNB, as well as the specific subsets requested by reviewers. Further, we have posted a GitHub repo with annotated code for both Stan models, and analyses in R of models outcomes.\\

{\bf Reviewer comments:} \\

{\bf Reviewer \#1:}\\


\emph{The manuscript ‘Phylogenetic estimates of species-level phenology improve ecological forecasting’ presents a novel methodology to model and estimate species-level phenological responses to temperature and day length. The authors have filled a large knowledge and ecological forecasting gap that has been previously overlooked in other research by using Bayesian hierarchical phylogenetic modelling – allowing the evolutionary history of the species studied to be a factor that can shape species responses to temperature and day length, rather than just using it as a corrective factor in modelling. The output of this new method will allow better ecological forecasting by considering a range of cues within a species, and the evolutionary history of the species which then allows for better predictions especially where data are sparse for species’ phenologies.}\\

\emph{Interestingly, the authors found limited responses across all species to phylogeny compared to temperature in phenological shifts. They also found that average shifts do not accurately explain species’ level responses. Finally, in applying their model, the authors found that there was an increase in variability across species phenological responses when a phylogenetic structure was employed compared to no phylogeny and this then led to decreases in estimate uncertainty for individual species' responses to temperature. This finding has broad implications for ecological forecasting.}\\

We thank the reviewer for their positive comments and are excited they see the broad value in our approach for ecological forecasting.\\

\emph{Major comments}\\
\emph{I have a concern about the generality of the data and therefore the generality of the conclusions drawn from the data. It is not clear in the methods section where the phenological data are located but of course temperate Europe and North America would be highly overrepresented in these data. It would be good to see the countries, regions or locations of the 44 studies from 33 papers which this papers data originate from, that are a subset of the OSPREE data. This location data should be made available at least in the supplementary material, but ideally in the materials and methods. Only when going further into the methods and cross-referencing Ettinger et al. 2020 is it evident where the broader OSPREE sites/papers are located and it is evident all studies are Europe/North America except ~4 in South America/Africa.}\\

%\textcolor{blue}{Add a map to the supp (I imagine you could do this quickly, but Cat could maybe help if you need it) and mention the geographical distribution in the main text methods ... something like: Across all studies in OSPREE, most studies were limited to North America (n=??) and Europe (n=??), with only XX in (insert other continents). Given our need of daily gridded data for chilling we only include studies from Europe and North America, with most of these sites in temperate areas and a few in European Mediterranean areas (see Map in Supp). Be sure to state in our response that we included ALL studies using our standardized search criteria and did not set out to be as limited as we may have ended up .... that said, we DO have some data from Mediterranean areas etc.}\\

The reviewer makes a fair point, we should have included a more graphic description of our dataset. We now include both an appendix showing the location of studies in the dataset (see Supporting Fig. SXX), a summary of the data that also accommodates a request by reviewer3 (see Supporting Table SXX), and have made the following edits to the methods in the main text (see lines XXx-XX). We would like to highlight that the geographical bias present in our dataset mirrors the existing bias in the literature. We have incorporated all studies meeting our standardized search criteria (refer to Methods, lines XX-XX), resulting in a limited yet valuable dataset that accurately represents species from temperate biomes in the Northern Hemisphere, including Mediterranean regions.\\

``In our dataset most studies come from Europe (n=37) and a few from North America (n=7). The same bias towards Europe is found across the full OSPREE dataset with less North American (n=19) than European (n=60) studies and only 3 studies located in the Southern Hemisphere. Given our need of daily gridded data for chilling we only include studies from Europe and North America, with most of these sites in temperate areas and a few in European Mediterranean areas (see Map in Supporting Fig. S6).''\\

We recognize that there are limitations to the available data and this was, in part, the motivation for devloping the model we present here. Using Bayesian methods that allow us to quantify uncertainty, and leveraging information on plant evolutionary relationships (which covary with biogeography) to generate more robust predictions.\\ 

\emph{Interestingly, phenological studies in the southern hemisphere are showing differing responses to climate than northern hemisphere studies (Chambers et al. 2013, Everingham et al 2021). And some studies have shown precipitation in the season prior to the phenological event (e.g. flowering; Everingham et al. 2021) is more correlated with shifts in flowering time than temperature especially in regions where inter-annual temperature varies much less and so the ‘chilling’ period prior to flowering is not as pronounced as the northern hemisphere and may not be a driving factor of within-species cues. Likewise, ‘leaf-out’ is a less important or not-existent phenological event in regions outside of north America, Europe and Eastern Asia as many species outside of these regions are evergreen. You suggest all species in your data “respond to all three primary cues – forcing, chilling, and photoperiod” however, how general is this response across other species outside of your study regions?}\\

%\textcolor{blue}{I would ask Dan/Faith/Mira/someone or someone to double-check these refs, but they are ALL observational so we need to politely point that out in our response (I started something below) and it wouldn't hurt to add a paragraph to the discussion about how exciting our method is and -- if we had data from more species, especially underrepresented areas -- then we could start to better understand the evolutionary history ... this would be easy to add to the paper and could make it sound all the more exciting, so much potential!}\\

We completely agree with the reviewer about the need for more data beyond the temperate zone. We used standardized meta-analytic methods to review the literature, but the outcome is data highly skewed towards angiosperms in Europe and North American temperate zones. Our analysed data does include some Mediterranean species (n=3) and we did not exclude any species based on whether they were evergreen or deciduous, though cutting experiments are strongly biased towards deciduous species. Indeed our full dataset includes a number of evergreen conifers, but they were too undersampled as species to be included (see lines XX-XX). We reviewed the references provided by the reviewer, but they are all observational and so cannot be included in our approach, which relies on experimentally manipulated cues. We seem thus to be limited by data and unfortunately we could only speculate about how transferable the responses we found are to other species. To answer the reviewer, until more experimental data is published we cannot know how general the relative importance of forcing, chilling and photoperiod would be for species in other regions. Yet, it is safe to assume that, one of our major findings---i.e., the large cross-species variability---would only be reinforced if additional species from other regions strongly differed in their relative cue importance.\\

We are now more explicit in the discussion about the limitations of current data on phenology coming from experiments and how adressing these limitations by future research is urged and raises new exciting research questions (e.g., is variability in cue sensitivity across species larger in temperate than tropical latitudes?; see lines XX-XX). \\


\emph{I believe the paper would significantly improve if this gap in the data and/or methods was addressed, ideally through the inclusion of more Southern Hemispheric or evergreen species data and the inclusion of other important phenological drivers and cues where relevant. Otherwise, a strong discussion of the transferability of this phylogenetic estimation of species-level phenology to other regions/phenological events/climatic drivers (e.g. precipitation), is required, especially in order to make the concluding impact statement the authors have drawn on such a broad scale “While we focused on spring phenology here, our approach suggests a path forward for more general forecasting of species-level climate change responses[....] Using this approach improved forecasts of phenological responses to climate change and could help anticipate impacts on critical ecosystem services from species-level shifts and thus aid mitigation and human adaption to warming.”}\\

We agree. As outlined above, we believe there is little data to address this gap currently but we now highlight in the discussion how much could be learned---both important to forecasting and to fundamentally understand the underlying evolutionary history of phenological cues--through more efforts to increase data beyond the temperate zone, especially in the Southern Hemisphere. \\

\emph{I also am concerned about the accessibility of the source code and data used to test and present the Bayesian hierarchical phylogenetic model in this study. Transparency of code does not equate to reproducibility of methods/analyses. If the authors believe that this robust new method will improve phenological forecasting predictions, their code and data should be reproducible to be implemented in future studies. As with my point above, please indicate the subset of data used in this study (locations, studies, references etc) rather than citing the original OSPREE dataset so that others can reproduce your Stan code. Likewise, making the Stan code more accessible through more comments (comments of the models - names or short description) rather than just the parts of the models ("slope" "intercept" "phylogeny") may allow other researchers to utilise your methods in Stan. Perhaps even making the source code open on a shared repository (e.g. GitHub) would also improve user accessibility in the future.}\\

%\textcolor{blue}{Make up a cute repo with the Stan code as it is now (general and simple), and a version that is better annotated and specific to this model? If you need help Deirdre knows this model well. Ailene previously posted the data, so you could ask her to help us post the updated data and maybe she can include the subset there? And/or include on GitHub....}\\

These are great points and we are happy to address them. As the reviewer noted, some of the data are already available on a federated data repository (KNB) with no limits on using the data. We have updated this entry with the full raw data and---given the reviewer's request---a subset specific to this paper is now available, together with the phylogenetic tree used to conduct analyses. Additionally, we have posted a GitHub repo -- see LINK \url{https://github.com/MoralesCastilla/PhenoPhyloMM}, 
where fully annotated code to conduct models in Stan and to analyse model outcomes in R, is available.\\

\emph{Minor comments:}\\

\emph{Abstract}\\
\emph{- First sentence to open the paper is important but here is unclear – is ‘adaptation’ referring to humans or societal adaptation to climate change? Or plant adaptation to climate change and therefore it should read “Knowledge of plants abilities to adapt to climate change hinges on accurate ecological forecasting to predict shifts in key ecosystem services such as carbon storage and biodiversity maintenance” as it isn’t the ability of plants to adapt that hinges on the ecological forecasting itself but the knowledge or quantification of this adaptation hinges on the ecological forecasting.}\\

We apologise for this lack of clarity, we were referring to societal adaptation, and we have rephrased this setence as follows: \emph{Our ability to adapt to climate change requires accurate ecological forecasting to predict shifts in key ecosystem services, such as carbon storage and biodiversity maintenance.}// %JD Not changed main text -  LW may want to rephrase this ...
\textcolor{teal}{Lizzie, confirm-edit this rephrasing and I'll change it in the main text.}\\


\emph{Introduction}\\
\emph{- Line 22: replace ‘confounded’ with ‘limited’ or ‘restricted’}\\
Done.\\

\emph{- The second and third paragraph of the introduction end on very similar concluding sentences making the introduction feel repetitive on the point that species-specific models are important. Although this is an important implication of the study, it would be good to see the concepts merged together in one paragraph or less repetition of this topic in these two paragraphs}\\
Done, we have now deleted the last sentence from the first of these two paragraphs.\\
%JD I suggest dleteing the last sentence from the first of these two paras (lines 20-21)

\emph{- Line 31: remove ‘at once’}\\
Done.\\

\emph{- Line 35: remove ‘whereas,’}\\
Edited, the sentence still needed an adversative conjunction.\\

\emph{- Line 39: remove ‘especially’}\\
Done.\\

\emph{- Line 44: elaborate on conflicting results as this statement comes across as vague. Provide some detail on the conflicting results.}\\
%\textcolor{teal}{We have some commented out text above this sentece that should be usable. Please Lizzie, could you help here?}\\
%JD I have added some details to main text - LW to smooth over
Done, we now clarify how previous results differed, see lines XXX-XX.

\emph{Results and discussion}\\
\emph{- Lines 140-156: Your results suggest that there is limited variability in species-level responses to photoperiod which is an interesting and unique result compared to previously published data. However there are limited species (and mainly those that are well studied and therefore have a better sampling of their populations across a broad geographic range) that have high variability. Could you test if variability in photoperiod is related to sampling effort or coverage? Perhaps you don’t see a signal at the species-level here as the day-length in the local environments does not vary very much between populations, especially compared to variation in chilling and forcing.}\\

We agree that this is a surprising finding, which is why we devote a substantial portion of the discussion to interpret this finding. We would like to note that our method already addresses issues with sample size (weighting and pulling estimates accordingly) and we are more explicit about it in the methods (\textcolor{blue}{see lines XXX-XXX, TO DO}). In addition, the variability in sampled photoperiods is very high--i.e., from 6h to 24h; see new Fig. SXXX in the Appendix---given our focus on experimental cutting studies. We do detect a signal of photoperiod at the species-level, but its magnitude and variability is lower for this cue than for forcing and chilling.\\

%\textcolor{blue}{I would (1) edit this section a TINY amount to mention the range of photoperiods in the treatments and (2) edit the methods a bit to say that these methods are especially robust to uneven sample sizes and variances.}\\

\emph{- Line 201: place a full stop at the end of the paragraph}\\
Done.\\

\emph{- Line 213: ‘forecasts for Acer campestre, which has only 6 observations, shift by up to 35\% in our phylogenetically informed 214 model’ – what is the direction and unit of measurement of the forecast shift, is it the accuracy improving by up to 35\% - clarify the result here}\\
We had provided more details in the Fig. caption, but now also clarify in the main text (see lines XXX).\\


\emph{Figures}\\
\emph{- Figure 1 requires some editing on the axes. Both the titles and labels need increasing in size on the x-axis (perhaps less breaks with greater font size on the axis ticks) and the list of species down the tree could be removed as it is nearly impossible to read. If it remains, it would require a larger font size and a removal of the underscore between genus\_species for each species and an italicising of the species names. It would also be good to increase the font of the family labels on the right side or utilise different colours to show the family groupings more clearly.}\\
We have done our best to accommodate the reviewer's suggestions. We would rather keep species names in the figure to facilitate tracking down species responses by zooming in. We have enlarged fonts where appropriate, italize species names, remove underscores and move family labels closer to species names.\\

\emph{- Figure 2: place lambda and sigma in the figure itself and explain in the caption – will make the figure slightly more intuitive}\\
Done.\\

\emph{- Figure 3: state the definition of the acronym PMM in the figure caption}\\
Done.\\

{\bf Reviewer \#2:}\\


\emph{In this manuscript, the authors estimated species-level responses to two major environmental cues of spring phenology, temperature and daylength by using Bayesian hierarchical phylogenetic models. They found that predicting how each species responds to a combination of cues is more important than the focus on identifying which cue is the strongest. The findings are very interesting, with the methods very novel, and the conclusions are very crucial for improving current ecosystem models for predicting future shifts in ecosystem functions under background of climate change. However, the robustness of results depended on the accuracy of quantifying chilling, forcing and daylength from an meta-analyzed experimental dataset. The research lacks discussion to overcome the limited dataset problem, and how to fit multi-species model in different geographic sites is also a problem. Besides, is this model also work on phylogenetic trees originated from genome datasets? There are multiple spelling and grammar mistakes in this manuscript. I suggest a major revision decision for this manuscript.}\\

%\textcolor{blue}{Add some text here we completely agree about the complexity of understanding forcing versus chilling ... tweak some bits of the discussion and reference those lines in a response here. We also need to clarify we calculated chilling across all studies using start dates and we did it for field chilling and experimentally-applied chilling ... }\\

We thank the reviewer for acknowledging interest and value in our work, and for pointing out issues in need for attention. We agree that understanding the intricated relationships between forcing and chilling is complicated and have now expanded in the discussion on this point (see lines XXX-XXX), as well as on data limitations.\\ 
%\textcolor{teal}{not so sure what tweaked bits of the discussion to add here. Help here would be appreciated Lizzie-Jonathan.} 
The reviewer is of course correct that our model can only be as good as the data that informs it, and is thus no different to any model fit in ecology. However, an advatage of Bayesian approach is that we are able to better accommodate uncertainty in our estimates, which might arise from multiple sources, including measurement or experimental error, and limited sample size. Critically, by partially pooling across species and weighting by phylogeny, we gain strength from species estimates that are informed by more data, such as within Betula and Fagaceae, but avoid skewing estimates for phylogenetically distant clade that may have been exposed to different selective regimes. As we mention in comments to Reviewer \#1, above, part of the motivation for devloping the model we present here was to help improve estimates when the avilable data are limited. \textcolor{teal}{we may need to add something more about advantages of our method when data are limited-add some text on uneven sample sizes and variances. Lizzie-Nacho}  %JD do we need to make changes to the main text - perhaps useful to add something more about advantages of our methoid when data are limited (perhaps if you follow LWs suggestion above to add some text on uneven sample sizes and variances  
\\
The reviewer raises another interesting point on the fitting of our model to species in different geographical areas, which can be extremely challenging when data are observational as disentangling cues, which often covary, present numerous difficulties. The data we include here are from experimental manipulations, allowing us to more easily compare treatements across datasets. Nonetheless, species with diferent geographic origins will likely have adapted in response to local climates, and thus we might expect them to respond to cues differently. Our model allows species to have individualistic respones, which we argue is one major advance of our approach, and also to have responses that can covary by phylogenetic relatedness, which might capture shared historical selective regimes.\\   

Regarding whether or not phylogenomic trees could be utilized with our method, yes, our method would potentially work with any phylogenetic tree. In our current implementation, we assume the phylogeny has branch lengths proportional to time. However, different assumptions could be made, and if the genes underlying plant responses to particular cues are known, it could be possible to estimate branch lengths directly from mutational changes along these sequences. In the absence fo such detailed gene specific data, evolutionary time provides a useful proxy for species differences. We are now explicit about this in our methods (see lines XXX-XXX). 
%JD I think we just need to clarify that we use time as a proxy for species differences, but it could be possible to change this assumption.
\textcolor{teal}{I have tried to include this idea in the methods, but please check the English there.}\\

Finally, we have conducted an exhaustive revision and corrected typos and grammar. Please see our detailed response to reviewer's comments below. \\


\emph{My two major concerns are as the following:}\\
\emph{(1) Regarding chilling, the authors stated that they estimated field chilling by Utah units. However, the Utah model was first used for agricultural crops and assumes that temperatures between 1.4 and 15.9 \degree C affect dormancy release differently. The assumption of the Utah model was against the recent findings that a wide range of temperatures (-2 to 10\degreeC) have very similar effects on dormancy releases (Baumgarten et al 2021, \url{doi: 10.1111/nph.17270}). How do you prove the Utah model is suitable for species investigated in this study? To strength the findings, more reliable and various algorithms should be included (e.g., $<$5\degree C model). Furthermore, the start date of chilling accumulation is unclear, and 33 papers where the data of this study is from often used different dates. How to deal with this problem? Another problem is why just estimated field chilling? Did all papers use natural chilling treatments rather than artificial chilling treatments?}\\

%\textcolor{blue}{Best way to deal with this if not too hard: We also could do chill portions instead of Utah chill and report those in the supp, that would be a STRONG reply and a good way to say to the editor we ran all new tests, got new data (chill portions) etc.}\\
The reviewer raises an important point here, which we address in a two-fold manner. First, we have clarified further the rationale underlying the use of estimated field chilling. We realize we may have overlooked this issue as we had discussed it in depth in a previous work \citep{ettinger2020}. Research on temperate tree phenology relies on modelled estimates of chilling---i.e., two most common metrics are Utah units and chill portions---each with limitations and underlying assumptions. These models are hypotheses ultimately affected by a lack of understanding of how chilling works---e.g., below what temperature does chilling start accummulating? Hence, it is likely that estimates of chilling are inaccurate for many species and yet, they are our best guess. We now elaborate on this issue in our discussion (see lines XXX-XXX).
\textcolor{teal}{Lizzie or Ailene, I would need help editing this response and adding any additional text on chilling to the ms (I made a first, raw attempt). This is very clear in Ettinger et al. 2020, so maybe a summary of what's there? I put this in the methods, but if you think it does not fit there an alternative would be to use a subsection in the Supp. Inf. where we place new analyses and some text justifying why we estimated chilling as we did.}\\

Second, we thank the reviewer's suggestion to include additional algorithms. We ran both our models (PMM and HMM) using chill portions instead of Utah units. The results are added as two new tables in Supporting Information (new Tables S3 and S4), showing coincident results, which strengthens our findings.\\  


\emph{(2) In some papers, forcing temperatures differed between day and night. Sometimes, the forcing temperature is always changing (e.g., gradually increased ), how to determine forcing temperature in these cases? Similarly, some experimental papers used a changing photoperiod (e.g., gradually increased daylength), how to determine the daylength in this case? All these details could not be found in Methods \& Materials.}\\

This is a great point and something we thought and worked deeply on (indeed our efforts led us to write an entire paper on the complexity and importance of this problem, see \urel{https://besjournals.onlinelibrary.wiley.com/doi/10.1111/1365-2435.14329}). We apologize this was not clear enough before and have now clarified our methods in lines XX-XX.  \\
%\textcolor{blue}{Tweak methods to make sure we're clear on this, then line reference here.}\\
\textcolor{teal}{Lizzie, I think I may need some help to edit this response in the ms too.}\\


\emph{Besides, there are some minor points:}\\

%\textcolor{blue}{A lot of the non-method points below are a touch weird to me ... for example, `suggest add some evolutionary prospectives' makes me think this person wants us to go crazy with what the phylogeny tells us; I advise against this. I suggest really minor edits for most of these points and then reference lines where we already do what they're asking ... for example the whole paragraph `Weak phylogenetic signal in photoperiod sensitivity...' is about evolutionary perspectives. }\\

\emph{Abstract:}\\
\emph{Add one or two lines for introduction of the dataset you applied for the analysis and construction of ecological models.}\\
Done, see lines XXX-XXX.\\

\emph{Introduction:}\\
\emph{Line 10. ``High variability observed in responses....". Please tell what Kind of variability?}\\
Done, see lines XXX-XXX.\\

\emph{Line 13- 15. ``Much of it, however,... few well-studied species". Not logical. Please check and revise.}\\
Done, see lines XXX-XXX.\\

\emph{Line 24-27. check and improve the expressions. Besides, in this section there is a lack of state of the art about the phylogenetic signals in plant phenophases and their relationship with climatic responses. The author should discuss it.}\\
We now provide a bit more background on phylogenetic signals of plant phenophases below, see lines XXX-XXX.
%JD I think we just have to be clear in defining 'exchangeable' (all species represent samples drawn form the same underlying distribution) and 'pooling' (species estimates are pulled towards taxa with greater sampling, for we have greater confidence in estimates)
%JD I have added more information on phylo signal in responses to cues around original lines 45-47 [LW to review]. Could expand this ...
\textcolor{teal}{please Lizzie, check JDs edits to the intro}. 
\\

\emph{What is the mechanism of shared evolutionary traits in plant phenophases, and how to predict these with climatic variations should be discussed in the introduction part.}\\
One often invoked mechanism is phylogenetic conservatism or the fact that ``the common ancestor of two sets of species cannot possess two separate evolutionary histories for the same trait'' (see lines XXX-XXX). Plant phenophases, as mentioned in the introduction can be considered as traits, and their sensitivity of such phenophases to temperature and daylight have proven to be phylogenetically conserved. We are now more explicit about how, while some previous works have explored phylogenetic signal in phenology (e.g., \cite{davies2013phylogenetic}, much of it had not focused on the phylogenetic signal of phenological sensitivity to cues, which we address in our paper (see lines XXX).
\textcolor{teal}{Jonathan, please review this bit, I gave a try but not so sure what else the reviewer wants from us}.\\
%JD I am on this!


\emph{Results and Discussion:}\\
\emph{I suggest add some evolutionary prospectives and their relationship with phenological sensitivity to three environmental cues, chilling, forcing, and photoperiod.}\\
We believe the whole section ``Phylogenetic structure of phenological cues'' and especially its third and fourth paragraphs discuss in depth the evolutionary implications and interpretation of our results. We purposely avoided overinterpreting our results but would be happy to add additional prospectives if the reviewer wishes to provide more specific guidelines of what additional points should be covered.\\

\emph{Figure 4 is not precise and should be revised to reflect the frequency distribution and future forecasting.}\\
We disagree with the reviewer in this point. The figure shows shifts in the forecasts and frequency distributions of these shifts. In addition, we provide a complementary figure (Fig. S6 in Supporting information) that specifically shows differences in forecasts. The point of the figure(s) is to compare forecasts among methods and show when their results are discrepant---i.e., underrepresented species. We would argue that igure(s) 4 and S6 serve this purpose.\\


\emph{What are the limitations of this model? The study for further research should be added in the discussion section also.}\\
We have now included a new paragraph discussing the major limitations of our model and data (e.g., the need for a dated phylogenetic tree, missrepresentation of non-temperate and Southern Hemisphere species) and suggest future research venues to address these limitations.
\textcolor{teal}{I have not yet done this, but will draft (by the end of today) a new paragraph on model-data limitations and potentialities at the end of the discussion. Please review.}.\\
%\textcolor{blue}{Could reference a new paragraph about non-temperate and Southern Hemisphere species here.}\\

\emph{Methods section:}\\
\emph{What are the criteria for choosing the citations after yielding the search from ISI Web of Science and Google Scholar?}\\
We review all citations, then included those where we could calculate chilling, forcing and photoperiod treatments, see line XX. \\

\emph{Line 260-263.``For our analysis here, ...resulting in 44 studies from 33 papers". These lines are unclear, and no parameter was discussed to extract the results. The author should revise these lines.}\\
We have clarified the paper selection criteria, which required either raw experimental data or including a figure from where data could be extracted, see lines XXX-XXX. 

\emph{Line 275. How do the Polytomies work and affect only 46 out of 191 species? Please discuss in detail. Line 287-295. These are very common and should be discussed further in one or two lines.}\\
Polytomies represent parts of the phylogenetic tree lacking resolution. These nuances are relatively common in plant phylogenies, and its presence is increased in our case where we favored increasing taxonomic representation at the cost of having a less resolved tree. While having a fully resolved tree would lead our PMM to detect even more subtle variation among congeneric species within the same polytomic genus, our approach shows how a non-resolved tree performs better than no tree at all (the HMM assumes all species are in the same polytomy). 

We now briefly discuss this issue, as suggested (see lines XXX).

\textcolor{teal}{Jonathan, could you  cite a paper that explain polytomies to back up this response?}\\
%JD I am on this!

\emph{The authors should describe the importance and significance of the Bayesian hierarchical phylogenetic model in this section.}\\
We stress the importance of our PMM and how it differs from previous approaches in lines XXX-XXX.\textcolor{teal}{TO DO!}


\emph{The authors should add a detailed description table of all studies for this research.}\\
We thank all reviewers for this suggestion. We have added a map, a table and a figure in the Supporting Information with a description of studies, their geographic distribution and the underlying data (see Table SXX, Figs. SXX, SXX).
%\textcolor{blue}{Yes! Deirdre made an awesome table for one of her papers, she might be able to help ... though we should try to evenly pass out tasks.}\\



{\bf Reviewer \#3:}\\

\emph{This study, titled “Phylogenetic estimates of species-level phenology improve ecological forecasting”, incorporated evolutionary history to study the impacts of different environmental cues (temperature and daylength) and concluded that doing so improved forecasting of plant phenology. I welcome this message yet there are some critical flaws in the methods.}\\

Thank you!\\

\emph{First, the conclusion of this study was largely based on the comparisons between the results of the phylogenetic and the non-phylogenetic (traditional) mixed models. However, doing so won't allow us to concluded that the phylogenetic models are *better / improved* than the traditional models. To prove this, one needs to compare the phylogenetic models against the "true" parameters and to compare the non-phylogenetic models against the "true" parameters, then compare which models' results are closer to the "true" parameters. This is easy for simulation but no so for empirical studies as the "true" parameters are unknown in general. However, this study used data collected from controlled experiments, which should allow the users to calculate an approximate of the "true" parameter. I did not see any visualization of the raw data, which should be the baseline of the comparisons instead of the results of the non-phylogenetic models.}\\
We thank all reviewers for a consistent suggestion about a better description of data and data sources. We have added a map, a table and a figure in the Supporting Information with a description of studies, their geographic distribution and the underlying data (see Table SXX, Figs. SXX, SXX).
%\textcolor{blue}{Some nice visualizations of the raw data in the supp would work.}\\

\emph{Second, similar as the first point, to compare the performance of the phylogenetic and the non-phylogenetic models in forecasting, one needs to compare with the "true" patterns. Given that the authors used some historical time windows, this should be doable. Again, the fact that having different results when fitting the phylogenetic and the non-phylogenetic models do not necessarily mean that the phylogenetic models are better. One needs to compare with the "true" parameters.}\\

We agree that to compare the performance of the phylogenetic and the non-phylogenetic models in forecasting, we would have needed the "true" values, which unfortunately do not exist for our data (even for historical time windows, we would lack accurate response data for most of the species in our dataset). Yet, the reviewer's comment seemed relevant enough to run some extra analyses to compare model performance. We have implemented a Leave-One-Clade-Out cross validation where we re-run models leaving one multi-species genus out at a time. First, we checked how stable model coefficients were to subsetting data. Second, we used fitted models to predict the observed response variable of the left out species. Both analyses confirmed that PMM outperforms HMM: model coefficients are more similar between subset models and full models, and subset models do a better job at predicting observed responses of left out species. We have included a new section in the Supporting information (see ``Model Cross Validation'') with full details of the new methods and results, which further confirm that the phylogenetic model outperforms the non-phylogenetic approach, at least for our dataset.\\
%\textcolor{blue}{It seems like the reviewer specifically does not want us to do simulations ... Not clear what they want though.... Perhaps subset the data and do CV using the current model and lambda=0 models? TO DISCUSS.}\\


\emph{Third, the authors set up the phylogenetic model with the phylogenetic var-covar matrix of the form $\sigma\lambda\Sigma$, which is equal to the traditional model when $\lambda=0$, and when $\lambda>0$, it will be a phylogenetic model. This method, however, indicate that it is an 'either / or' logic here. In real world, it is most likely that we have both phylogenetic and non-phylogenetic components of variations. A better way would to set an additional non-phylogenetic term along with the phylogenetic term explicitly (e.g., $\lambda \Sigma + (1-\lambda)I$). In this way, both the phylogenetic and the non-phylogenetic terms can be estimated simultaneously, and one can test whether the phylogenetic component is necessary.}\\

We agree with the reviewer about the importance of including both phylogenetic variation and variation that is independent of phylogeny in the model. In fact, our model already includes both components of variation. In Equation (1), $\sigma_e^2$ is a variance term that is completely independent of phylogeny. Furthermore, in contrast to the reviewer's statement, it is not  the case that our phylogenetic variance-covariance matrix has the form $\lambda \bm{\Sigma}$ (we omit the leading $\sigma$ term in the reviewer's first equation under the assumption that $\bm{\Sigma}$ is already a variance-covariance matrix). Rather, the phyogenetic variance-covariance matrix in Equation (4) represents a $\lambda$ transformation of a variance-covariance matrix, which is exactly equal to reviewer's suggested equation, $\lambda \bm{\Sigma} + (1 - \lambda)\sigma^2\bm{I}$ (note we have added $\sigma^2$ before $\bm{I}$ as we believe this was the reviewer's intention, given that $\sigma^2$ will generally be much larger than $1-\lambda$) . To illustrate:\\

\begin{multline} \lambda \bm{\Sigma} + (1 - \lambda) \sigma^2{\bf I} =
  \begin{bmatrix}
    \lambda\sigma^2 & \lambda\sigma_{12} & \lambda\sigma_{1,3} \\
    \lambda\sigma_{2,1} & \lambda\sigma^2 & \lambda\sigma_{2,3} \\
    \lambda\sigma_{3,1} & \lambda\sigma_{3,2} & \lambda\sigma^2 \\
  \end{bmatrix}
  +
  \begin{bmatrix}
    \sigma^2 - \lambda\sigma^2 & 0 & 0 \\
    0 & \sigma^2 - \lambda\sigma^2 & 0 \\
    0 & 0 & \sigma^2-\lambda\sigma^2 \\
  \end{bmatrix}
  \\=
  \begin{bmatrix}
    \lambda\sigma_1^2 + \sigma^2 - \lambda\sigma_1^2 & \lambda\sigma_{12} +0 & \lambda\sigma_{1,3} +0\\
    \lambda\sigma_{2,1} +0 & \lambda\sigma_2^2 + \sigma^2 - \lambda\sigma_2^2 & \lambda\sigma_{2,3} +0\\
    \lambda\sigma_{3,1} +0 & \lambda\sigma_{3,2} +0 & \lambda\sigma_3^2 + \sigma_3^2 - \lambda\sigma^2 \\
  \end{bmatrix}
  \\=
  \begin{bmatrix}
    \sigma^2 & \lambda\sigma_{12} & \lambda\sigma_{1,3}\\
    \lambda\sigma_{2,1} & \sigma^2 & \lambda\sigma_{2,3}\\
    \lambda\sigma_{3,1}& \lambda\sigma_{3,2} & \sigma^2\\
  \end{bmatrix}
  \\
\end{multline}

Assuming that the covariance terms above, $\sigma_{x,y}=\sigma_i \rho_{xy}$, where $\rho_{xy}$ is the phylogenetic correlation, this matches Equation (4) in our paper.\\
%We thank the reviewer for noticing this and have included a new section in the Supporting Information explaining how their suggested approach is mathematically equivalent to what we do (see section ``XXX'' in the Supporting Information).
%\textcolor{blue}{Get Will to write a response to this (start asking NOW and ask OFTEN so he does this by deadline). We could also add something to the supp explaining the model more.....}\\

\emph{Also, why did not include "study" as a random term here? I expect multiple measurements from the same study, right?}\\

This is a great question and something we are actively working on. Some of the major differences between studies are accounted for in how we have carefully calculated the treatments of chilling, forcing and photoperiod. Beyond that, some species in our data occur across many studies, but most occur in only one study (162 out of 191 species) making it very difficult to separate out species versus study effects. This is not a new problem (see \cite{kharouba2018} for more discussion), but a difficult one. We're currently working on a version of this model with statistical expert Michael Betancourt to try to tease out species (in a phylogenetic framework) from study using far more data and still struggling. This goal unfortunately is not possible for this dataset currently, which we now acknowledge in the methods (lines XX-XX). \\

%\textcolor{blue}{We can add a half or full sentence to methods along the lines of `Many species occurred in only one study, making it difficult to separate the effects of study and species, thus we do not include study as a separate parameter here and average over it in our model estimates.' }\\


\emph{Minor comments:}\\
\emph{L80-84: Disagree about the claim here. The models described in Ives and Helmus 2011, as well as Hadfield 2010 also allowed different estimations of the phylogenetic components of different predictors. In other words, the model described here can be fitted with those approches too.}\\
We have edited this bit as we see how it may have been confusing, by removing references to the models by Ives and Helmus and by Hadfield and making explicit that we meant how traditional phylogenetic models were concerned with phylogenetic correction. We would like to mention that we did reach out to Anthony Ives to try to code our PMM model in ``phyr'' and never could. However we would like to see these methods used more, so if the reviewer can provide code to implement the PMM models in these packages then we will happily test and include it.

%\textcolor{blue}{Hmm, we could just remove the two refs and leave Freckleton. We never got these models to work in other packages.. My last email from Tony Ives is: ``Itshould take (phylogenetically) random slopes pretty easily. Having multiple observations for the same species shouldn't be a problem; you can just include a species-level random effect. (Okay, to do this in phyr you have to code the matrices yourself, but you've already done that for stan).' but I think adding a species level random effect is not what we want... I suggest we pull the refs but reply in saying we did reach out to Tony to try to code these in phyr and never could. However we'd like to see these methods used more, so if the reviewer can provide code to implement the models we have here in these packages then we'll happily test and include it ... TO DISCUSS though a minor point.}\\

\emph{L100-108: In theory, models without considering auto-correlations should still give unbiased estimations of the mean.}\\
We did not intend to imply that HMM estimates were biased. We simply report how much coefficients shift from one model to the other, showing that shifts are not too large. This issue has been amply discussed in papers comparing coefficients among models with and without considering auto-correlations (see e.g., \cite{mauricio2009coefficient}). If the reviewer wishes, we could include some discussion of the topic but we fear doing so would divert attention. Alternatively, we now go further comparing the PMM and HMM outcomes in the new Supporting section on model cross-validation, and hope the discussion therein satisfies the reviewer. 


\emph{L174-188: Good discussion here.}\\

Thank you!\\ %\textcolor{blue}{This actually is a lovely paragraph .... including our sigma simulations.}\\

\emph{L263: What is the total number of data points?}\\
Included.\\ 

\emph{L275: With a branch length of 0? Or do you mean with a branch length of the congeneric basal node age?}\\
Thanks for spotting this, corrected.\\

\emph{Eqn 3: what is n?? Number of species? If so, you already used *j* to represent species in eqn. 1, why use a different letter here?}\\
We have now reviewed and corrected mathematical notation in our equations.\\


\emph{L308: Trait *i*? *i* was used previously for observations, can you use a different letter to be less confusion?}\\
We have now reviewed and corrected mathematical notation in our equations.\\



\bibliography{phylorefs}
\bibliographystyle{amnat}




\end{document}

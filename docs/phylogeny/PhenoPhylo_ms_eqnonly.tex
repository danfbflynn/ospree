\documentclass[11pt]{article}
\usepackage[top=1.00in, bottom=1.0in, left=1.1in, right=1.1in]{geometry}
\renewcommand{\baselinestretch}{1.1}
\usepackage{graphicx}
\usepackage{natbib}
\usepackage{gensymb}
\usepackage{amsmath}
\usepackage[utf8]{inputenc}
\usepackage{lineno}
\usepackage{xr-hyper}
% \externaldocument{limitingcues_supp}

\def\labelitemi{--}
\renewcommand{\baselinestretch}{1.2}
\parindent=0pt
\parskip=5pt

\begin{document}
For each observation $i$ of species $j$, we assumed that the timing of phenological events were generated from the following sampling distribution:
\begin{align}
  \label{modely}
  y_{i,j} \sim \mathcal{N}(\mu_j, \sigma_e^2)
\end{align}
where
\begin{align}
  \label{modelmu}
  \mu_j = \alpha_j + \beta_{chill,j} X_{chill} + \beta_{force,j} X_{force} + \beta_{photo,j} X_{photo}
\end{align}
where $\sigma_e^2$ represents random error unrelated to the phylogeny. 

Predictors $X_{chill}$, $X_{force}$, $X_{photo}$ are standardized chilling, forcing, and photoperiod, and their effects on the phenology of species $j$ are determined by parameters $\beta_{chill,j}$, $\beta_{force,j}$, $\beta_{photo,j}$, representing species' responses (or sensitivities) to each of the cues. These responses, including the species-specific intercept $\alpha_j$, are elements of the following normal random vectors:

\begin{align}
    \label{phybetas}
  \boldsymbol{\alpha} = [\alpha_1, \ldots, \alpha_n]^T & \text{ such that }
  \boldsymbol{\alpha} \sim \mathcal{N}(\mu_{\alpha},\boldsymbol{\Sigma_{\alpha}}) \\
  \boldsymbol{\beta_{chill}} =  [\beta_{1,1}, \ldots, \beta_{1,n}]^T & \text{ such that }
  \boldsymbol{\beta_{chill}} \sim \mathcal{N}(\mu_{\beta_1},\boldsymbol{\Sigma_{\beta_{chill}}}) \nonumber \\
  \boldsymbol{\beta_{force}} =  [\beta_{2,1}, \ldots, \beta_{2,n}]^T & \text{ such that }
  \boldsymbol{\beta_{force}} \sim \mathcal{N}(\mu_{\beta_2},\boldsymbol{\Sigma_{\beta_{force}}}) \nonumber \\
  \boldsymbol{\beta_{photo}} =  [\beta_{3,1}, \ldots, \beta_{3,n}]^T & \text{ such that }
  \boldsymbol{\beta_{photo}} \sim \mathcal{N}(\mu_{\beta_3},\boldsymbol{\Sigma_{\beta_{photo}}}) \nonumber
\end{align}

\noindent where the means of the multivariate normal distributions are root trait values (i.e., values of cue responses prior to evolving across a phylogenetic tree) and $\boldsymbol{\Sigma_i}$ % we need to decide for a nomenclature that is as widespread/easy to understand as possible. Other papers use bold V, or C to refer to the VCV matrix. Any suggestions are welcome
are $n \times n$ phylogenetic variance-covariance matrices of the form: \\ 
\begin{align}
  \label{phymat}
\begin{bmatrix}
  \sigma^2_i & \lambda_i \times \sigma_{i} \times \rho_{12} & \ldots & \lambda_i \times \sigma_{i} \times \rho_{1n} \\
  \lambda_i \times \sigma_i \times \rho_{21} & \sigma^2_i & \ldots & \lambda_i \times \sigma_{i} \times \rho_{2n} \\
  \vdots & \vdots & \ddots & \vdots \\
  \lambda_i \times \sigma_i \times \rho_{n1} & \lambda_i \times \sigma_i \times \rho_{n2} & \ldots & \sigma^2_i \\
\end{bmatrix}
\end{align}

\noindent where $\sigma_i^2$ is the rate of evolution across a tree for trait $i$ (here assumed to be constant along all branches), $\lambda_i$ scales branch lengths and therefore is a measure of the ``phylogenetic signal'' or extent of phylogenetic relatedness on each model parameter (i.e., $\alpha_{j}$, $\beta_{force,j}$, $\beta_{force,j}$, $\beta_{photo,j}$), and $\rho_{xy}$ is the phylogenetic correlation between species $x$ and $y$, or the fraction of the tree shared by the two species.

The above specification is equivalent to writing equation \ref{modelmu} in terms of root trait values and residuals, such that:

\begin{align}
  \label{eqfive}
  \mu_j = \mu_\alpha + \mu_{\beta_{chill}} X_{chill} + \mu_{\beta_{force}} X_{force} + \mu_{\beta_{photo}} X_{photo} + e_{\alpha_{j}} + e_{\beta_{force,j}} + e_{\beta_{chill,j}} + e_{\beta_{photo,j}}
\end{align}

\noindent where the residual phylogenetic error terms (e.g., $e_{\alpha_{j}}$) are elements of normal random vectors from multivariate normal distributions centered on $0$ with the same phylogenetic variance-covariance matrices as in equation \ref{phymat}.

\end{document}